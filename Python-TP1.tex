
% Default to the notebook output style

    


% Inherit from the specified cell style.




    
\documentclass{article}

    
    
    \usepackage{graphicx} % Used to insert images
    \usepackage{adjustbox} % Used to constrain images to a maximum size 
    \usepackage{color} % Allow colors to be defined
    \usepackage{enumerate} % Needed for markdown enumerations to work
    \usepackage{geometry} % Used to adjust the document margins
    \usepackage{amsmath} % Equations
    \usepackage{amssymb} % Equations
    \usepackage{eurosym} % defines \euro
    \usepackage[mathletters]{ucs} % Extended unicode (utf-8) support
    \usepackage[utf8x]{inputenc} % Allow utf-8 characters in the tex document
    \usepackage{fancyvrb} % verbatim replacement that allows latex
    \usepackage{grffile} % extends the file name processing of package graphics 
                         % to support a larger range 
    % The hyperref package gives us a pdf with properly built
    % internal navigation ('pdf bookmarks' for the table of contents,
    % internal cross-reference links, web links for URLs, etc.)
    \usepackage{hyperref}
    \usepackage{longtable} % longtable support required by pandoc >1.10
    \usepackage{booktabs}  % table support for pandoc > 1.12.2
    

    
    
    \definecolor{orange}{cmyk}{0,0.4,0.8,0.2}
    \definecolor{darkorange}{rgb}{.71,0.21,0.01}
    \definecolor{darkgreen}{rgb}{.12,.54,.11}
    \definecolor{myteal}{rgb}{.26, .44, .56}
    \definecolor{gray}{gray}{0.45}
    \definecolor{lightgray}{gray}{.95}
    \definecolor{mediumgray}{gray}{.8}
    \definecolor{inputbackground}{rgb}{.95, .95, .85}
    \definecolor{outputbackground}{rgb}{.95, .95, .95}
    \definecolor{traceback}{rgb}{1, .95, .95}
    % ansi colors
    \definecolor{red}{rgb}{.6,0,0}
    \definecolor{green}{rgb}{0,.65,0}
    \definecolor{brown}{rgb}{0.6,0.6,0}
    \definecolor{blue}{rgb}{0,.145,.698}
    \definecolor{purple}{rgb}{.698,.145,.698}
    \definecolor{cyan}{rgb}{0,.698,.698}
    \definecolor{lightgray}{gray}{0.5}
    
    % bright ansi colors
    \definecolor{darkgray}{gray}{0.25}
    \definecolor{lightred}{rgb}{1.0,0.39,0.28}
    \definecolor{lightgreen}{rgb}{0.48,0.99,0.0}
    \definecolor{lightblue}{rgb}{0.53,0.81,0.92}
    \definecolor{lightpurple}{rgb}{0.87,0.63,0.87}
    \definecolor{lightcyan}{rgb}{0.5,1.0,0.83}
    
    % commands and environments needed by pandoc snippets
    % extracted from the output of `pandoc -s`
    \providecommand{\tightlist}{%
      \setlength{\itemsep}{0pt}\setlength{\parskip}{0pt}}
    \DefineVerbatimEnvironment{Highlighting}{Verbatim}{commandchars=\\\{\}}
    % Add ',fontsize=\small' for more characters per line
    \newenvironment{Shaded}{}{}
    \newcommand{\KeywordTok}[1]{\textcolor[rgb]{0.00,0.44,0.13}{\textbf{{#1}}}}
    \newcommand{\DataTypeTok}[1]{\textcolor[rgb]{0.56,0.13,0.00}{{#1}}}
    \newcommand{\DecValTok}[1]{\textcolor[rgb]{0.25,0.63,0.44}{{#1}}}
    \newcommand{\BaseNTok}[1]{\textcolor[rgb]{0.25,0.63,0.44}{{#1}}}
    \newcommand{\FloatTok}[1]{\textcolor[rgb]{0.25,0.63,0.44}{{#1}}}
    \newcommand{\CharTok}[1]{\textcolor[rgb]{0.25,0.44,0.63}{{#1}}}
    \newcommand{\StringTok}[1]{\textcolor[rgb]{0.25,0.44,0.63}{{#1}}}
    \newcommand{\CommentTok}[1]{\textcolor[rgb]{0.38,0.63,0.69}{\textit{{#1}}}}
    \newcommand{\OtherTok}[1]{\textcolor[rgb]{0.00,0.44,0.13}{{#1}}}
    \newcommand{\AlertTok}[1]{\textcolor[rgb]{1.00,0.00,0.00}{\textbf{{#1}}}}
    \newcommand{\FunctionTok}[1]{\textcolor[rgb]{0.02,0.16,0.49}{{#1}}}
    \newcommand{\RegionMarkerTok}[1]{{#1}}
    \newcommand{\ErrorTok}[1]{\textcolor[rgb]{1.00,0.00,0.00}{\textbf{{#1}}}}
    \newcommand{\NormalTok}[1]{{#1}}
    
    % Define a nice break command that doesn't care if a line doesn't already
    % exist.
    \def\br{\hspace*{\fill} \\* }
    % Math Jax compatability definitions
    \def\gt{>}
    \def\lt{<}
    % Document parameters
    \title{Python-TP1}
    
    
    

    % Pygments definitions
    
\makeatletter
\def\PY@reset{\let\PY@it=\relax \let\PY@bf=\relax%
    \let\PY@ul=\relax \let\PY@tc=\relax%
    \let\PY@bc=\relax \let\PY@ff=\relax}
\def\PY@tok#1{\csname PY@tok@#1\endcsname}
\def\PY@toks#1+{\ifx\relax#1\empty\else%
    \PY@tok{#1}\expandafter\PY@toks\fi}
\def\PY@do#1{\PY@bc{\PY@tc{\PY@ul{%
    \PY@it{\PY@bf{\PY@ff{#1}}}}}}}
\def\PY#1#2{\PY@reset\PY@toks#1+\relax+\PY@do{#2}}

\expandafter\def\csname PY@tok@gd\endcsname{\def\PY@tc##1{\textcolor[rgb]{0.63,0.00,0.00}{##1}}}
\expandafter\def\csname PY@tok@gu\endcsname{\let\PY@bf=\textbf\def\PY@tc##1{\textcolor[rgb]{0.50,0.00,0.50}{##1}}}
\expandafter\def\csname PY@tok@gt\endcsname{\def\PY@tc##1{\textcolor[rgb]{0.00,0.27,0.87}{##1}}}
\expandafter\def\csname PY@tok@gs\endcsname{\let\PY@bf=\textbf}
\expandafter\def\csname PY@tok@gr\endcsname{\def\PY@tc##1{\textcolor[rgb]{1.00,0.00,0.00}{##1}}}
\expandafter\def\csname PY@tok@cm\endcsname{\let\PY@it=\textit\def\PY@tc##1{\textcolor[rgb]{0.25,0.50,0.50}{##1}}}
\expandafter\def\csname PY@tok@vg\endcsname{\def\PY@tc##1{\textcolor[rgb]{0.10,0.09,0.49}{##1}}}
\expandafter\def\csname PY@tok@m\endcsname{\def\PY@tc##1{\textcolor[rgb]{0.40,0.40,0.40}{##1}}}
\expandafter\def\csname PY@tok@mh\endcsname{\def\PY@tc##1{\textcolor[rgb]{0.40,0.40,0.40}{##1}}}
\expandafter\def\csname PY@tok@go\endcsname{\def\PY@tc##1{\textcolor[rgb]{0.53,0.53,0.53}{##1}}}
\expandafter\def\csname PY@tok@ge\endcsname{\let\PY@it=\textit}
\expandafter\def\csname PY@tok@vc\endcsname{\def\PY@tc##1{\textcolor[rgb]{0.10,0.09,0.49}{##1}}}
\expandafter\def\csname PY@tok@il\endcsname{\def\PY@tc##1{\textcolor[rgb]{0.40,0.40,0.40}{##1}}}
\expandafter\def\csname PY@tok@cs\endcsname{\let\PY@it=\textit\def\PY@tc##1{\textcolor[rgb]{0.25,0.50,0.50}{##1}}}
\expandafter\def\csname PY@tok@cp\endcsname{\def\PY@tc##1{\textcolor[rgb]{0.74,0.48,0.00}{##1}}}
\expandafter\def\csname PY@tok@gi\endcsname{\def\PY@tc##1{\textcolor[rgb]{0.00,0.63,0.00}{##1}}}
\expandafter\def\csname PY@tok@gh\endcsname{\let\PY@bf=\textbf\def\PY@tc##1{\textcolor[rgb]{0.00,0.00,0.50}{##1}}}
\expandafter\def\csname PY@tok@ni\endcsname{\let\PY@bf=\textbf\def\PY@tc##1{\textcolor[rgb]{0.60,0.60,0.60}{##1}}}
\expandafter\def\csname PY@tok@nl\endcsname{\def\PY@tc##1{\textcolor[rgb]{0.63,0.63,0.00}{##1}}}
\expandafter\def\csname PY@tok@nn\endcsname{\let\PY@bf=\textbf\def\PY@tc##1{\textcolor[rgb]{0.00,0.00,1.00}{##1}}}
\expandafter\def\csname PY@tok@no\endcsname{\def\PY@tc##1{\textcolor[rgb]{0.53,0.00,0.00}{##1}}}
\expandafter\def\csname PY@tok@na\endcsname{\def\PY@tc##1{\textcolor[rgb]{0.49,0.56,0.16}{##1}}}
\expandafter\def\csname PY@tok@nb\endcsname{\def\PY@tc##1{\textcolor[rgb]{0.00,0.50,0.00}{##1}}}
\expandafter\def\csname PY@tok@nc\endcsname{\let\PY@bf=\textbf\def\PY@tc##1{\textcolor[rgb]{0.00,0.00,1.00}{##1}}}
\expandafter\def\csname PY@tok@nd\endcsname{\def\PY@tc##1{\textcolor[rgb]{0.67,0.13,1.00}{##1}}}
\expandafter\def\csname PY@tok@ne\endcsname{\let\PY@bf=\textbf\def\PY@tc##1{\textcolor[rgb]{0.82,0.25,0.23}{##1}}}
\expandafter\def\csname PY@tok@nf\endcsname{\def\PY@tc##1{\textcolor[rgb]{0.00,0.00,1.00}{##1}}}
\expandafter\def\csname PY@tok@si\endcsname{\let\PY@bf=\textbf\def\PY@tc##1{\textcolor[rgb]{0.73,0.40,0.53}{##1}}}
\expandafter\def\csname PY@tok@s2\endcsname{\def\PY@tc##1{\textcolor[rgb]{0.73,0.13,0.13}{##1}}}
\expandafter\def\csname PY@tok@vi\endcsname{\def\PY@tc##1{\textcolor[rgb]{0.10,0.09,0.49}{##1}}}
\expandafter\def\csname PY@tok@nt\endcsname{\let\PY@bf=\textbf\def\PY@tc##1{\textcolor[rgb]{0.00,0.50,0.00}{##1}}}
\expandafter\def\csname PY@tok@nv\endcsname{\def\PY@tc##1{\textcolor[rgb]{0.10,0.09,0.49}{##1}}}
\expandafter\def\csname PY@tok@s1\endcsname{\def\PY@tc##1{\textcolor[rgb]{0.73,0.13,0.13}{##1}}}
\expandafter\def\csname PY@tok@kd\endcsname{\let\PY@bf=\textbf\def\PY@tc##1{\textcolor[rgb]{0.00,0.50,0.00}{##1}}}
\expandafter\def\csname PY@tok@sh\endcsname{\def\PY@tc##1{\textcolor[rgb]{0.73,0.13,0.13}{##1}}}
\expandafter\def\csname PY@tok@sc\endcsname{\def\PY@tc##1{\textcolor[rgb]{0.73,0.13,0.13}{##1}}}
\expandafter\def\csname PY@tok@sx\endcsname{\def\PY@tc##1{\textcolor[rgb]{0.00,0.50,0.00}{##1}}}
\expandafter\def\csname PY@tok@bp\endcsname{\def\PY@tc##1{\textcolor[rgb]{0.00,0.50,0.00}{##1}}}
\expandafter\def\csname PY@tok@c1\endcsname{\let\PY@it=\textit\def\PY@tc##1{\textcolor[rgb]{0.25,0.50,0.50}{##1}}}
\expandafter\def\csname PY@tok@kc\endcsname{\let\PY@bf=\textbf\def\PY@tc##1{\textcolor[rgb]{0.00,0.50,0.00}{##1}}}
\expandafter\def\csname PY@tok@c\endcsname{\let\PY@it=\textit\def\PY@tc##1{\textcolor[rgb]{0.25,0.50,0.50}{##1}}}
\expandafter\def\csname PY@tok@mf\endcsname{\def\PY@tc##1{\textcolor[rgb]{0.40,0.40,0.40}{##1}}}
\expandafter\def\csname PY@tok@err\endcsname{\def\PY@bc##1{\setlength{\fboxsep}{0pt}\fcolorbox[rgb]{1.00,0.00,0.00}{1,1,1}{\strut ##1}}}
\expandafter\def\csname PY@tok@mb\endcsname{\def\PY@tc##1{\textcolor[rgb]{0.40,0.40,0.40}{##1}}}
\expandafter\def\csname PY@tok@ss\endcsname{\def\PY@tc##1{\textcolor[rgb]{0.10,0.09,0.49}{##1}}}
\expandafter\def\csname PY@tok@sr\endcsname{\def\PY@tc##1{\textcolor[rgb]{0.73,0.40,0.53}{##1}}}
\expandafter\def\csname PY@tok@mo\endcsname{\def\PY@tc##1{\textcolor[rgb]{0.40,0.40,0.40}{##1}}}
\expandafter\def\csname PY@tok@kn\endcsname{\let\PY@bf=\textbf\def\PY@tc##1{\textcolor[rgb]{0.00,0.50,0.00}{##1}}}
\expandafter\def\csname PY@tok@mi\endcsname{\def\PY@tc##1{\textcolor[rgb]{0.40,0.40,0.40}{##1}}}
\expandafter\def\csname PY@tok@gp\endcsname{\let\PY@bf=\textbf\def\PY@tc##1{\textcolor[rgb]{0.00,0.00,0.50}{##1}}}
\expandafter\def\csname PY@tok@o\endcsname{\def\PY@tc##1{\textcolor[rgb]{0.40,0.40,0.40}{##1}}}
\expandafter\def\csname PY@tok@kr\endcsname{\let\PY@bf=\textbf\def\PY@tc##1{\textcolor[rgb]{0.00,0.50,0.00}{##1}}}
\expandafter\def\csname PY@tok@s\endcsname{\def\PY@tc##1{\textcolor[rgb]{0.73,0.13,0.13}{##1}}}
\expandafter\def\csname PY@tok@kp\endcsname{\def\PY@tc##1{\textcolor[rgb]{0.00,0.50,0.00}{##1}}}
\expandafter\def\csname PY@tok@w\endcsname{\def\PY@tc##1{\textcolor[rgb]{0.73,0.73,0.73}{##1}}}
\expandafter\def\csname PY@tok@kt\endcsname{\def\PY@tc##1{\textcolor[rgb]{0.69,0.00,0.25}{##1}}}
\expandafter\def\csname PY@tok@ow\endcsname{\let\PY@bf=\textbf\def\PY@tc##1{\textcolor[rgb]{0.67,0.13,1.00}{##1}}}
\expandafter\def\csname PY@tok@sb\endcsname{\def\PY@tc##1{\textcolor[rgb]{0.73,0.13,0.13}{##1}}}
\expandafter\def\csname PY@tok@k\endcsname{\let\PY@bf=\textbf\def\PY@tc##1{\textcolor[rgb]{0.00,0.50,0.00}{##1}}}
\expandafter\def\csname PY@tok@se\endcsname{\let\PY@bf=\textbf\def\PY@tc##1{\textcolor[rgb]{0.73,0.40,0.13}{##1}}}
\expandafter\def\csname PY@tok@sd\endcsname{\let\PY@it=\textit\def\PY@tc##1{\textcolor[rgb]{0.73,0.13,0.13}{##1}}}

\def\PYZbs{\char`\\}
\def\PYZus{\char`\_}
\def\PYZob{\char`\{}
\def\PYZcb{\char`\}}
\def\PYZca{\char`\^}
\def\PYZam{\char`\&}
\def\PYZlt{\char`\<}
\def\PYZgt{\char`\>}
\def\PYZsh{\char`\#}
\def\PYZpc{\char`\%}
\def\PYZdl{\char`\$}
\def\PYZhy{\char`\-}
\def\PYZsq{\char`\'}
\def\PYZdq{\char`\"}
\def\PYZti{\char`\~}
% for compatibility with earlier versions
\def\PYZat{@}
\def\PYZlb{[}
\def\PYZrb{]}
\makeatother


    % Exact colors from NB
    \definecolor{incolor}{rgb}{0.0, 0.0, 0.5}
    \definecolor{outcolor}{rgb}{0.545, 0.0, 0.0}



    
    % Prevent overflowing lines due to hard-to-break entities
    \sloppy 
    % Setup hyperref package
    \hypersetup{
      breaklinks=true,  % so long urls are correctly broken across lines
      colorlinks=true,
      urlcolor=blue,
      linkcolor=darkorange,
      citecolor=darkgreen,
      }
    % Slightly bigger margins than the latex defaults
    
    \geometry{verbose,tmargin=1in,bmargin=1in,lmargin=1in,rmargin=1in}
    
    \date{}
    \begin{document}
    
    
    \maketitle
    
    

    
    \subparagraph{TP 1 -- Informatique CM3 --}\label{tp-1-informatique-cm3}

    \section{Python @ Polytech'Lille}\label{python-polytechlille}

    Le texte de cette session de travaux pratiques est également disponible
ici:

https://github.com/ecalzavarini/python-at-polytech-lille/blob/master/Python-TP1.ipynb

    \subsection{Introduction aux commandes
Linux}\label{introduction-aux-commandes-linux}

    Avant de commencer avec notre introduction au langage de programmation
Python, nous devons familiariser avec l'environnement du système
d'explotation Linux.

La première étape est d'apprendre les commandes les plus courantes. Ces
commandes peuvent être tapées sur une console Linux.

Pour savoir comment accéder à une console sur la distribution Linux
Debian dont vous disposez, voir: https://wiki.debian.org/fr/Console ou
simplement recherchez l'application ``terminal'' sur le menu des
applications.

    \paragraph{La commande pwd:}\label{la-commande-pwd}

    La commande ``pwd'' affiche le répertoire courant, c'est à dire le
répertoire où l'on se trouve au moment où on tape la commande.

    \paragraph{La commande ls:}\label{la-commande-ls}

    La commande ``ls'' permet de lister ce que contient un répertoire. Il y
a plusieurs manières de l'utiliser :

``ls'' donne la liste brute de ce qui se trouve dans le répertoire

``ls -l'' donne la liste de ce qui se trouve dans le répertoire avec des
informations complémentaires (droits de lecture, écriture et exécution,
propriétaire, taille, date de création ou de dernière modification).

``ls -a'' liste tous (pensez au mot ``all'') les fichiers du répertoire,
y compris les fichiers cachés. Cette option est très utile lorsque l'on
se trouve dans son répertoire personnel car il contient les fichiers de
configuration de l'utilisateur dont les noms commencent généralement par
un point et seule l'option -a permet de détecter leur existence.

    \paragraph{La commande mkdir:}\label{la-commande-mkdir}

    La commande ``mkdir'' permet de créer un répertoire.

``mkdir nom\_repertoire'' crée le répertoire ayant pour nom
nom\_repertoire.

    \paragraph{La commande cd:}\label{la-commande-cd}

    La commande ``cd'' permet de se déplacer dans l'arborescence de
repertoires. Il y a plusieurs manières de l'utiliser :

``cd nom\_repertoire'' permet d'aller dans le répertoire nom\_repertoire

``cd ..'' permet de retourner dans le répertoire parent (celui qui est
au-dessus dans l'arborescence de répertoires).

    \paragraph{L'éditeur de texte emacs:}\label{luxe9diteur-de-texte-emacs}

    Pour créer ou modifier un fichier ``emacs -nw nom\_fichier.txt''

Pour sauvegarder tapez la combinaison de touches suivante: Ctrl-x
Ctrl-s.

Pour quitter emacs: Ctrl-x Ctrl-c.

    \paragraph{La commande rm:}\label{la-commande-rm}

    La commande ``rm'' permet de supprimer un fichier :

``rm nom\_fichier'' supprime le fichier nom\_fichier si il se trouve
dans le répertoire courant.

``rm --r'' supprime un répertoire et ses sous répertoires.

    \paragraph{La commande top:}\label{la-commande-top}

    La commande ``top'' affiche en continu des informations décrivant
l'activité du système. Elle permet surtout de suivre les ressources que
les processus utilisent (quantité de mémoire, pourcentage de
CPU\ldots{}). Sous top il est possible d'expédier de manière interactive
un signal à un processus, par exemple afin de le stopper, en tapant
``k'', top demande ensuite l'identifiant (PID) du processus concerné.
Pour quitter top, appuyer simplement sur la touche ``q''.

    \paragraph{Archivage de données
(tar):}\label{archivage-de-donnuxe9es-tar}

    La commande ``tar'' gère des archives, contenant chacune au moins un
répertoire ou fichier.

Vous aurez souvent besoin de ``tar xzf nom\_du\_fichier.tar.gz'' , qui
décompacte une archive au format .tar.gz ou .tgz. L'extension .tar.gz ou
.tgz indique que le fichier est une archive tar et qu'il est compacté.
Les arguments (aussi appelés options) employés dans la commande
précédente (``xzf'') peuvent être ainsi compris:

x (extraction) déclenche l'extraction de certains fichiers d'une archive
(lorsque l'on ne spécifie pas les noms des fichiers que l'on souhaite
extraire de l'archive, tar les extrait tous)

z décompacte l'archive grâce à la commande gzip

f traite un fichier-archive dont le nom suit (ici:
``nom\_du\_fichier.tar.gz'')

Si je me trouve dans le répertoire ``/home/delcros/'' la commande
suivante créera une archive du répertoire ``/home/delcros/personnel'' :

``tar cvzf personnel.tgz personnel''

l'option ``c'' permet de créer une archive

L'option ``z'' compacte l'archive grâce à gzip.

    \paragraph{Questions :}\label{questions}

    \begin{enumerate}
\def\labelenumi{\arabic{enumi})}
\itemsep1pt\parskip0pt\parsep0pt
\item
  Ouvrir une consolle Linux et créer un répertoire TP-PYTHON dans le
  répertoire courant. Comment vérifier qu'il a bien été créé ?
\end{enumerate}

    \begin{enumerate}
\def\labelenumi{\arabic{enumi})}
\setcounter{enumi}{1}
\itemsep1pt\parskip0pt\parsep0pt
\item
  Aller dans le répertoire que vous venez de créer, puis lister ce qu'il
  contient. Quelle commande vous permet de savoir si vous êtes bien dans
  le répertoire TP-PYTHON ?
\end{enumerate}

    \begin{enumerate}
\def\labelenumi{\arabic{enumi})}
\setcounter{enumi}{2}
\itemsep1pt\parskip0pt\parsep0pt
\item
  Créer un fichier appelé README.txt dans le répertoire TP-PYTHON avec
  la commande ``emacs'', taper quelques lignes dedans. Enregistrer le
  ficher. Vérifier qu'il a bien été créé.
\end{enumerate}

    \begin{enumerate}
\def\labelenumi{\arabic{enumi})}
\setcounter{enumi}{3}
\itemsep1pt\parskip0pt\parsep0pt
\item
  Créer un fichier-archive du repertoire TP-PYTHON.
\end{enumerate}

    \begin{enumerate}
\def\labelenumi{\arabic{enumi})}
\setcounter{enumi}{4}
\itemsep1pt\parskip0pt\parsep0pt
\item
  Supprimer le fichier que vous venez de créer, comment vérifier qu'il a
  bien été supprimé ?
\end{enumerate}

    \begin{enumerate}
\def\labelenumi{\arabic{enumi})}
\setcounter{enumi}{5}
\itemsep1pt\parskip0pt\parsep0pt
\item
  Supprimer le répertoire TP-PYTHON. Vérifier qu'il a bien été supprimé.
\end{enumerate}

    \begin{enumerate}
\def\labelenumi{\arabic{enumi})}
\setcounter{enumi}{6}
\itemsep1pt\parskip0pt\parsep0pt
\item
  Lancer l'interprète du langage Python (écrire ``python'' dans la
  consolle de Linux). Supposons maintenant que, pour une raison
  quelconque, le programme est bloqué. Tuez l'interprète Python en
  utilisant la commande ``top''.
\end{enumerate}

    \subsection{Introduction à la programmation numérique en
Python}\label{introduction-uxe0-la-programmation-numuxe9rique-en-python}

    Pour lancer l'interprète du langage Python écrire ``python'' dans la
consolle de Linux.

Saisir votre série d'instructions Python après l'invite de la ligne de
commande ``\textgreater{}\textgreater{}\textgreater{}'' (à noter que
chaque instruction doit être validée avec la touche ``return'').

    Une deuxième possibilité d'utilisation beaucoup plus pratique est de
créer un fichier (appelé script), par exemple avec l'éditeur de text
emacs : ``emacs -nw script.py''

    Une fois que les instructions ont été écrites dans ce fichier, le script
peut être exécuté avec la commande ``python script.py''.

    

    Le site le plus utile pour trouver des informations sur python est :
https://docs.python.org/2/\\voir notamment dans les sections
``Tutorial'' et ``Language reference''.

    

    \subsubsection{Modalités pour accomplir le TP et compte
rendu}\label{modalituxe9s-pour-accomplir-le-tp-et-compte-rendu}

    Pendant ce TP vous aurez à écrire plusieurs scripts (nous vous suggérons
de les nommer script1.py, script2.py, \ldots{}). Les scripts doivent
être accompagnés par un document descriptif unique (README.txt). Dans ce
fichier, vous devrez décrire le mode de fonctionnement des scripts et,
si besoin, mettre vos commentaires. Merci d'y écrires aussi vos nomes et
prénoms complets. Tous les fichiers doivent être mis dans un dossier
appelé TP1-nom1-nom2 et ensuite être compressés dans un fichier
TP1-nom1-nom2.tgz .

Enfin vous allez envoyer ce fichier par email à l'enseignant : soit
Enrico (enrico.calzavarini@polytech-lille.fr) soit Stefano
(stefano.berti@polytech-lille.fr).

    \paragraph{Vous avez une semaine de temps pour compléter le TP,
c'est-à-dire que la date limite pour envoyer vos travaux c'est dans 7
jours à partir
d'aujourd'hui.}\label{vous-avez-une-semaine-de-temps-pour-compluxe9ter-le-tp-cest-uxe0-dire-que-la-date-limite-pour-envoyer-vos-travaux-cest-dans-7-jours-uxe0-partir-daujourdhui.}

    \subsection{Script 1 : entrer ou afficher du texte et des
données}\label{script-1-entrer-ou-afficher-du-texte-et-des-donnuxe9es}

    Nous voulons écrire un script qui tout d'abord affiche simplement une
message de bienvenue :

    \begin{Verbatim}[commandchars=\\\{\}]
{\color{incolor}In [{\color{incolor}10}]:} \PY{k}{print}\PY{p}{(}\PY{l+s}{\PYZdq{}}\PY{l+s}{Bienvenue au premier TP d}\PY{l+s}{\PYZsq{}}\PY{l+s}{informatique}\PY{l+s}{\PYZdq{}}\PY{p}{)}
\end{Verbatim}

    \begin{Verbatim}[commandchars=\\\{\}]
Bienvenue au premier TP d'informatique
    \end{Verbatim}

    Vérifiez que le programme ci-dessus fonctionne quand il est écrit dans
un script (appelé par exemple script1.py) et interprété par python avec
la commande ``python script1.py''.

    Nous souhaitons maintenant que ce programme nous demande aussi combien
d'étudiants sont présents dans la salle et qu'il affiche ce numéro sur
l'écran.

    A noter que la lecture de données peut se faire par interrogation de
l'utilisateur. Par exemple :

    \begin{Verbatim}[commandchars=\\\{\}]
{\color{incolor}In [{\color{incolor}11}]:} \PY{n}{p} \PY{o}{=} \PY{n+nb}{input}\PY{p}{(}\PY{l+s}{\PYZdq{}}\PY{l+s}{Entrer la valeur de la precision p, p =}\PY{l+s}{\PYZdq{}}\PY{p}{)}
\end{Verbatim}

    \begin{Verbatim}[commandchars=\\\{\}]
Entrer la valeur de la precision p, p =0.1
    \end{Verbatim}

    l'écriture des données enregistrées dans une variable peut se faire tout
simplement encore avec la fonction ``print''

    \begin{Verbatim}[commandchars=\\\{\}]
{\color{incolor}In [{\color{incolor}12}]:} \PY{k}{print}\PY{p}{(}\PY{l+s}{\PYZdq{}}\PY{l+s}{la valeure de la precision est :}\PY{l+s}{\PYZdq{}} \PY{o}{+} \PY{n+nb}{str}\PY{p}{(}\PY{n}{p}\PY{p}{)} \PY{p}{)}
\end{Verbatim}

    \begin{Verbatim}[commandchars=\\\{\}]
la valeure de la precision est :0.1
    \end{Verbatim}

    ou de façon équivalente :

    \begin{Verbatim}[commandchars=\\\{\}]
{\color{incolor}In [{\color{incolor}13}]:} \PY{k}{print}\PY{p}{(}\PY{l+s}{\PYZdq{}}\PY{l+s}{la valeure de la precision est : }\PY{l+s+si}{\PYZpc{}e}\PY{l+s}{\PYZdq{}} \PY{o}{\PYZpc{}}\PY{k}{p} )
\end{Verbatim}

    \begin{Verbatim}[commandchars=\\\{\}]
la valeure de la precision est : 1.000000e-01
    \end{Verbatim}

    \subsubsection{Commentaires en python}\label{commentaires-en-python}

    Une bonne habitude à prendre est d'écrire des commentaires dans le
srtipts. Nous pouvons le faire avec le symbole " \# " , par exemple :

    \begin{Verbatim}[commandchars=\\\{\}]
{\color{incolor}In [{\color{incolor}14}]:} \PY{k}{print}\PY{p}{(}\PY{l+s}{\PYZdq{}}\PY{l+s}{la valeure de p est : }\PY{l+s+si}{\PYZpc{}e}\PY{l+s}{\PYZdq{}} \PY{o}{\PYZpc{}}\PY{k}{p} ) \PYZsh{} ici nous imprimons le résultat final
\end{Verbatim}

    \begin{Verbatim}[commandchars=\\\{\}]
la valeure de p est : 1.000000e-01
    \end{Verbatim}

    \paragraph{Écrivez donc votre premier
script!}\label{uxe9crivez-donc-votre-premier-script}

    \subsection{Script 2 : effectuer des opérations simples sur les
variables}\label{script-2-effectuer-des-opuxe9rations-simples-sur-les-variables}

    Nous souhaiton écrire un script qui fait la conversion d'unité pour des
mesures de pression.

    Ce script devra demander à l'utilisateur d'introduire une valeur de
pression et une unité de mesure (choisi parmi: Pascal (Pa) , bar (bar) ,
atmosphère (atm) , Torricelli (torr) et pounds per square inch (psi)),
puis demander à l'utilisateur quelle unité il souhaite pour la
conversion et enfin afficher la valeur de pression dans l'unité
demandée.

    Pour mener à bien ce script les informations suivantes peuvent être
utiles :

    La table de conversion des unités de pression peut être repèrée ici :

http://en.wikipedia.org/wiki/Pressure\_measurement\#Units

    \subsubsection{L'affectation de données dans des
variables}\label{laffectation-de-donnuxe9es-dans-des-variables}

    \begin{Verbatim}[commandchars=\\\{\}]
{\color{incolor}In [{\color{incolor}15}]:} \PY{n}{a} \PY{o}{=} \PY{l+m+mf}{5.0} 
         \PY{n}{b} \PY{p}{,} \PY{n}{c} \PY{o}{=} \PY{l+m+mi}{10} \PY{p}{,} \PY{l+m+mi}{3}
         \PY{n}{name} \PY{o}{=} \PY{l+s}{\PYZdq{}}\PY{l+s}{Mark}\PY{l+s}{\PYZdq{}}
\end{Verbatim}

    Ici ``a'' , ``b'' , ``c'' et ``name'' sont des identificateurs (ou
simplement des variables).

Python ne nécessite pas de déclarer explicitement les types de
variables, différemment qu'en Fortran, C et d'autres langages. Juste
affectez une variable et Python comprendera automatiquement, ce n'est
pas nécessaire de préciser le type (réel, entier, character,
etc\ldots{})

Vous pouvez demander à Python de vous dire quel type il a attribué à vos
variables:

    \begin{Verbatim}[commandchars=\\\{\}]
{\color{incolor}In [{\color{incolor}16}]:} \PY{n+nb}{type}\PY{p}{(}\PY{n}{a}\PY{p}{)} 
\end{Verbatim}

            \begin{Verbatim}[commandchars=\\\{\}]
{\color{outcolor}Out[{\color{outcolor}16}]:} float
\end{Verbatim}
        
    \begin{Verbatim}[commandchars=\\\{\}]
{\color{incolor}In [{\color{incolor}17}]:} \PY{n+nb}{type}\PY{p}{(}\PY{n}{b}\PY{p}{)} 
\end{Verbatim}

            \begin{Verbatim}[commandchars=\\\{\}]
{\color{outcolor}Out[{\color{outcolor}17}]:} int
\end{Verbatim}
        
    \begin{Verbatim}[commandchars=\\\{\}]
{\color{incolor}In [{\color{incolor}18}]:} \PY{n+nb}{type}\PY{p}{(}\PY{n}{name}\PY{p}{)}
\end{Verbatim}

            \begin{Verbatim}[commandchars=\\\{\}]
{\color{outcolor}Out[{\color{outcolor}18}]:} str
\end{Verbatim}
        
    Portez une attention particulière à l'attribution de valeurs à virgule
flottante à des variables ou vous risquez d'obtenir des valeurs que vous
ne vous attendez pas dans vos programmes. Par exemple ,

    \begin{Verbatim}[commandchars=\\\{\}]
{\color{incolor}In [{\color{incolor}19}]:} \PY{k}{print}\PY{p}{(}\PY{n}{b}\PY{o}{/}\PY{n}{c}\PY{p}{)}
\end{Verbatim}

    \begin{Verbatim}[commandchars=\\\{\}]
3
    \end{Verbatim}

    Vous voyez , si vous divisez un nombre entier par un nombre entier ,
Python retourne une réponse arrondie à l'entier le plus proche . Mais si
vous voulez une résultat à virgule flottante (type float), l'un des
numéros doit être de type float. Le simple ajout d'un point décimal fera
l'affaire :

    \begin{Verbatim}[commandchars=\\\{\}]
{\color{incolor}In [{\color{incolor}20}]:} \PY{n}{b} \PY{p}{,} \PY{n}{c} \PY{o}{=} \PY{l+m+mf}{10.} \PY{p}{,} \PY{l+m+mf}{3.}
         \PY{k}{print} \PY{p}{(}\PY{n}{b}\PY{o}{/}\PY{n}{c}\PY{p}{)}
\end{Verbatim}

    \begin{Verbatim}[commandchars=\\\{\}]
3.33333333333
    \end{Verbatim}

    \subsubsection{Utilisation des listes}\label{utilisation-des-listes}

    Une liste est une collections d'objets :

    \begin{Verbatim}[commandchars=\\\{\}]
{\color{incolor}In [{\color{incolor}21}]:} \PY{c}{\PYZsh{} Definition et affectation d\PYZsq{}une liste}
         \PY{n}{l} \PY{o}{=} \PY{p}{[}\PY{l+m+mi}{0}\PY{p}{,} \PY{l+m+mi}{5}\PY{p}{,} \PY{l+m+mi}{6}\PY{p}{,} \PY{l+m+mi}{4}\PY{p}{]}  
         
         \PY{c}{\PYZsh{} Les indices commencent à 0, et non à 1}
         \PY{k}{print}\PY{p}{(}\PY{n}{l}\PY{p}{[}\PY{l+m+mi}{0}\PY{p}{]}\PY{p}{)}      
         
         \PY{c}{\PYZsh{} Le dernier indice est le 3ème}
         \PY{k}{print}\PY{p}{(}\PY{n}{l}\PY{p}{[}\PY{l+m+mi}{3}\PY{p}{]}\PY{p}{)}      
         
         \PY{c}{\PYZsh{} On peut compter à partir de la fin, en utilisant des indices negatifs}
         \PY{k}{print}\PY{p}{(}\PY{n}{l}\PY{p}{[}\PY{o}{\PYZhy{}}\PY{l+m+mi}{1}\PY{p}{]}\PY{p}{)}     
         
         \PY{c}{\PYZsh{} On peut selectioner une tranche (slice) d\PYZsq{}elements dans la liste}
         \PY{k}{print}\PY{p}{(}\PY{n}{l}\PY{p}{[}\PY{l+m+mi}{0}\PY{p}{:}\PY{l+m+mi}{3}\PY{p}{]}\PY{p}{)}    
         
         \PY{c}{\PYZsh{} La syntaxe du “slicing” est [start:stop:step]}
         \PY{k}{print}\PY{p}{(}\PY{n}{l}\PY{p}{[}\PY{l+m+mi}{0}\PY{p}{:}\PY{l+m+mi}{3}\PY{p}{:}\PY{l+m+mi}{2}\PY{p}{]}\PY{p}{)}  
\end{Verbatim}

    \begin{Verbatim}[commandchars=\\\{\}]
0
4
4
[0, 5, 6]
[0, 6]
    \end{Verbatim}

    Une liste peut être modifiée:

    \begin{Verbatim}[commandchars=\\\{\}]
{\color{incolor}In [{\color{incolor}22}]:} \PY{n}{l}\PY{p}{[}\PY{l+m+mi}{2}\PY{p}{]} \PY{o}{=} \PY{n}{l}\PY{p}{[}\PY{l+m+mi}{2}\PY{p}{]}\PY{o}{*}\PY{l+m+mi}{2} \PY{o}{+} \PY{l+m+mi}{1}
         
         \PY{k}{print}\PY{p}{(}\PY{n}{l}\PY{p}{)}
\end{Verbatim}

    \begin{Verbatim}[commandchars=\\\{\}]
[0, 5, 13, 4]
    \end{Verbatim}

    Si nous voulons connaître le nombre d'éléments dans une liste :

    \begin{Verbatim}[commandchars=\\\{\}]
{\color{incolor}In [{\color{incolor}23}]:} \PY{n+nb}{len}\PY{p}{(}\PY{n}{l}\PY{p}{)}
\end{Verbatim}

            \begin{Verbatim}[commandchars=\\\{\}]
{\color{outcolor}Out[{\color{outcolor}23}]:} 4
\end{Verbatim}
        
    Nous pouvons créer de la même manière une liste de mots :

    \begin{Verbatim}[commandchars=\\\{\}]
{\color{incolor}In [{\color{incolor}24}]:} \PY{n}{mots} \PY{o}{=} \PY{p}{[}\PY{l+s}{\PYZdq{}}\PY{l+s}{arbre}\PY{l+s}{\PYZdq{}} \PY{p}{,} \PY{l+s}{\PYZdq{}}\PY{l+s}{avion}\PY{l+s}{\PYZdq{}} \PY{p}{,} \PY{l+s}{\PYZdq{}}\PY{l+s}{nuage}\PY{l+s}{\PYZdq{}}\PY{p}{]}
         
         \PY{k}{print}\PY{p}{(}\PY{n}{mots}\PY{p}{[}\PY{l+m+mi}{1}\PY{p}{]}\PY{p}{)}
\end{Verbatim}

    \begin{Verbatim}[commandchars=\\\{\}]
avion
    \end{Verbatim}

    \subsubsection{Structure alternative
``if''}\label{structure-alternative-if}

    Regardez l'exemple ci-dessous concernant le calcul des racines réelles
d'une équation algébrique du second ordre, pour comprendre le
fonctionnement d'une structure ``if'' :

    \begin{Verbatim}[commandchars=\\\{\}]
{\color{incolor}In [{\color{incolor}25}]:} \PY{n}{d} \PY{o}{=} \PY{n}{b}\PY{o}{*}\PY{o}{*}\PY{l+m+mi}{2}\PY{o}{\PYZhy{}}\PY{l+m+mi}{4}\PY{o}{*}\PY{n}{a}\PY{o}{*}\PY{n}{c}
         
         \PY{k}{if} \PY{n}{d} \PY{o}{\PYZgt{}} \PY{l+m+mi}{0} \PY{p}{:}
             \PY{n}{x1} \PY{o}{=} \PY{p}{(}\PY{o}{\PYZhy{}}\PY{n}{b} \PY{o}{\PYZhy{}} \PY{n}{sqrt}\PY{p}{(}\PY{n}{d}\PY{p}{)}\PY{p}{)}\PY{o}{/}\PY{p}{(}\PY{l+m+mi}{2}\PY{o}{*}\PY{n}{a}\PY{p}{)}
             \PY{n}{x2} \PY{o}{=} \PY{p}{(}\PY{o}{\PYZhy{}}\PY{n}{b} \PY{o}{+} \PY{n}{sqrt}\PY{p}{(}\PY{n}{d}\PY{p}{)}\PY{p}{)}\PY{o}{/}\PY{p}{(}\PY{l+m+mi}{2}\PY{o}{*}\PY{n}{a}\PY{p}{)}
             \PY{k}{print}\PY{p}{(}\PY{l+s}{\PYZdq{}}\PY{l+s}{RESULTATS :}\PY{l+s}{\PYZdq{}}\PY{p}{)}
             \PY{k}{print}\PY{p}{(}\PY{l+s}{\PYZdq{}}\PY{l+s}{Deux racines distinctes : x1 = }\PY{l+s}{\PYZdq{}} \PY{o}{+} \PY{n+nb}{str}\PY{p}{(}\PY{n}{x1}\PY{p}{)} \PY{o}{+} \PY{l+s}{\PYZdq{}}\PY{l+s}{ et x2 = }\PY{l+s}{\PYZdq{}} \PY{o}{+} \PY{n+nb}{str}\PY{p}{(}\PY{n}{x2}\PY{p}{)}\PY{p}{)}
         \PY{k}{elif} \PY{n}{d} \PY{o}{==} \PY{l+m+mi}{0} \PY{p}{:}
             \PY{n}{x1} \PY{o}{=} \PY{o}{\PYZhy{}}\PY{n}{b}\PY{o}{/}\PY{p}{(}\PY{l+m+mi}{2}\PY{o}{*}\PY{n}{a}\PY{p}{)}
             \PY{k}{print}\PY{p}{(}\PY{l+s}{\PYZdq{}}\PY{l+s}{RESULTATS :}\PY{l+s}{\PYZdq{}}\PY{p}{)}
             \PY{k}{print}\PY{p}{(}\PY{l+s}{\PYZdq{}}\PY{l+s}{Une racine double : x1 = }\PY{l+s}{\PYZdq{}} \PY{o}{+} \PY{n+nb}{str}\PY{p}{(}\PY{n}{x1}\PY{p}{)}\PY{p}{)}
         \PY{k}{else} \PY{p}{:}
             \PY{k}{print}\PY{p}{(}\PY{l+s}{\PYZdq{}}\PY{l+s}{RESULTATS :}\PY{l+s}{\PYZdq{}}\PY{p}{)}
             \PY{k}{print}\PY{p}{(}\PY{l+s}{\PYZdq{}}\PY{l+s}{Aucune solution!}\PY{l+s}{\PYZdq{}}\PY{p}{)}    
\end{Verbatim}

    \begin{Verbatim}[commandchars=\\\{\}]
RESULTATS :
Deux racines distinctes : x1 = -1.63245553203 et x2 = -0.367544467966
    \end{Verbatim}

    \subsubsection{à noter : indentation nécessaire en langage
Python}\label{uxe0-noter-indentation-nuxe9cessaire-en-langage-python}

    Pour les structures de controle du langage Python, comme ``if''(et nous
verros ensuite ``for''), il n'y a pas de mot clé ``end'' pour signaler
la fin du bloc de lignes concernées par les structures
``if'',``while'',``for''. Il faut par contre indenter les lignes (c'est
à dire créer des décalages à l'aide de la touche tabulation ``tab'' du
clavier) afin de définir une dépendance d'un bloc de lignes par rapport
à un autre.

    \subsection{Script 3 : lire des données à partir d'un fichier et les
élaborer}\label{script-3-lire-des-donnuxe9es-uxe0-partir-dun-fichier-et-les-uxe9laborer}

    Nous disposons des données de pression enregistrées (à la fréquence
d'une mesure par seconde) par un tube de Pitot d'un avion en vol à une
altitude de 10000 mètres :

https://github.com/ecalzavarini/python-at-polytech-lille/blob/master/pressure.txt

Vous pouvez télécharger ce fichier avec la commande de Linux " wget ``.
Ecrivez sur votre console :''wget adresse\_internet\_du\_fichier ``.

    \begin{Verbatim}[commandchars=\\\{\}]
{\color{incolor}In [{\color{incolor}26}]:} \PY{k+kn}{from} \PY{n+nn}{IPython.display} \PY{k+kn}{import} \PY{n}{Image}
         \PY{n}{Image}\PY{p}{(}\PY{l+s}{\PYZsq{}}\PY{l+s}{http://upload.wikimedia.org/wikipedia/commons/0/00/Pitot\PYZus{}tube\PYZus{}wings\PYZhy{}fr.jpg}\PY{l+s}{\PYZsq{}}\PY{p}{)}
\end{Verbatim}
\texttt{\color{outcolor}Out[{\color{outcolor}26}]:}
    
    \begin{center}
    \adjustimage{max size={0.9\linewidth}{0.9\paperheight}}{Python-TP1_files/Python-TP1_85_0.jpg}
    \end{center}
    { \hspace*{\fill} \\}
    

    On est interessé à écrire un script qui lit ces données et qui les
transforme dans des mesures de vitesse. La relation entre la pression et
la vitesse est donnée par la loi de Bernoulli:

    \[V(t) = \sqrt{\frac{2\left(p(t)-p_s\right)}{\rho}}.\] Ici:

\(p(t)\) est la pression mesurée au cours du temps (\(t\)) en Pascal,

\(p_s\) est la pression statique, \(p_s=4 \, kPa\),

\(\rho\) est la masse volumique de l'air, \(\rho=0.91875 \, kg/m^3\).

    Pour la réalisation de ce script les informations suivantes vous seront
utiles :

    \subsubsection{La lecture des données}\label{la-lecture-des-donnuxe9es}

    \begin{Verbatim}[commandchars=\\\{\}]
{\color{incolor}In [{\color{incolor}27}]:} \PY{c}{\PYZsh{} Ouverture d\PYZsq{}un fichier}
         \PY{n}{f} \PY{o}{=} \PY{n+nb}{open}\PY{p}{(}\PY{l+s}{\PYZdq{}}\PY{l+s}{pressure.txt}\PY{l+s}{\PYZdq{}}\PY{p}{,} \PY{l+s}{\PYZdq{}}\PY{l+s}{r}\PY{l+s}{\PYZdq{}}\PY{p}{)}
         \PY{c}{\PYZsh{} définition d\PYZsq{}une liste vide}
         \PY{n}{data}\PY{o}{=}\PY{p}{[}\PY{p}{]}
         \PY{c}{\PYZsh{} lecture de fichier complet et affectation des éléments de la liste }
         \PY{n}{data}\PY{o}{=}\PY{n}{f}\PY{o}{.}\PY{n}{read}\PY{p}{(}\PY{p}{)}\PY{o}{.}\PY{n}{split}\PY{p}{(}\PY{p}{)}
         \PY{c}{\PYZsh{} fermeture du fichier f}
         \PY{n}{f}\PY{o}{.}\PY{n}{close}\PY{p}{(}\PY{p}{)}
         
         \PY{c}{\PYZsh{}contrôle de type}
         \PY{k}{print}\PY{p}{(} \PY{n+nb}{type}\PY{p}{(}\PY{n}{data}\PY{p}{[}\PY{l+m+mi}{0}\PY{p}{]}\PY{p}{)} \PY{p}{)}
         \PY{c}{\PYZsh{}impression de la première donnée }
         \PY{k}{print}\PY{p}{(}\PY{n}{data}\PY{p}{[}\PY{l+m+mi}{0}\PY{p}{]}\PY{p}{)}
\end{Verbatim}

    \begin{Verbatim}[commandchars=\\\{\}]
<type 'str'>
53274.300000
    \end{Verbatim}

    \begin{Verbatim}[commandchars=\\\{\}]
{\color{incolor}In [{\color{incolor}28}]:} \PY{c}{\PYZsh{}conversion au type désiré }
         \PY{n}{data}\PY{p}{[}\PY{l+m+mi}{0}\PY{p}{]} \PY{o}{=} \PY{n+nb}{float}\PY{p}{(}\PY{n}{data}\PY{p}{[}\PY{l+m+mi}{0}\PY{p}{]}\PY{p}{)}
         \PY{c}{\PYZsh{}contrôle de type}
         \PY{k}{print} \PY{p}{(} \PY{n+nb}{type}\PY{p}{(}\PY{n}{data}\PY{p}{[}\PY{l+m+mi}{0}\PY{p}{]}\PY{p}{)} \PY{p}{)}
         \PY{c}{\PYZsh{}impression}
         \PY{k}{print} \PY{n}{data}\PY{p}{[}\PY{l+m+mi}{0}\PY{p}{]}
\end{Verbatim}

    \begin{Verbatim}[commandchars=\\\{\}]
<type 'float'>
53274.3
    \end{Verbatim}

    à noter que lorsque les données sont lues de la manière ci-dessus, elles
sont stockées dans une liste de type chaîne de caractères (string). Si
nous devons effectuer des opérations mathématiques sur elles, nous
devons les convertir en un type numérique ( par exemple ``float'' ou
``int'' ).

    \subsubsection{Ecriture dans un fichier}\label{ecriture-dans-un-fichier}

    \begin{Verbatim}[commandchars=\\\{\}]
{\color{incolor}In [{\color{incolor}29}]:} \PY{n}{val1} \PY{o}{=} \PY{l+m+mf}{5.0}  \PY{c}{\PYZsh{} affectation d\PYZsq{}une variable réelle}
         
         \PY{n}{val2} \PY{o}{=} \PY{n+nb}{str}\PY{p}{(}\PY{n}{val1}\PY{p}{)} \PY{c}{\PYZsh{} conversion en chaîne de caractères}
         
         \PY{c}{\PYZsh{} Ouverture d\PYZsq{}un nouveau fichier}
         \PY{n}{f} \PY{o}{=} \PY{n+nb}{open}\PY{p}{(}\PY{l+s}{\PYZdq{}}\PY{l+s}{output.txt}\PY{l+s}{\PYZdq{}}\PY{p}{,} \PY{l+s}{\PYZdq{}}\PY{l+s}{w}\PY{l+s}{\PYZdq{}}\PY{p}{)}
         \PY{c}{\PYZsh{} écriture dans le fichier }
         \PY{n}{f}\PY{o}{.}\PY{n}{write}\PY{p}{(}\PY{n}{val2}\PY{p}{)}  
         \PY{c}{\PYZsh{} fermeture du fichier }
         \PY{n}{f}\PY{o}{.}\PY{n}{close}\PY{p}{(}\PY{p}{)}
\end{Verbatim}

    à noter que la commande ``write'' ne peut écrire que des chaînes de
caractères (string).

    \subsubsection{La structure repetitive
``for''}\label{la-structure-repetitive-for}

    Regardez l'exemple ci-dessous concernant tri par sélection de données,
pour comprendre le fonctionnement d'une structure boucle ``for'' :

    \begin{Verbatim}[commandchars=\\\{\}]
{\color{incolor}In [{\color{incolor}30}]:} \PY{c}{\PYZsh{} Tri par sélection}
         \PY{c}{\PYZsh{} Données initiales }
         \PY{n}{M} \PY{o}{=} \PY{p}{[}\PY{l+m+mf}{1.} \PY{p}{,} \PY{l+m+mf}{2.} \PY{p}{,} \PY{l+m+mf}{5.} \PY{p}{,} \PY{l+m+mf}{40.} \PY{p}{,} \PY{l+m+mf}{65.} \PY{p}{,} \PY{l+m+mf}{2.}\PY{p}{,}\PY{l+m+mf}{8.} \PY{p}{,} \PY{l+m+mf}{98.} \PY{p}{,}\PY{l+m+mf}{115.} \PY{p}{,}\PY{l+m+mf}{0.} \PY{p}{,} \PY{l+m+mf}{3.}\PY{p}{]}
         
         \PY{c}{\PYZsh{} Détermination de la longueur  de la liste}
         \PY{n}{N} \PY{o}{=} \PY{n+nb}{len}\PY{p}{(}\PY{n}{M}\PY{p}{)} 
         
         \PY{k}{for} \PY{n}{i} \PY{o+ow}{in} \PY{n+nb}{range}\PY{p}{(}\PY{l+m+mi}{0}\PY{p}{,}\PY{n}{N}\PY{o}{\PYZhy{}}\PY{l+m+mi}{1}\PY{p}{)}\PY{p}{:}   
             \PY{k}{for} \PY{n}{j} \PY{o+ow}{in} \PY{n+nb}{range}\PY{p}{(}\PY{n}{i}\PY{o}{+}\PY{l+m+mi}{1}\PY{p}{,}\PY{n}{N}\PY{p}{)}\PY{p}{:}
                 \PY{k}{if} \PY{n}{M}\PY{p}{[}\PY{n}{j}\PY{p}{]} \PY{o}{\PYZlt{}} \PY{n}{M}\PY{p}{[}\PY{n}{i}\PY{p}{]}\PY{p}{:}
                     \PY{n}{val\PYZus{}temp} \PY{o}{=} \PY{n}{M}\PY{p}{[}\PY{n}{i}\PY{p}{]}
                     \PY{n}{M}\PY{p}{[}\PY{n}{i}\PY{p}{]} \PY{o}{=} \PY{n}{M}\PY{p}{[}\PY{n}{j}\PY{p}{]}
                     \PY{n}{M}\PY{p}{[}\PY{n}{j}\PY{p}{]} \PY{o}{=} \PY{n}{val\PYZus{}temp}
         \PY{k}{print}\PY{p}{(}\PY{l+s}{\PYZdq{}}\PY{l+s}{Liste triée :}\PY{l+s}{\PYZdq{}}\PY{p}{)}
         \PY{k}{print}\PY{p}{(}\PY{n}{M}\PY{p}{)}
\end{Verbatim}

    \begin{Verbatim}[commandchars=\\\{\}]
Liste triée :
[0.0, 1.0, 2.0, 2.0, 3.0, 5.0, 8.0, 40.0, 65.0, 98.0, 115.0]
    \end{Verbatim}

    à noter que la commande " range(a,b) " produit une liste de nombres
entiers progressifs compris entre a et b-1 , par exemple :

    \begin{Verbatim}[commandchars=\\\{\}]
{\color{incolor}In [{\color{incolor}31}]:} \PY{n+nb}{range}\PY{p}{(}\PY{l+m+mi}{0}\PY{p}{,}\PY{l+m+mi}{10}\PY{p}{)}
\end{Verbatim}

            \begin{Verbatim}[commandchars=\\\{\}]
{\color{outcolor}Out[{\color{outcolor}31}]:} [0, 1, 2, 3, 4, 5, 6, 7, 8, 9]
\end{Verbatim}
        
    \subsubsection{Utiliser les fonctions
mathématiques}\label{utiliser-les-fonctions-mathuxe9matiques}

    En Python afin de pouvoir utiliser les fonctions mathématiques (par
exemple, exp, sqrt , sin , cos , tan), vous devez les importer par
avance à partir de la bibliothèque mathématique ``math'' :

    \begin{Verbatim}[commandchars=\\\{\}]
{\color{incolor}In [{\color{incolor}32}]:} \PY{c}{\PYZsh{} importing all functions from the math library}
         \PY{k+kn}{from} \PY{n+nn}{math} \PY{k+kn}{import} \PY{o}{*}      
\end{Verbatim}

    Nous vous demandons donc de lire la serie des données de pression dans
le fichier et de calculer la vitesse de l' avion en utilisant la
relation donnée.

    On demande ensuite de calculer la valeur moyenne de la vitesse de
l'avion (en Km/heure) et de l'imprimer.

    Question bonus : Seriez-vous capable de calculer l'accélération de
l'avion au cours du temps, et de l'enregistrer dans un fichier?

    \subsection{Script 4 : représentation graphique des
données}\label{script-4-repruxe9sentation-graphique-des-donnuxe9es}

    Nous souhaitons analyser graphiquement les données , par exemple en les
traçant ou en faisant leur histogramme .

    Pour la réalisation de ce script les informations suivantes vous seront
utiles :

    \subsubsection{Bibliothèques de
Python}\label{bibliothuxe8ques-de-python}

    Python dispose de nombreuses bibliothèques qui fournissent des
fonctionnalités avancées comme la possibilté d'effectuer des opérations
sur des matrices, des fonctions graphiques, et bien plus encore. Nous
pouvons donc importer ces bibliothèques de fonctions pour étendre les
capacités de Python dans nos programmes.

Nous allons commencer par l'importation de quelques bibliothèques pour
nous aider.

Premièrement, nous l'avons déjà vu, nous utilisons Math, la bibliothèque
mathématique standard qui comprend les fonctions trigonométriques et
transcendantales, ainsi que d'autres fonctions spéciales.

Pour voir la liste complète des fonctions de la bibliothèque Math:
https://docs.python.org/2/library/math.html\#module-math

Notre deuxième bibliothèque préférée est NumPy (Numerical Python),
fournissant un grande nombre d'opérations matricielles utiles (cette
bibliothèque est similaire à MATLAB ). Nous allons l'utiliser beaucoup!

La documentation de numpy se trouve ici :
http://docs.scipy.org/doc/numpy/reference/

La troisième bibliothèque dont nous avons besoin est Matplotlib, une
bibliothèque de graphique 2D que nous allons utiliser pour tracer nos
résultats.

La documentation de référence est disponible ici :
http://matplotlib.org/contents.html

Le code suivant sera au sommet de la plupart de vos programmes :

    \begin{Verbatim}[commandchars=\\\{\}]
{\color{incolor}In [{\color{incolor}33}]:} \PY{c}{\PYZsh{} importing all functions from the math library}
         \PY{k+kn}{from} \PY{n+nn}{math} \PY{k+kn}{import} \PY{o}{*}            
         \PY{c}{\PYZsh{} we import the array library, and call it \PYZsq{}np\PYZsq{}}
         \PY{k+kn}{import} \PY{n+nn}{numpy} \PY{k+kn}{as} \PY{n+nn}{np}                 
         \PY{c}{\PYZsh{} import plotting library, and call it \PYZsq{}plt\PYZsq{}}
         \PY{k+kn}{import} \PY{n+nn}{matplotlib.pyplot} \PY{k+kn}{as} \PY{n+nn}{plt}    
\end{Verbatim}

    Nous importons donc toutes les fonctions de la bibliothèque ``math'' et
nous importons aussi une bibliothèque nommée ``numpy''. Enfin, nous
importons un module appelé ``pyplot'' d'une grande bibliothèque appelée
``matplotlib''.

Les deux dernières lignes ci-dessus ont créé des raccourcis pour les
bibliothèques tel que ``np'' et ``plt'' respectivement. La raison de la
création des raccourcis est que nous avons besoin de taper les noms de
bibliothèque assez souvent dans le code. Pour utiliser une fonction
appartenant à l'une de ces bibliothèques, nous devons dire à Python où
la chercher. Pour cela, chaque nom de fonction est écrit après le nom de
la bibliothèque, avec un point entre les deux. Par exemple :

    \begin{Verbatim}[commandchars=\\\{\}]
{\color{incolor}In [{\color{incolor}34}]:} \PY{n}{np}\PY{o}{.}\PY{n}{zeros}\PY{p}{(}\PY{l+m+mi}{5}\PY{p}{)}
\end{Verbatim}

            \begin{Verbatim}[commandchars=\\\{\}]
{\color{outcolor}Out[{\color{outcolor}34}]:} array([ 0.,  0.,  0.,  0.,  0.])
\end{Verbatim}
        
    \subsubsection{Tracer des courbes}\label{tracer-des-courbes}

    \begin{Verbatim}[commandchars=\\\{\}]
{\color{incolor}In [{\color{incolor}35}]:} \PY{o}{\PYZpc{}}\PY{k}{matplotlib} inline  
         \PY{c}{\PYZsh{}ignorez la ligne ci\PYZhy{}dessus. c\PYZsq{}est juste une commande de notre éditeur de texte}
         
         \PY{n}{x}\PY{o}{=}\PY{n}{np}\PY{o}{.}\PY{n}{linspace}\PY{p}{(}\PY{o}{\PYZhy{}}\PY{l+m+mi}{5}\PY{p}{,}\PY{l+m+mi}{5}\PY{p}{,}\PY{l+m+mi}{100}\PY{p}{)}  \PY{c}{\PYZsh{} nous définissons une liste (array) avec Numpy }
         \PY{n}{plt}\PY{o}{.}\PY{n}{plot}\PY{p}{(}\PY{n}{x}\PY{p}{,}\PY{n}{np}\PY{o}{.}\PY{n}{sin}\PY{p}{(}\PY{n}{x}\PY{p}{)}\PY{p}{)}  \PY{c}{\PYZsh{} on utilise la fonction sinus de Numpy}
         \PY{n}{plt}\PY{o}{.}\PY{n}{plot}\PY{p}{(}\PY{n}{x}\PY{p}{,}\PY{n}{np}\PY{o}{.}\PY{n}{cos}\PY{p}{(}\PY{n}{x}\PY{p}{)}\PY{p}{)}  
         \PY{n}{plt}\PY{o}{.}\PY{n}{ylabel}\PY{p}{(}\PY{l+s}{\PYZsq{}}\PY{l+s}{fonction sinus}\PY{l+s}{\PYZsq{}}\PY{p}{)}
         \PY{n}{plt}\PY{o}{.}\PY{n}{xlabel}\PY{p}{(}\PY{l+s}{\PYZdq{}}\PY{l+s}{l}\PY{l+s}{\PYZsq{}}\PY{l+s}{axe des abcisses}\PY{l+s}{\PYZdq{}}\PY{p}{)}
         \PY{n}{plt}\PY{o}{.}\PY{n}{show}\PY{p}{(}\PY{p}{)}
\end{Verbatim}

    \begin{center}
    \adjustimage{max size={0.9\linewidth}{0.9\paperheight}}{Python-TP1_files/Python-TP1_116_0.png}
    \end{center}
    { \hspace*{\fill} \\}
    
    \subsubsection{Changer le style des
courbes}\label{changer-le-style-des-courbes}

    \begin{Verbatim}[commandchars=\\\{\}]
{\color{incolor}In [{\color{incolor}36}]:} \PY{o}{\PYZpc{}}\PY{k}{matplotlib} inline  
         
         \PY{n}{t1}\PY{o}{=}\PY{n}{np}\PY{o}{.}\PY{n}{linspace}\PY{p}{(}\PY{l+m+mi}{0}\PY{p}{,}\PY{l+m+mi}{5}\PY{p}{,}\PY{l+m+mi}{10}\PY{p}{)}
         \PY{n}{t2}\PY{o}{=}\PY{n}{np}\PY{o}{.}\PY{n}{linspace}\PY{p}{(}\PY{l+m+mi}{0}\PY{p}{,}\PY{l+m+mi}{5}\PY{p}{,}\PY{l+m+mi}{20}\PY{p}{)}
         \PY{n}{plt}\PY{o}{.}\PY{n}{plot}\PY{p}{(}\PY{n}{t1}\PY{p}{,} \PY{n}{t1}\PY{p}{,} \PY{l+s}{\PYZsq{}}\PY{l+s}{r\PYZhy{}\PYZhy{}}\PY{l+s}{\PYZsq{}}\PY{p}{,} \PY{n}{t1}\PY{p}{,} \PY{n}{t1}\PY{o}{*}\PY{o}{*}\PY{l+m+mi}{2}\PY{p}{,} \PY{l+s}{\PYZsq{}}\PY{l+s}{bs}\PY{l+s}{\PYZsq{}}\PY{p}{,} \PY{n}{t2}\PY{p}{,} \PY{n}{t2}\PY{o}{*}\PY{o}{*}\PY{l+m+mi}{3}\PY{p}{,} \PY{l+s}{\PYZsq{}}\PY{l+s}{g\PYZca{}\PYZhy{}}\PY{l+s}{\PYZsq{}}\PY{p}{)}
\end{Verbatim}

            \begin{Verbatim}[commandchars=\\\{\}]
{\color{outcolor}Out[{\color{outcolor}36}]:} [<matplotlib.lines.Line2D at 0x10ff47c50>,
          <matplotlib.lines.Line2D at 0x10ff47dd0>,
          <matplotlib.lines.Line2D at 0x10ff8c290>]
\end{Verbatim}
        
    \begin{center}
    \adjustimage{max size={0.9\linewidth}{0.9\paperheight}}{Python-TP1_files/Python-TP1_118_1.png}
    \end{center}
    { \hspace*{\fill} \\}
    
    \subsubsection{Un histogramme et un affichage de texte sur le
graphique}\label{un-histogramme-et-un-affichage-de-texte-sur-le-graphique}

    \begin{Verbatim}[commandchars=\\\{\}]
{\color{incolor}In [{\color{incolor}37}]:} \PY{o}{\PYZpc{}}\PY{k}{matplotlib} inline
         
         \PY{n}{mu}\PY{p}{,} \PY{n}{sigma} \PY{o}{=} \PY{l+m+mi}{100}\PY{p}{,} \PY{l+m+mi}{15}
         \PY{c}{\PYZsh{}generation de 10000 nombres aleatoires}
         \PY{n}{x} \PY{o}{=} \PY{n}{mu} \PY{o}{+} \PY{n}{sigma} \PY{o}{*} \PY{n}{np}\PY{o}{.}\PY{n}{random}\PY{o}{.}\PY{n}{randn}\PY{p}{(}\PY{l+m+mi}{10000}\PY{p}{)}  
         
         \PY{n}{bins} \PY{o}{=} \PY{l+m+mi}{30}
         
         \PY{c}{\PYZsh{}  histogramme des donnees}
         \PY{n}{plt}\PY{o}{.}\PY{n}{hist}\PY{p}{(}\PY{n}{x}\PY{p}{,} \PY{n}{bins}\PY{p}{,} \PY{n}{normed}\PY{o}{=}\PY{l+m+mi}{1}\PY{p}{,} \PY{n}{facecolor}\PY{o}{=}\PY{l+s}{\PYZsq{}}\PY{l+s}{r}\PY{l+s}{\PYZsq{}}\PY{p}{,} \PY{n}{alpha}\PY{o}{=}\PY{l+m+mf}{0.8}\PY{p}{)}
         \PY{n}{plt}\PY{o}{.}\PY{n}{xlabel}\PY{p}{(}\PY{l+s}{\PYZsq{}}\PY{l+s}{Donnees}\PY{l+s}{\PYZsq{}}\PY{p}{)}
         \PY{n}{plt}\PY{o}{.}\PY{n}{ylabel}\PY{p}{(}\PY{l+s}{\PYZsq{}}\PY{l+s}{Probabilite}\PY{l+s}{\PYZsq{}}\PY{p}{)}
         \PY{n}{plt}\PY{o}{.}\PY{n}{title}\PY{p}{(}\PY{l+s}{\PYZsq{}}\PY{l+s}{Histogramme}\PY{l+s}{\PYZsq{}}\PY{p}{)}
         \PY{n}{plt}\PY{o}{.}\PY{n}{text}\PY{p}{(}\PY{l+m+mi}{60}\PY{p}{,} \PY{o}{.}\PY{l+m+mo}{025}\PY{p}{,} \PY{l+s}{r\PYZsq{}}\PY{l+s}{\PYZdl{}}\PY{l+s}{\PYZbs{}}\PY{l+s}{mu=100,}\PY{l+s}{\PYZbs{}}\PY{l+s}{ }\PY{l+s}{\PYZbs{}}\PY{l+s}{sigma=15\PYZdl{}}\PY{l+s}{\PYZsq{}}\PY{p}{)}
         \PY{n}{plt}\PY{o}{.}\PY{n}{axis}\PY{p}{(}\PY{p}{[}\PY{l+m+mi}{40}\PY{p}{,} \PY{l+m+mi}{160}\PY{p}{,} \PY{l+m+mi}{0}\PY{p}{,} \PY{l+m+mf}{0.03}\PY{p}{]}\PY{p}{)}
         \PY{n}{plt}\PY{o}{.}\PY{n}{grid}\PY{p}{(}\PY{n+nb+bp}{True}\PY{p}{)}
\end{Verbatim}

    \begin{center}
    \adjustimage{max size={0.9\linewidth}{0.9\paperheight}}{Python-TP1_files/Python-TP1_120_0.png}
    \end{center}
    { \hspace*{\fill} \\}
    
    \subsubsection{Questions :}\label{questions}

    D'abord, nous demandons d'écrire un script qui trace un graphique de la
pression (en kPa) en fonction du temps (en heures). N'oubliez pas
d'indiquer les étiquettes sur les axes.

    Sur le même graphique tracer aussi trois lignes (en couleurs
differentes) l'une représentant la valeur moyenne ( \(\mu\) ) de la
pression et les deux autres la valeurs moyennes plus / moins l'écart
type ( \(\sigma\) ).

    \begin{Verbatim}[commandchars=\\\{\}]
{\color{incolor}In [{\color{incolor}38}]:} \PY{k+kn}{from} \PY{n+nn}{IPython.display} \PY{k+kn}{import} \PY{n}{Image}
         \PY{n}{Image}\PY{p}{(}\PY{n}{filename}\PY{o}{=}\PY{l+s}{\PYZsq{}}\PY{l+s}{mean\PYZhy{}sigma.jpg}\PY{l+s}{\PYZsq{}}\PY{p}{)}
\end{Verbatim}
\texttt{\color{outcolor}Out[{\color{outcolor}38}]:}
    
    \begin{center}
    \adjustimage{max size={0.9\linewidth}{0.9\paperheight}}{Python-TP1_files/Python-TP1_124_0.jpg}
    \end{center}
    { \hspace*{\fill} \\}
    

    Deuxièmement, nous demandons également de construire un histogramme,
mais cette fois de la vitesse (en Km/h).


    % Add a bibliography block to the postdoc
    
    
    
    \end{document}
