
% Default to the notebook output style

    


% Inherit from the specified cell style.




    
\documentclass{article}

    
    
    \usepackage{graphicx} % Used to insert images
    \usepackage{adjustbox} % Used to constrain images to a maximum size 
    \usepackage{color} % Allow colors to be defined
    \usepackage{enumerate} % Needed for markdown enumerations to work
    \usepackage{geometry} % Used to adjust the document margins
    \usepackage{amsmath} % Equations
    \usepackage{amssymb} % Equations
    \usepackage{eurosym} % defines \euro
    \usepackage[mathletters]{ucs} % Extended unicode (utf-8) support
    \usepackage[utf8x]{inputenc} % Allow utf-8 characters in the tex document
    \usepackage{fancyvrb} % verbatim replacement that allows latex
    \usepackage{grffile} % extends the file name processing of package graphics 
                         % to support a larger range 
    % The hyperref package gives us a pdf with properly built
    % internal navigation ('pdf bookmarks' for the table of contents,
    % internal cross-reference links, web links for URLs, etc.)
    \usepackage{hyperref}
    \usepackage{longtable} % longtable support required by pandoc >1.10
    \usepackage{booktabs}  % table support for pandoc > 1.12.2
    

    
    
    \definecolor{orange}{cmyk}{0,0.4,0.8,0.2}
    \definecolor{darkorange}{rgb}{.71,0.21,0.01}
    \definecolor{darkgreen}{rgb}{.12,.54,.11}
    \definecolor{myteal}{rgb}{.26, .44, .56}
    \definecolor{gray}{gray}{0.45}
    \definecolor{lightgray}{gray}{.95}
    \definecolor{mediumgray}{gray}{.8}
    \definecolor{inputbackground}{rgb}{.95, .95, .85}
    \definecolor{outputbackground}{rgb}{.95, .95, .95}
    \definecolor{traceback}{rgb}{1, .95, .95}
    % ansi colors
    \definecolor{red}{rgb}{.6,0,0}
    \definecolor{green}{rgb}{0,.65,0}
    \definecolor{brown}{rgb}{0.6,0.6,0}
    \definecolor{blue}{rgb}{0,.145,.698}
    \definecolor{purple}{rgb}{.698,.145,.698}
    \definecolor{cyan}{rgb}{0,.698,.698}
    \definecolor{lightgray}{gray}{0.5}
    
    % bright ansi colors
    \definecolor{darkgray}{gray}{0.25}
    \definecolor{lightred}{rgb}{1.0,0.39,0.28}
    \definecolor{lightgreen}{rgb}{0.48,0.99,0.0}
    \definecolor{lightblue}{rgb}{0.53,0.81,0.92}
    \definecolor{lightpurple}{rgb}{0.87,0.63,0.87}
    \definecolor{lightcyan}{rgb}{0.5,1.0,0.83}
    
    % commands and environments needed by pandoc snippets
    % extracted from the output of `pandoc -s`
    \providecommand{\tightlist}{%
      \setlength{\itemsep}{0pt}\setlength{\parskip}{0pt}}
    \DefineVerbatimEnvironment{Highlighting}{Verbatim}{commandchars=\\\{\}}
    % Add ',fontsize=\small' for more characters per line
    \newenvironment{Shaded}{}{}
    \newcommand{\KeywordTok}[1]{\textcolor[rgb]{0.00,0.44,0.13}{\textbf{{#1}}}}
    \newcommand{\DataTypeTok}[1]{\textcolor[rgb]{0.56,0.13,0.00}{{#1}}}
    \newcommand{\DecValTok}[1]{\textcolor[rgb]{0.25,0.63,0.44}{{#1}}}
    \newcommand{\BaseNTok}[1]{\textcolor[rgb]{0.25,0.63,0.44}{{#1}}}
    \newcommand{\FloatTok}[1]{\textcolor[rgb]{0.25,0.63,0.44}{{#1}}}
    \newcommand{\CharTok}[1]{\textcolor[rgb]{0.25,0.44,0.63}{{#1}}}
    \newcommand{\StringTok}[1]{\textcolor[rgb]{0.25,0.44,0.63}{{#1}}}
    \newcommand{\CommentTok}[1]{\textcolor[rgb]{0.38,0.63,0.69}{\textit{{#1}}}}
    \newcommand{\OtherTok}[1]{\textcolor[rgb]{0.00,0.44,0.13}{{#1}}}
    \newcommand{\AlertTok}[1]{\textcolor[rgb]{1.00,0.00,0.00}{\textbf{{#1}}}}
    \newcommand{\FunctionTok}[1]{\textcolor[rgb]{0.02,0.16,0.49}{{#1}}}
    \newcommand{\RegionMarkerTok}[1]{{#1}}
    \newcommand{\ErrorTok}[1]{\textcolor[rgb]{1.00,0.00,0.00}{\textbf{{#1}}}}
    \newcommand{\NormalTok}[1]{{#1}}
    
    % Define a nice break command that doesn't care if a line doesn't already
    % exist.
    \def\br{\hspace*{\fill} \\* }
    % Math Jax compatability definitions
    \def\gt{>}
    \def\lt{<}
    % Document parameters
    \title{Python-TP2}
    
    
    

    % Pygments definitions
    
\makeatletter
\def\PY@reset{\let\PY@it=\relax \let\PY@bf=\relax%
    \let\PY@ul=\relax \let\PY@tc=\relax%
    \let\PY@bc=\relax \let\PY@ff=\relax}
\def\PY@tok#1{\csname PY@tok@#1\endcsname}
\def\PY@toks#1+{\ifx\relax#1\empty\else%
    \PY@tok{#1}\expandafter\PY@toks\fi}
\def\PY@do#1{\PY@bc{\PY@tc{\PY@ul{%
    \PY@it{\PY@bf{\PY@ff{#1}}}}}}}
\def\PY#1#2{\PY@reset\PY@toks#1+\relax+\PY@do{#2}}

\expandafter\def\csname PY@tok@gd\endcsname{\def\PY@tc##1{\textcolor[rgb]{0.63,0.00,0.00}{##1}}}
\expandafter\def\csname PY@tok@gu\endcsname{\let\PY@bf=\textbf\def\PY@tc##1{\textcolor[rgb]{0.50,0.00,0.50}{##1}}}
\expandafter\def\csname PY@tok@gt\endcsname{\def\PY@tc##1{\textcolor[rgb]{0.00,0.27,0.87}{##1}}}
\expandafter\def\csname PY@tok@gs\endcsname{\let\PY@bf=\textbf}
\expandafter\def\csname PY@tok@gr\endcsname{\def\PY@tc##1{\textcolor[rgb]{1.00,0.00,0.00}{##1}}}
\expandafter\def\csname PY@tok@cm\endcsname{\let\PY@it=\textit\def\PY@tc##1{\textcolor[rgb]{0.25,0.50,0.50}{##1}}}
\expandafter\def\csname PY@tok@vg\endcsname{\def\PY@tc##1{\textcolor[rgb]{0.10,0.09,0.49}{##1}}}
\expandafter\def\csname PY@tok@m\endcsname{\def\PY@tc##1{\textcolor[rgb]{0.40,0.40,0.40}{##1}}}
\expandafter\def\csname PY@tok@mh\endcsname{\def\PY@tc##1{\textcolor[rgb]{0.40,0.40,0.40}{##1}}}
\expandafter\def\csname PY@tok@go\endcsname{\def\PY@tc##1{\textcolor[rgb]{0.53,0.53,0.53}{##1}}}
\expandafter\def\csname PY@tok@ge\endcsname{\let\PY@it=\textit}
\expandafter\def\csname PY@tok@vc\endcsname{\def\PY@tc##1{\textcolor[rgb]{0.10,0.09,0.49}{##1}}}
\expandafter\def\csname PY@tok@il\endcsname{\def\PY@tc##1{\textcolor[rgb]{0.40,0.40,0.40}{##1}}}
\expandafter\def\csname PY@tok@cs\endcsname{\let\PY@it=\textit\def\PY@tc##1{\textcolor[rgb]{0.25,0.50,0.50}{##1}}}
\expandafter\def\csname PY@tok@cp\endcsname{\def\PY@tc##1{\textcolor[rgb]{0.74,0.48,0.00}{##1}}}
\expandafter\def\csname PY@tok@gi\endcsname{\def\PY@tc##1{\textcolor[rgb]{0.00,0.63,0.00}{##1}}}
\expandafter\def\csname PY@tok@gh\endcsname{\let\PY@bf=\textbf\def\PY@tc##1{\textcolor[rgb]{0.00,0.00,0.50}{##1}}}
\expandafter\def\csname PY@tok@ni\endcsname{\let\PY@bf=\textbf\def\PY@tc##1{\textcolor[rgb]{0.60,0.60,0.60}{##1}}}
\expandafter\def\csname PY@tok@nl\endcsname{\def\PY@tc##1{\textcolor[rgb]{0.63,0.63,0.00}{##1}}}
\expandafter\def\csname PY@tok@nn\endcsname{\let\PY@bf=\textbf\def\PY@tc##1{\textcolor[rgb]{0.00,0.00,1.00}{##1}}}
\expandafter\def\csname PY@tok@no\endcsname{\def\PY@tc##1{\textcolor[rgb]{0.53,0.00,0.00}{##1}}}
\expandafter\def\csname PY@tok@na\endcsname{\def\PY@tc##1{\textcolor[rgb]{0.49,0.56,0.16}{##1}}}
\expandafter\def\csname PY@tok@nb\endcsname{\def\PY@tc##1{\textcolor[rgb]{0.00,0.50,0.00}{##1}}}
\expandafter\def\csname PY@tok@nc\endcsname{\let\PY@bf=\textbf\def\PY@tc##1{\textcolor[rgb]{0.00,0.00,1.00}{##1}}}
\expandafter\def\csname PY@tok@nd\endcsname{\def\PY@tc##1{\textcolor[rgb]{0.67,0.13,1.00}{##1}}}
\expandafter\def\csname PY@tok@ne\endcsname{\let\PY@bf=\textbf\def\PY@tc##1{\textcolor[rgb]{0.82,0.25,0.23}{##1}}}
\expandafter\def\csname PY@tok@nf\endcsname{\def\PY@tc##1{\textcolor[rgb]{0.00,0.00,1.00}{##1}}}
\expandafter\def\csname PY@tok@si\endcsname{\let\PY@bf=\textbf\def\PY@tc##1{\textcolor[rgb]{0.73,0.40,0.53}{##1}}}
\expandafter\def\csname PY@tok@s2\endcsname{\def\PY@tc##1{\textcolor[rgb]{0.73,0.13,0.13}{##1}}}
\expandafter\def\csname PY@tok@vi\endcsname{\def\PY@tc##1{\textcolor[rgb]{0.10,0.09,0.49}{##1}}}
\expandafter\def\csname PY@tok@nt\endcsname{\let\PY@bf=\textbf\def\PY@tc##1{\textcolor[rgb]{0.00,0.50,0.00}{##1}}}
\expandafter\def\csname PY@tok@nv\endcsname{\def\PY@tc##1{\textcolor[rgb]{0.10,0.09,0.49}{##1}}}
\expandafter\def\csname PY@tok@s1\endcsname{\def\PY@tc##1{\textcolor[rgb]{0.73,0.13,0.13}{##1}}}
\expandafter\def\csname PY@tok@kd\endcsname{\let\PY@bf=\textbf\def\PY@tc##1{\textcolor[rgb]{0.00,0.50,0.00}{##1}}}
\expandafter\def\csname PY@tok@sh\endcsname{\def\PY@tc##1{\textcolor[rgb]{0.73,0.13,0.13}{##1}}}
\expandafter\def\csname PY@tok@sc\endcsname{\def\PY@tc##1{\textcolor[rgb]{0.73,0.13,0.13}{##1}}}
\expandafter\def\csname PY@tok@sx\endcsname{\def\PY@tc##1{\textcolor[rgb]{0.00,0.50,0.00}{##1}}}
\expandafter\def\csname PY@tok@bp\endcsname{\def\PY@tc##1{\textcolor[rgb]{0.00,0.50,0.00}{##1}}}
\expandafter\def\csname PY@tok@c1\endcsname{\let\PY@it=\textit\def\PY@tc##1{\textcolor[rgb]{0.25,0.50,0.50}{##1}}}
\expandafter\def\csname PY@tok@kc\endcsname{\let\PY@bf=\textbf\def\PY@tc##1{\textcolor[rgb]{0.00,0.50,0.00}{##1}}}
\expandafter\def\csname PY@tok@c\endcsname{\let\PY@it=\textit\def\PY@tc##1{\textcolor[rgb]{0.25,0.50,0.50}{##1}}}
\expandafter\def\csname PY@tok@mf\endcsname{\def\PY@tc##1{\textcolor[rgb]{0.40,0.40,0.40}{##1}}}
\expandafter\def\csname PY@tok@err\endcsname{\def\PY@bc##1{\setlength{\fboxsep}{0pt}\fcolorbox[rgb]{1.00,0.00,0.00}{1,1,1}{\strut ##1}}}
\expandafter\def\csname PY@tok@mb\endcsname{\def\PY@tc##1{\textcolor[rgb]{0.40,0.40,0.40}{##1}}}
\expandafter\def\csname PY@tok@ss\endcsname{\def\PY@tc##1{\textcolor[rgb]{0.10,0.09,0.49}{##1}}}
\expandafter\def\csname PY@tok@sr\endcsname{\def\PY@tc##1{\textcolor[rgb]{0.73,0.40,0.53}{##1}}}
\expandafter\def\csname PY@tok@mo\endcsname{\def\PY@tc##1{\textcolor[rgb]{0.40,0.40,0.40}{##1}}}
\expandafter\def\csname PY@tok@kn\endcsname{\let\PY@bf=\textbf\def\PY@tc##1{\textcolor[rgb]{0.00,0.50,0.00}{##1}}}
\expandafter\def\csname PY@tok@mi\endcsname{\def\PY@tc##1{\textcolor[rgb]{0.40,0.40,0.40}{##1}}}
\expandafter\def\csname PY@tok@gp\endcsname{\let\PY@bf=\textbf\def\PY@tc##1{\textcolor[rgb]{0.00,0.00,0.50}{##1}}}
\expandafter\def\csname PY@tok@o\endcsname{\def\PY@tc##1{\textcolor[rgb]{0.40,0.40,0.40}{##1}}}
\expandafter\def\csname PY@tok@kr\endcsname{\let\PY@bf=\textbf\def\PY@tc##1{\textcolor[rgb]{0.00,0.50,0.00}{##1}}}
\expandafter\def\csname PY@tok@s\endcsname{\def\PY@tc##1{\textcolor[rgb]{0.73,0.13,0.13}{##1}}}
\expandafter\def\csname PY@tok@kp\endcsname{\def\PY@tc##1{\textcolor[rgb]{0.00,0.50,0.00}{##1}}}
\expandafter\def\csname PY@tok@w\endcsname{\def\PY@tc##1{\textcolor[rgb]{0.73,0.73,0.73}{##1}}}
\expandafter\def\csname PY@tok@kt\endcsname{\def\PY@tc##1{\textcolor[rgb]{0.69,0.00,0.25}{##1}}}
\expandafter\def\csname PY@tok@ow\endcsname{\let\PY@bf=\textbf\def\PY@tc##1{\textcolor[rgb]{0.67,0.13,1.00}{##1}}}
\expandafter\def\csname PY@tok@sb\endcsname{\def\PY@tc##1{\textcolor[rgb]{0.73,0.13,0.13}{##1}}}
\expandafter\def\csname PY@tok@k\endcsname{\let\PY@bf=\textbf\def\PY@tc##1{\textcolor[rgb]{0.00,0.50,0.00}{##1}}}
\expandafter\def\csname PY@tok@se\endcsname{\let\PY@bf=\textbf\def\PY@tc##1{\textcolor[rgb]{0.73,0.40,0.13}{##1}}}
\expandafter\def\csname PY@tok@sd\endcsname{\let\PY@it=\textit\def\PY@tc##1{\textcolor[rgb]{0.73,0.13,0.13}{##1}}}

\def\PYZbs{\char`\\}
\def\PYZus{\char`\_}
\def\PYZob{\char`\{}
\def\PYZcb{\char`\}}
\def\PYZca{\char`\^}
\def\PYZam{\char`\&}
\def\PYZlt{\char`\<}
\def\PYZgt{\char`\>}
\def\PYZsh{\char`\#}
\def\PYZpc{\char`\%}
\def\PYZdl{\char`\$}
\def\PYZhy{\char`\-}
\def\PYZsq{\char`\'}
\def\PYZdq{\char`\"}
\def\PYZti{\char`\~}
% for compatibility with earlier versions
\def\PYZat{@}
\def\PYZlb{[}
\def\PYZrb{]}
\makeatother


    % Exact colors from NB
    \definecolor{incolor}{rgb}{0.0, 0.0, 0.5}
    \definecolor{outcolor}{rgb}{0.545, 0.0, 0.0}



    
    % Prevent overflowing lines due to hard-to-break entities
    \sloppy 
    % Setup hyperref package
    \hypersetup{
      breaklinks=true,  % so long urls are correctly broken across lines
      colorlinks=true,
      urlcolor=blue,
      linkcolor=darkorange,
      citecolor=darkgreen,
      }
    % Slightly bigger margins than the latex defaults
    
    \geometry{verbose,tmargin=1in,bmargin=1in,lmargin=1in,rmargin=1in}
    
    
\date{}
    \begin{document}
    
    
    \maketitle
    
    

    
    \subparagraph{TP 2 -- Informatique CM3 --}\label{tp-2-informatique-cm3}

    \section{Python @ Polytech'Lille}\label{python-polytechlille}

    Le texte de cette session de travaux pratiques est également disponible
ici:

https://github.com/ecalzavarini/python-at-polytech-lille/blob/master/Python-TP2.ipynb

    \subsubsection{Modalités pour accomplir le TP et compte
rendu}\label{modalituxe9s-pour-accomplir-le-tp-et-compte-rendu}

    Nous vous rappelons que les modalités pour accomplir ce TP sont les
mêmes de la première séance :

    Vous aurez à écrire plusieurs scripts (script1.py , script2.py ,
\ldots{}). Les scripts doivent être accompagnés par un document
descriptif unique (README.txt). Dans ce fichier, vous devrez décrire le
mode de fonctionnement des scripts et, si besoin, mettre vos
commentaires. Merci d'y écrire aussi vos noms et prenoms complets. Tous
les fichiers doivent être mis dans un dossier appelé TP2-nom1-nom2 et
ensuite être compressés dans un fichier archive TP2-nom1-nom2.tgz .

Enfin vous allez envoyer ce fichier par email à l'enseignant: soit
Enrico (enrico.calzavarini@polytech-lille.fr) soit Stefano
(stefano.berti@polytech-lille.fr).

    \paragraph{Vous avez une semaine de temps pour compléter le TP,
c'est-à-dire que la date limite pour envoyer vos travaux c'est dans 7
jours à partir
d'aujourd'hui.}\label{vous-avez-une-semaine-de-temps-pour-compluxe9ter-le-tp-cest-uxe0-dire-que-la-date-limite-pour-envoyer-vos-travaux-cest-dans-7-jours-uxe0-partir-daujourdhui.}

    \subsubsection{Gestion des accents en
Python}\label{gestion-des-accents-en-python}

    Dans la première séance de TP, vous avez souvent rencontré un message
d'erreur issu de l'utilisation des caractères accentués dans les scripts
en Python. Voici une solution à ce problème. En première ligne d'un
script, il faut insérer la ligne :

    \begin{Verbatim}[commandchars=\\\{\}]
{\color{incolor}In [{\color{incolor}1}]:} \PY{c}{\PYZsh{} \PYZhy{}*\PYZhy{} coding: utf\PYZhy{}8 \PYZhy{}*\PYZhy{}}
\end{Verbatim}

    ou bien

    \begin{Verbatim}[commandchars=\\\{\}]
{\color{incolor}In [{\color{incolor}2}]:} \PY{c}{\PYZsh{} \PYZhy{}*\PYZhy{} coding: latin\PYZhy{}1 \PYZhy{}*\PYZhy{}}
\end{Verbatim}

    Il s'agit d'un pseudo-commentaire indiquant à Python le système de
codage utilisé, ici l'utf-8 (ou le latin-1) qui comprend les caractères
spéciaux comme les caractères accentués, les apostrophes, les cédilles ,
etc .

    \subsubsection{Utiliser les fonctions en
Python}\label{utiliser-les-fonctions-en-python}

    Vous savez déjà ce qu'est une fonction mathématique d'une variable
réelle. Considérons par exemple la fonction f(x) suivante :

f : x ---\textgreater{} 2 x + 1

pour la définir en Python :

    \begin{Verbatim}[commandchars=\\\{\}]
{\color{incolor}In [{\color{incolor}3}]:} \PY{c}{\PYZsh{}definition d\PYZsq{}une fonction}
        \PY{k}{def} \PY{n+nf}{f}\PY{p}{(}\PY{n}{x}\PY{p}{)}\PY{p}{:}
            \PY{k}{return} \PY{l+m+mi}{2} \PY{o}{*} \PY{n}{x} \PY{o}{+} \PY{l+m+mi}{1}
        
        \PY{c}{\PYZsh{}utilisation de la fonction}
        
        \PY{k}{print}\PY{p}{(}\PY{n}{f}\PY{p}{(}\PY{l+m+mi}{4}\PY{p}{)}\PY{p}{)}
\end{Verbatim}

    \begin{Verbatim}[commandchars=\\\{\}]
9
    \end{Verbatim}

    Avez-vous remarqué qu'à la deuxième ligne, on n'a pas commencé à écrire
au début de la ligne? De la même façon que pour les structures de
contrôle `if' ou `for', nous devons appliquer la règle d'indentation.
Cette indentation est indispensable pour que l'interpréteur Python
comprenne la fin d'un bloc de définition d'une fonction.

    Le principe de définition de fonctions est intéressant pour deux raisons
:

\begin{enumerate}
\def\labelenumi{\arabic{enumi})}
\item
  cela nous permet de ne pas répéter un calcul long à taper,
\item
  Python possède un type spécial dédié au fonctions, que l'on peut donc
  manipuler, mettre dans des listes pour les étudier les unes à la suite
  des autres.
\end{enumerate}

Par exemple :

    \begin{Verbatim}[commandchars=\\\{\}]
{\color{incolor}In [{\color{incolor}4}]:} \PY{k}{print}\PY{p}{(} \PY{n}{f}\PY{p}{(}\PY{l+m+mi}{1}\PY{p}{)} \PY{p}{,} \PY{n}{f}\PY{p}{(}\PY{l+m+mi}{2}\PY{p}{)} \PY{p}{,} \PY{n}{f}\PY{p}{(}\PY{l+m+mi}{3}\PY{p}{)} \PY{p}{,} \PY{n}{f}\PY{p}{(}\PY{l+m+mi}{4}\PY{p}{)} \PY{p}{,} \PY{n}{f}\PY{p}{(}\PY{l+m+mi}{5}\PY{p}{)} \PY{p}{,} \PY{n}{f}\PY{p}{(}\PY{l+m+mi}{6}\PY{p}{)}\PY{p}{)}
\end{Verbatim}

    \begin{Verbatim}[commandchars=\\\{\}]
(3, 5, 7, 9, 11, 13)
    \end{Verbatim}

    \begin{Verbatim}[commandchars=\\\{\}]
{\color{incolor}In [{\color{incolor}5}]:} \PY{k}{print}\PY{p}{(} \PY{n+nb}{type}\PY{p}{(}\PY{n}{f}\PY{p}{)}\PY{p}{)}
\end{Verbatim}

    \begin{Verbatim}[commandchars=\\\{\}]
<type 'function'>
    \end{Verbatim}

    \begin{Verbatim}[commandchars=\\\{\}]
{\color{incolor}In [{\color{incolor}6}]:} \PY{c}{\PYZsh{} definition d\PYZsq{}une seconde fonction}
        \PY{k}{def} \PY{n+nf}{hi}\PY{p}{(}\PY{n}{name}\PY{p}{)}\PY{p}{:}
            \PY{k}{print}\PY{p}{(}\PY{l+s}{\PYZdq{}}\PY{l+s}{hello }\PY{l+s}{\PYZdq{}} \PY{o}{+} \PY{n}{name} \PY{o}{+} \PY{l+s}{\PYZdq{}}\PY{l+s}{ from Python!!!}\PY{l+s}{\PYZdq{}}\PY{p}{)}
        
        \PY{c}{\PYZsh{}exemple d\PYZsq{}utilisation    }
        \PY{n}{hi}\PY{p}{(}\PY{l+s}{\PYZdq{}}\PY{l+s}{Mark}\PY{l+s}{\PYZdq{}}\PY{p}{)}    
\end{Verbatim}

    \begin{Verbatim}[commandchars=\\\{\}]
hello Mark from Python!!!
    \end{Verbatim}

    \begin{Verbatim}[commandchars=\\\{\}]
{\color{incolor}In [{\color{incolor}7}]:} \PY{c}{\PYZsh{}definition  d\PYZsq{}une trosieme fonction}
        \PY{k+kn}{from} \PY{n+nn}{random} \PY{k+kn}{import} \PY{n}{choice}
        \PY{k}{def} \PY{n+nf}{lettre}\PY{p}{(}\PY{p}{)}\PY{p}{:}
            \PY{k}{return} \PY{n}{choice}\PY{p}{(}\PY{l+s}{\PYZsq{}}\PY{l+s}{abcdefghijklmnopqrstuvwxyz}\PY{l+s}{\PYZsq{}}\PY{p}{)}
        
        \PY{c}{\PYZsh{}exemple d\PYZsq{}utilisation }
        \PY{n}{lettre}\PY{p}{(}\PY{p}{)}
\end{Verbatim}

            \begin{Verbatim}[commandchars=\\\{\}]
{\color{outcolor}Out[{\color{outcolor}7}]:} 'm'
\end{Verbatim}
        
    \begin{Verbatim}[commandchars=\\\{\}]
{\color{incolor}In [{\color{incolor}8}]:} \PY{c}{\PYZsh{}definition d\PYZsq{}une liste de fonctions}
        \PY{n}{mes\PYZus{}fonctions} \PY{o}{=} \PY{p}{[}\PY{n}{f}\PY{p}{,}\PY{n}{hi}\PY{p}{,}\PY{n}{lettre}\PY{p}{]}
        
        \PY{c}{\PYZsh{}utilisation}
        \PY{n}{mes\PYZus{}fonctions}\PY{p}{[}\PY{l+m+mi}{1}\PY{p}{]}\PY{p}{(}\PY{l+s}{\PYZdq{}}\PY{l+s}{Dominique}\PY{l+s}{\PYZdq{}}\PY{p}{)}
        
        \PY{n}{mes\PYZus{}fonctions}\PY{p}{[}\PY{l+m+mi}{2}\PY{p}{]}\PY{p}{(}\PY{p}{)}
\end{Verbatim}

    \begin{Verbatim}[commandchars=\\\{\}]
hello Dominique from Python!!!
    \end{Verbatim}

            \begin{Verbatim}[commandchars=\\\{\}]
{\color{outcolor}Out[{\color{outcolor}8}]:} 'g'
\end{Verbatim}
        
    \subsection{Script 1 : calcul des forces sur un ballon de
football}\label{script-1-calcul-des-forces-sur-un-ballon-de-football}

    \begin{Verbatim}[commandchars=\\\{\}]
{\color{incolor}In [{\color{incolor}9}]:} \PY{k+kn}{from} \PY{n+nn}{IPython.display} \PY{k+kn}{import} \PY{n}{Image}
        \PY{n}{Image}\PY{p}{(}\PY{n}{filename}\PY{o}{=}\PY{l+s}{\PYZsq{}}\PY{l+s}{kick.jpg}\PY{l+s}{\PYZsq{}}\PY{p}{)}
\end{Verbatim}
\texttt{\color{outcolor}Out[{\color{outcolor}9}]:}
    
    \begin{center}
    \adjustimage{max size={0.9\linewidth}{0.9\paperheight}}{Python-TP2_files/Python-TP2_24_0.jpg}
    \end{center}
    { \hspace*{\fill} \\}
    

    Les forces sur un ballon de football en vol suite à un coup de pied d'un
joueur sont deux: la force de la pesanteur (le poids \(F_P\)) et la
force de traînée exercée par le frottement de l'air sur le ballon
(\(F_T\)). Leurs expressions sont les suivantes :

    \[F_P = M\ g\]

\[F_T = 0.5 \ C_D \ \rho \ u^2\ S \]

    Où \(M\) est la masse du ballon, \(g\) l'accélération de gravité,
\(C_D\) le coefficient de traînée, \(\rho\) la masse volumique de l'air,
\(u\) la vitesse du ballon et enfin \(S\) la section du ballon
\(S=\pi\ R^2\) (avec R le rayon).

    Nous demandons d'écrire un script qui tout d'abord demande à
l'utilisateur d'entrer les valeurs du rayon du ballon (en mètres), de la
masse du ballon (en kg) et de la vitesse du ballon ainsi que l'unité de
mesure pour cette dernière (soit ``m/s'' ou ``km/h''). Ensuite le script
devra calculer les forces \(F_P\) et \(F_T\), afficher les deux valeurs
ainsi que leur rapport \(F_T/F_P\).

    On demande de faire tout cela à l'aide de quatre fonctions :

\begin{enumerate}
\def\labelenumi{\arabic{enumi})}
\item
  une fonction qui convertit la vitesse de ``km/h'' en ``m/s'' si besoin
\item
  une fonction qui calcule la surface \(S\) du ballon à partir du rayon
  \(R\)
\item
  une fonction qui calcule \(F_T\)
\item
  une fonction qui calcule \(F_P\)
\end{enumerate}

    Les données du problème sont l'accélération de gravité
\(g=9.81 \, m /s^2\), la masse volumique de l'air
\(\rho = 1.2 \, kg/m^{-3}\), le coefficient de trainée \(C_D=0.2\).
Tourner le script plusieurs fois et dire ce qui change dans le rapport
\(F_T/F_P\) pour des valeurs de vitesse croissantes (p.ex.
\(u=10 \, km/h\), \(u=120 \, km/h\)) et à masse et rayon fixes.

    \subsection{Script 2 : étude de la fonction de transfert du système
ressort - amortisseur (suspension) d'un
automobile}\label{script-2-uxe9tude-de-la-fonction-de-transfert-du-systuxe8me-ressort---amortisseur-suspension-dun-automobile}

    L'objet de ce script est d'illustrer à l'aide de Python les rôles
respectifs joués par le ressort et l'amortisseur d'un système
automobile. Pour ce faire, nous étudions un ressort couplé à un
amortisseur en parallèle.

Comme dans la figure ci-dessous:

    \begin{Verbatim}[commandchars=\\\{\}]
{\color{incolor}In [{\color{incolor}10}]:} \PY{k+kn}{from} \PY{n+nn}{IPython.display} \PY{k+kn}{import} \PY{n}{Image}
         \PY{n}{Image}\PY{p}{(}\PY{n}{filename}\PY{o}{=}\PY{l+s}{\PYZsq{}}\PY{l+s}{masse\PYZhy{}ressort\PYZhy{}amortisseur.jpg}\PY{l+s}{\PYZsq{}}\PY{p}{)}
\end{Verbatim}
\texttt{\color{outcolor}Out[{\color{outcolor}10}]:}
    
    \begin{center}
    \adjustimage{max size={0.9\linewidth}{0.9\paperheight}}{Python-TP2_files/Python-TP2_33_0.jpg}
    \end{center}
    { \hspace*{\fill} \\}
    

    L'équation vérifiée par le système est la suivante :

\[ \ddot{x} + \frac{\omega_0}{Q} \dot{x} + \omega_0^2\ x = \frac{F}{m} \cos( \omega t) \]

    Ici \(m\) est la masse de la roue, \(F\) et \(\omega\) sont
respectivement l'intensité de la force appliquée (qui modélise l'effet
des ondulations du sol sur la roue) et sa pulsation , \(\omega_0\) la
pulsation caractéristique du ressort (\(\omega_0=\sqrt{k/m}\)) et enfin
\(Q\) est un coefficient appelé coefficient de qualité. Ici on négligera
le poids de la roue.

Un système très amorti a un \(Q\) faible. À l'inverse, un \(Q\) élevé
correspond à un système peu amorti. Pour fixer les idées, le \(Q\) d'une
voiture avec des amortisseurs en bon état est légèrement supérieur à 1.

    De la solution de l'équation du système ressort-amortisseur on trouve la
fonction de transfert qui décrit la réponse du système en fonction de la
pulsation \(\omega\).

En particulier la fonction de transfert (le module de cette fonction
pour la précision) s'écrit :

    \[ T(\omega) = \frac{F}{m \ \omega_0^2} \frac{1}{\sqrt{ \left( 1 - \omega^2/\omega_0^2 \right)^2 +  Q^{-2} \ \omega^2/\omega_0^2 }} \]

    ou en utilisant la pulsation adimensionnée u = ω/ω0 :

    \[ T(u) = \frac{F}{m \ \omega_0^2} \frac{1}{\sqrt{ \left( 1 - u^2 \right)^2 +  u^{2}/ \      Q^2}} \]

    Nous demandons d'écrire un script qui trace un graphique du module de la
fonction de transfert \(T(u)\) en fonction de la pulsation adimensionnée
\(u\) (dans l'intervalle \([0,2]\)) pour toutes les valeurs de \(Q\)
dans l'intervalle \([0,10]\) avec rapport entre une valeur de \(Q\) et
la valeur précédente de \(\delta Q = 2.0\) (facteur de redoublement).

    Prendre la valeur suivante pour l'amplitude de la fonction de transfert
\[ \frac{F}{m \omega_0^2} = 2 \]

    \subparagraph{Ça pourrait être utile
:}\label{uxe7a-pourrait-uxeatre-utile}

    \paragraph{La boucle while}\label{la-boucle-while}

    Le but de la boucle ``while'' est de répéter certaines instructions tant
qu'une condition est respectée. On n'est pas donc obligé de savoir au
départ le nombre de répétitions à faire.

    \begin{Verbatim}[commandchars=\\\{\}]
{\color{incolor}In [{\color{incolor}11}]:} \PY{n}{nb\PYZus{}repetitions} \PY{o}{=} \PY{l+m+mi}{3}
         \PY{n}{i} \PY{o}{=} \PY{l+m+mi}{1}
         
         \PY{k}{while} \PY{n}{i} \PY{o}{\PYZlt{}}\PY{o}{=} \PY{n}{nb\PYZus{}repetitions} \PY{p}{:}
             \PY{k}{print} \PY{l+s}{\PYZdq{}}\PY{l+s}{Et }\PY{l+s}{\PYZdq{}}\PY{o}{+}\PY{n+nb}{str}\PY{p}{(}\PY{n}{i}\PY{p}{)}\PY{o}{+}\PY{l+s}{\PYZdq{}}\PY{l+s}{!}\PY{l+s}{\PYZdq{}}
             \PY{n}{i} \PY{o}{=} \PY{n}{i}\PY{o}{+}\PY{l+m+mi}{1}
         \PY{k}{print} \PY{l+s}{\PYZdq{}}\PY{l+s}{Zéro!}\PY{l+s}{\PYZdq{}}
\end{Verbatim}

    \begin{Verbatim}[commandchars=\\\{\}]
Et 1!
Et 2!
Et 3!
Zéro!
    \end{Verbatim}

    L'instruction ``break'' sert, non pas à interrompre le programme, mais à
sortir de la boucle.

    \begin{Verbatim}[commandchars=\\\{\}]
{\color{incolor}In [{\color{incolor}12}]:} \PY{n}{nb\PYZus{}repetitions} \PY{o}{=} \PY{l+m+mi}{3}
         \PY{n}{i} \PY{o}{=} \PY{l+m+mi}{1}
         
         \PY{k}{while} \PY{n}{i} \PY{o}{\PYZlt{}}\PY{o}{=} \PY{n}{nb\PYZus{}repetitions} \PY{p}{:}
             \PY{k}{print} \PY{l+s}{\PYZdq{}}\PY{l+s}{Et }\PY{l+s}{\PYZdq{}}\PY{o}{+}\PY{n+nb}{str}\PY{p}{(}\PY{n}{i}\PY{p}{)}\PY{o}{+}\PY{l+s}{\PYZdq{}}\PY{l+s}{!}\PY{l+s}{\PYZdq{}}
             \PY{n}{i} \PY{o}{=} \PY{n}{i}\PY{o}{+}\PY{l+m+mi}{1}
             \PY{k}{if} \PY{n}{i}\PY{o}{==}\PY{l+m+mi}{3}\PY{p}{:} 
                 \PY{k}{break}
         \PY{k}{print} \PY{l+s}{\PYZdq{}}\PY{l+s}{Zéro!}\PY{l+s}{\PYZdq{}}
\end{Verbatim}

    \begin{Verbatim}[commandchars=\\\{\}]
Et 1!
Et 2!
Zéro!
    \end{Verbatim}

    \paragraph{Création d'une légende sur un
graphique}\label{cruxe9ation-dune-luxe9gende-sur-un-graphique}

    \begin{Verbatim}[commandchars=\\\{\}]
{\color{incolor}In [{\color{incolor}13}]:} \PY{o}{\PYZpc{}} \PY{n}{matplotlib} \PY{n}{inline}  
         
         \PY{k+kn}{from} \PY{n+nn}{math} \PY{k+kn}{import} \PY{o}{*}
         \PY{k+kn}{import} \PY{n+nn}{numpy} \PY{k+kn}{as} \PY{n+nn}{np}
         \PY{k+kn}{import} \PY{n+nn}{matplotlib.pyplot} \PY{k+kn}{as} \PY{n+nn}{plt}
         
         \PY{c}{\PYZsh{} Création d\PYZsq{}une array numpy : Temps = Abscisses }
         \PY{n}{t} \PY{o}{=} \PY{n}{np}\PY{o}{.}\PY{n}{linspace}\PY{p}{(}\PY{l+m+mi}{0}\PY{p}{,}\PY{l+m+mi}{10}\PY{p}{,}\PY{l+m+mi}{400}\PY{p}{)}    
         
         \PY{c}{\PYZsh{} Fonction U dépendant d\PYZsq{}un paramètre Q}
         \PY{k}{def} \PY{n+nf}{U}\PY{p}{(}\PY{n}{t}\PY{p}{,}\PY{n}{Q}\PY{p}{)}\PY{p}{:}
             \PY{k}{return} \PY{n}{np}\PY{o}{.}\PY{n}{exp}\PY{p}{(}\PY{o}{\PYZhy{}}\PY{l+m+mi}{2}\PY{o}{/}\PY{n}{Q}\PY{o}{*}\PY{n}{t}\PY{p}{)}\PY{o}{*}\PY{n}{np}\PY{o}{.}\PY{n}{cos}\PY{p}{(}\PY{l+m+mi}{2}\PY{o}{*}\PY{n}{pi}\PY{o}{*}\PY{n}{t}\PY{p}{)}  
         
         \PY{c}{\PYZsh{} Plot1 avec Q = 2 et édition du label correspondant}
         \PY{n}{Q}\PY{o}{=}\PY{l+m+mf}{2.0}
         \PY{n}{plt}\PY{o}{.}\PY{n}{plot}\PY{p}{(}\PY{n}{t}\PY{p}{,}\PY{n}{U}\PY{p}{(}\PY{n}{t}\PY{p}{,}\PY{n}{Q}\PY{p}{)}\PY{p}{,}\PY{n}{label}\PY{o}{=}\PY{l+s}{\PYZdq{}}\PY{l+s}{Q=}\PY{l+s}{\PYZdq{}} \PY{o}{+} \PY{n+nb}{str}\PY{p}{(}\PY{n}{Q}\PY{p}{)}\PY{p}{)} 
         
         \PY{c}{\PYZsh{} Plot2 avec Q = 10 et édition du label correspondant}
         \PY{n}{Q}\PY{o}{=}\PY{l+m+mf}{10.0}
         \PY{n}{plt}\PY{o}{.}\PY{n}{plot}\PY{p}{(}\PY{n}{t}\PY{p}{,}\PY{n}{U}\PY{p}{(}\PY{n}{t}\PY{p}{,}\PY{n}{Q}\PY{p}{)}\PY{p}{,}\PY{n}{label}\PY{o}{=}\PY{l+s}{\PYZdq{}}\PY{l+s}{Q=}\PY{l+s}{\PYZdq{}} \PY{o}{+} \PY{n+nb}{str}\PY{p}{(}\PY{n}{Q}\PY{p}{)}\PY{p}{)} 
         
         \PY{c}{\PYZsh{} Appel de la légende}
         \PY{n}{plt}\PY{o}{.}\PY{n}{legend}\PY{p}{(}\PY{n}{loc}\PY{o}{=}\PY{l+s}{\PYZsq{}}\PY{l+s}{upper right}\PY{l+s}{\PYZsq{}}\PY{p}{)}    
         
         \PY{n}{plt}\PY{o}{.}\PY{n}{show}\PY{p}{(}\PY{p}{)}
\end{Verbatim}

    \begin{center}
    \adjustimage{max size={0.9\linewidth}{0.9\paperheight}}{Python-TP2_files/Python-TP2_49_0.png}
    \end{center}
    { \hspace*{\fill} \\}
    
    \subsection{Script 3 : Calcul de la valeur critique du coefficient de
qualité d'une
suspension}\label{script-3-calcul-de-la-valeur-critique-du-coefficient-de-qualituxe9-dune-suspension}

    Écrire un script qui calcule la valeur critique de \(Q\), c'est-à-dire
la valeur pour laquelle on observe une transition entre le régime de
vibrations sur-amorties (\(T(u)\) non croissante) et celui de résonance
(\(T(u)\) non monotone, avec un maximum pour \(0< u < 1\)).

    Pour cela faire, on peut par exemple calculer la valeur maximale
\(T_{max}\) de \(T(u)\) pour des valeurs croissantes de \(Q\), et
arrêter le calcul lorsque \(T_{max}\) est supérieur à
\(F / (m \omega_0^2)\).

Dans ce cas on vous suggère de prendre \(Q=0.5\) comme valeur de départ
et de l'incrémenter pas à pas d'une valeur \(\delta Q = 10^{-6}\).

    Utiliser encore : \(F /(m \omega_0^2) = 2\)

    \subsubsection{Calcul de la valeur maximale avec
numpy}\label{calcul-de-la-valeur-maximale-avec-numpy}

    Pour ce script, il peut être utile d'utiliser la fonction numpy pour
calculer le maximum d'une liste :

    \begin{Verbatim}[commandchars=\\\{\}]
{\color{incolor}In [{\color{incolor}14}]:}  \PY{k+kn}{import} \PY{n+nn}{numpy} \PY{k+kn}{as} \PY{n+nn}{np}
          
          \PY{n}{val}\PY{o}{=}\PY{n}{np}\PY{o}{.}\PY{n}{linspace}\PY{p}{(}\PY{l+m+mi}{0}\PY{p}{,}\PY{l+m+mi}{7}\PY{p}{,}\PY{l+m+mi}{500}\PY{p}{)}
          
          \PY{n}{val\PYZus{}max} \PY{o}{=} \PY{n}{np}\PY{o}{.}\PY{n}{amax}\PY{p}{(}\PY{n}{val}\PY{p}{)}
             
          \PY{k}{print}\PY{p}{(}\PY{n}{val\PYZus{}max}\PY{p}{)}
\end{Verbatim}

    \begin{Verbatim}[commandchars=\\\{\}]
7.0
    \end{Verbatim}

    \subsection{Script 4 : Préparation des données d'entrée pour une
machine-outil à contrôle
numérique}\label{script-4-pruxe9paration-des-donnuxe9es-dentruxe9e-pour-une-machine-outil-uxe0-contruxf4le-numuxe9rique}

    Les machines-outil d'usinage à contrôle numérique peuvent être
contrôlées à travers un fichier de données d'entrée. Ce fichier contient
toutes les spécifications géométriques de l'objet à usiner et il a
souvent la forme d'une matrice en deux dimensions, dans laquelle chaque
élément identifie une coordonnée spatiale (x, y) alors que la valeur de
l'élément contient l'information sur l'usinage (par exemple dans une
fraiseuse ``\(0\)'' signifie couper et ``\(1\)'' signifie garder tel
qu'il est).

Nous considérons ici une machine-outil fraiseuse avec résolution du
dixième de millimètre, et une surface de travail de
\((10 \times 10) cm^2\). Nous voulons découper de la pièce brute une
bielle à partir d'un bloc de métal de dimensions \((10 \times 10) cm^2\)
(l'épaisseur du bloc n'a pas d' importance ici).

La géométrie de la bielle est montrée dans la figure ci-dessous :

    \begin{Verbatim}[commandchars=\\\{\}]
{\color{incolor}In [{\color{incolor}15}]:} \PY{k+kn}{from} \PY{n+nn}{IPython.display} \PY{k+kn}{import} \PY{n}{Image}
         \PY{n}{Image}\PY{p}{(}\PY{n}{filename}\PY{o}{=}\PY{l+s}{\PYZsq{}}\PY{l+s}{bielle.jpg}\PY{l+s}{\PYZsq{}}\PY{p}{)}
\end{Verbatim}
\texttt{\color{outcolor}Out[{\color{outcolor}15}]:}
    
    \begin{center}
    \adjustimage{max size={0.9\linewidth}{0.9\paperheight}}{Python-TP2_files/Python-TP2_59_0.jpg}
    \end{center}
    { \hspace*{\fill} \\}
    

    Détails de la géométrie de la bielle:

\begin{itemize}
\item
  le grand anneau a rayon interne \(r_1 = 10.0\ mm\), rayon externe
  \(r_2 = 16.0\ mm\) , position du centre \(x_1 = 50.0 \ mm\) ,
  \(y_1 = 20.0 \ mm\).
\item
  le petit anneau a rayon interne \(r_3 = 3.0\ mm\) , rayon externe
  \(r_4 = 11.2\ mm\) , position du centre \(x_2 = 50.0 \ mm\) ,
  \(y_2 = 75.0 \ mm\).
\item
  la barre de liaison a longueur \(l=30.0\ mm\) , hauteur
  \(h = 10.0\ mm\) et centre géométrique en \(x_3 = 50.0 \ mm\),
  \(y_3 = 50.0 \ mm\).
\end{itemize}

    On demande d'écrire un script qui crée la matrice pour usiner la bielle
en figure.

Cette matrice devra avoir \(1000\) x \(1000\) éléments, chaque élément
correspondant à \((0.1 \times 0.1) mm^2\), et la valeur des éléments
doit être soit \(0\) si le matériau doit être enlevé (ou bien
complètement fraisé dans ce cas) soit \(1\) s'il doit être gardé comme
il est.

\begin{enumerate}
\def\labelenumi{\alph{enumi})}
\item
  Montrer à travers un graphique la matrice que vous avez obtenu
\item
  Tracer aussi un graphique de la section longitudinale centrale de la
  bielle.
\end{enumerate}

    \subparagraph{Cela pourrait vous être utile
:}\label{cela-pourrait-vous-uxeatre-utile}

    \paragraph{Créer et manipuler un tableau multidimensionnel avec
numpy}\label{cruxe9er-et-manipuler-un-tableau-multidimensionnel-avec-numpy}

    \begin{Verbatim}[commandchars=\\\{\}]
{\color{incolor}In [{\color{incolor}16}]:} \PY{k+kn}{import} \PY{n+nn}{numpy} \PY{k+kn}{as} \PY{n+nn}{np}
         
         \PY{c}{\PYZsh{}définition d\PYZsq{}une matrice avec numpy}
         \PY{n}{tableau} \PY{o}{=} \PY{n}{np}\PY{o}{.}\PY{n}{array} \PY{p}{(}\PY{p}{[}\PY{p}{[}\PY{l+m+mi}{0} \PY{p}{,}\PY{l+m+mi}{1} \PY{p}{,}\PY{l+m+mi}{2}\PY{p}{]} \PY{p}{,}\PY{p}{[}\PY{l+m+mi}{1}\PY{p}{,}\PY{l+m+mi}{2}\PY{p}{,}\PY{l+m+mi}{3}\PY{p}{]}\PY{p}{,} \PY{p}{[}\PY{l+m+mi}{4}\PY{p}{,}\PY{l+m+mi}{6}\PY{p}{,}\PY{l+m+mi}{12}\PY{p}{]}\PY{p}{,} \PY{p}{[}\PY{l+m+mi}{44} \PY{p}{,}\PY{l+m+mi}{55} \PY{p}{,}\PY{l+m+mi}{56}\PY{p}{]}\PY{p}{]}\PY{p}{)} 
         
         \PY{k}{print}\PY{p}{(}\PY{n}{tableau}\PY{p}{)}
\end{Verbatim}

    \begin{Verbatim}[commandchars=\\\{\}]
[[ 0  1  2]
 [ 1  2  3]
 [ 4  6 12]
 [44 55 56]]
    \end{Verbatim}

    \begin{Verbatim}[commandchars=\\\{\}]
{\color{incolor}In [{\color{incolor}17}]:} \PY{c}{\PYZsh{} copier un element du tableau, tableau[num. ligne][num. colonne]}
         \PY{n}{n}\PY{o}{=}\PY{n}{tableau}\PY{p}{[}\PY{l+m+mi}{3}\PY{p}{,}\PY{l+m+mi}{0}\PY{p}{]} 
         
         \PY{k}{print}\PY{p}{(}\PY{n}{n}\PY{p}{)} 
\end{Verbatim}

    \begin{Verbatim}[commandchars=\\\{\}]
44
    \end{Verbatim}

    Noter que l'indice de gauche indique toujours le nombre de la ligne
tandis que l'indice de droite indique le numéro de la colonne.

Ce système d'indexage (dite convention ``row-major order'' ou ``ligne
par ligne'') n'est pas toujours intuitif : dans le cas d'une matrice
\(A[i, j]\) qui contient des informations spatiales l'indice à gauche (
\(i\) ) identifie la coordonnée verticale \(y\) et l'indice de droite (
\(j\) ) identifie la coordonnée horizontale \(x\).

    \begin{Verbatim}[commandchars=\\\{\}]
{\color{incolor}In [{\color{incolor}18}]:} \PY{n}{c}\PY{o}{=}\PY{n}{tableau}\PY{p}{[}\PY{p}{:}\PY{p}{,}\PY{l+m+mi}{0}\PY{p}{]} \PY{c}{\PYZsh{} copier une colonne du tableau}
         
         \PY{k}{print}\PY{p}{(}\PY{n}{c}\PY{p}{)}
\end{Verbatim}

    \begin{Verbatim}[commandchars=\\\{\}]
[ 0  1  4 44]
    \end{Verbatim}

    \begin{Verbatim}[commandchars=\\\{\}]
{\color{incolor}In [{\color{incolor}19}]:} \PY{n}{d}\PY{o}{=}\PY{n}{tableau}\PY{p}{[}\PY{p}{:}\PY{l+m+mi}{2}\PY{p}{,}\PY{p}{:}\PY{l+m+mi}{2}\PY{p}{]} \PY{c}{\PYZsh{} copier une colonne du tableau}
         
         \PY{k}{print}\PY{p}{(}\PY{n}{d}\PY{p}{)}
\end{Verbatim}

    \begin{Verbatim}[commandchars=\\\{\}]
[[0 1]
 [1 2]]
    \end{Verbatim}

    \begin{Verbatim}[commandchars=\\\{\}]
{\color{incolor}In [{\color{incolor}20}]:} \PY{n}{tableau} \PY{o}{=} \PY{n}{tableau} \PY{o}{+} \PY{l+m+mi}{2}  \PY{c}{\PYZsh{}ajouter 2 aux éléments du tableau}
         \PY{k}{print}\PY{p}{(}\PY{n}{tableau}\PY{p}{)}
\end{Verbatim}

    \begin{Verbatim}[commandchars=\\\{\}]
[[ 2  3  4]
 [ 3  4  5]
 [ 6  8 14]
 [46 57 58]]
    \end{Verbatim}

    \begin{Verbatim}[commandchars=\\\{\}]
{\color{incolor}In [{\color{incolor}21}]:} \PY{n}{tableau2} \PY{o}{=} \PY{n}{np}\PY{o}{.}\PY{n}{zeros}\PY{p}{(}\PY{p}{(}\PY{l+m+mi}{4}\PY{p}{,}\PY{l+m+mi}{3}\PY{p}{)}\PY{p}{)}  \PY{c}{\PYZsh{}définition d\PYZsq{}une matrice composée de zéros}
         
         \PY{k}{print}\PY{p}{(}\PY{n}{tableau2}\PY{p}{)}
\end{Verbatim}

    \begin{Verbatim}[commandchars=\\\{\}]
[[ 0.  0.  0.]
 [ 0.  0.  0.]
 [ 0.  0.  0.]
 [ 0.  0.  0.]]
    \end{Verbatim}

    \paragraph{Représenter graphiquement une
matrice}\label{repruxe9senter-graphiquement-une-matrice}

    \begin{Verbatim}[commandchars=\\\{\}]
{\color{incolor}In [{\color{incolor}22}]:} \PY{k+kn}{import} \PY{n+nn}{numpy} \PY{k+kn}{as} \PY{n+nn}{np}
         \PY{k+kn}{import} \PY{n+nn}{matplotlib.pyplot} \PY{k+kn}{as} \PY{n+nn}{plt}
         
         \PY{c}{\PYZsh{}création d\PYZsq{}une matrice aléatoire de 80x50 elements}
         \PY{n}{my\PYZus{}mat} \PY{o}{=} \PY{n}{np}\PY{o}{.}\PY{n}{random}\PY{o}{.}\PY{n}{random}\PY{p}{(}\PY{p}{(}\PY{l+m+mi}{50}\PY{p}{,} \PY{l+m+mi}{80}\PY{p}{)}\PY{p}{)} 
         
         \PY{c}{\PYZsh{} représentation graphique}
         \PY{n}{plt}\PY{o}{.}\PY{n}{matshow}\PY{p}{(}\PY{n}{my\PYZus{}mat}\PY{p}{)} 
         
         \PY{c}{\PYZsh{} définition des marqueurs en x}
         \PY{n}{plt}\PY{o}{.}\PY{n}{xticks}\PY{p}{(}\PY{n+nb}{range}\PY{p}{(}\PY{l+m+mi}{0}\PY{p}{,}\PY{l+m+mi}{80}\PY{p}{,}\PY{l+m+mi}{10}\PY{p}{)}\PY{p}{)} 
         \PY{c}{\PYZsh{} définition des marqueurs en y}
         \PY{n}{plt}\PY{o}{.}\PY{n}{yticks}\PY{p}{(}\PY{n+nb}{range}\PY{p}{(}\PY{l+m+mi}{0}\PY{p}{,}\PY{l+m+mi}{50}\PY{p}{,}\PY{l+m+mi}{10}\PY{p}{)}\PY{p}{)} 
         
         \PY{n}{plt}\PY{o}{.}\PY{n}{colorbar}\PY{p}{(}\PY{p}{)}
         
         \PY{n}{plt}\PY{o}{.}\PY{n}{show}\PY{p}{(}\PY{p}{)}        
\end{Verbatim}

    \begin{center}
    \adjustimage{max size={0.9\linewidth}{0.9\paperheight}}{Python-TP2_files/Python-TP2_72_0.png}
    \end{center}
    { \hspace*{\fill} \\}
    
    \begin{Verbatim}[commandchars=\\\{\}]
{\color{incolor}In [{\color{incolor}23}]:} \PY{c}{\PYZsh{} représentation graphique en blanc et noir}
         \PY{n}{plt}\PY{o}{.}\PY{n}{matshow}\PY{p}{(}\PY{n}{my\PYZus{}mat}\PY{p}{,}\PY{n}{cmap}\PY{o}{=}\PY{l+s}{\PYZsq{}}\PY{l+s}{Greys}\PY{l+s}{\PYZsq{}}\PY{p}{)} 
         \PY{n}{plt}\PY{o}{.}\PY{n}{colorbar}\PY{p}{(}\PY{p}{)}
         \PY{n}{plt}\PY{o}{.}\PY{n}{show}\PY{p}{(}\PY{p}{)}        
\end{Verbatim}

    \begin{center}
    \adjustimage{max size={0.9\linewidth}{0.9\paperheight}}{Python-TP2_files/Python-TP2_73_0.png}
    \end{center}
    { \hspace*{\fill} \\}
    

    % Add a bibliography block to the postdoc
    
    
    
    \end{document}
