
% Default to the notebook output style

    


% Inherit from the specified cell style.




    
\documentclass{article}

    
    
    \usepackage{graphicx} % Used to insert images
    \usepackage{adjustbox} % Used to constrain images to a maximum size 
    \usepackage{color} % Allow colors to be defined
    \usepackage{enumerate} % Needed for markdown enumerations to work
    \usepackage{geometry} % Used to adjust the document margins
    \usepackage{amsmath} % Equations
    \usepackage{amssymb} % Equations
    \usepackage[mathletters]{ucs} % Extended unicode (utf-8) support
    \usepackage[utf8x]{inputenc} % Allow utf-8 characters in the tex document
    \usepackage{fancyvrb} % verbatim replacement that allows latex
    \usepackage{grffile} % extends the file name processing of package graphics 
                         % to support a larger range 
    % The hyperref package gives us a pdf with properly built
    % internal navigation ('pdf bookmarks' for the table of contents,
    % internal cross-reference links, web links for URLs, etc.)
    \usepackage{hyperref}
    \usepackage{longtable} % longtable support required by pandoc >1.10
    \usepackage{booktabs}  % table support for pandoc > 1.12.2
    

    
    
    \definecolor{orange}{cmyk}{0,0.4,0.8,0.2}
    \definecolor{darkorange}{rgb}{.71,0.21,0.01}
    \definecolor{darkgreen}{rgb}{.12,.54,.11}
    \definecolor{myteal}{rgb}{.26, .44, .56}
    \definecolor{gray}{gray}{0.45}
    \definecolor{lightgray}{gray}{.95}
    \definecolor{mediumgray}{gray}{.8}
    \definecolor{inputbackground}{rgb}{.95, .95, .85}
    \definecolor{outputbackground}{rgb}{.95, .95, .95}
    \definecolor{traceback}{rgb}{1, .95, .95}
    % ansi colors
    \definecolor{red}{rgb}{.6,0,0}
    \definecolor{green}{rgb}{0,.65,0}
    \definecolor{brown}{rgb}{0.6,0.6,0}
    \definecolor{blue}{rgb}{0,.145,.698}
    \definecolor{purple}{rgb}{.698,.145,.698}
    \definecolor{cyan}{rgb}{0,.698,.698}
    \definecolor{lightgray}{gray}{0.5}
    
    % bright ansi colors
    \definecolor{darkgray}{gray}{0.25}
    \definecolor{lightred}{rgb}{1.0,0.39,0.28}
    \definecolor{lightgreen}{rgb}{0.48,0.99,0.0}
    \definecolor{lightblue}{rgb}{0.53,0.81,0.92}
    \definecolor{lightpurple}{rgb}{0.87,0.63,0.87}
    \definecolor{lightcyan}{rgb}{0.5,1.0,0.83}
    
    % commands and environments needed by pandoc snippets
    % extracted from the output of `pandoc -s`
    \DefineVerbatimEnvironment{Highlighting}{Verbatim}{commandchars=\\\{\}}
    % Add ',fontsize=\small' for more characters per line
    \newenvironment{Shaded}{}{}
    \newcommand{\KeywordTok}[1]{\textcolor[rgb]{0.00,0.44,0.13}{\textbf{{#1}}}}
    \newcommand{\DataTypeTok}[1]{\textcolor[rgb]{0.56,0.13,0.00}{{#1}}}
    \newcommand{\DecValTok}[1]{\textcolor[rgb]{0.25,0.63,0.44}{{#1}}}
    \newcommand{\BaseNTok}[1]{\textcolor[rgb]{0.25,0.63,0.44}{{#1}}}
    \newcommand{\FloatTok}[1]{\textcolor[rgb]{0.25,0.63,0.44}{{#1}}}
    \newcommand{\CharTok}[1]{\textcolor[rgb]{0.25,0.44,0.63}{{#1}}}
    \newcommand{\StringTok}[1]{\textcolor[rgb]{0.25,0.44,0.63}{{#1}}}
    \newcommand{\CommentTok}[1]{\textcolor[rgb]{0.38,0.63,0.69}{\textit{{#1}}}}
    \newcommand{\OtherTok}[1]{\textcolor[rgb]{0.00,0.44,0.13}{{#1}}}
    \newcommand{\AlertTok}[1]{\textcolor[rgb]{1.00,0.00,0.00}{\textbf{{#1}}}}
    \newcommand{\FunctionTok}[1]{\textcolor[rgb]{0.02,0.16,0.49}{{#1}}}
    \newcommand{\RegionMarkerTok}[1]{{#1}}
    \newcommand{\ErrorTok}[1]{\textcolor[rgb]{1.00,0.00,0.00}{\textbf{{#1}}}}
    \newcommand{\NormalTok}[1]{{#1}}
    
    % Define a nice break command that doesn't care if a line doesn't already
    % exist.
    \def\br{\hspace*{\fill} \\* }
    % Math Jax compatability definitions
    \def\gt{>}
    \def\lt{<}
    % Document parameters
    \title{Python-TP3-2014}
    
    
    

    % Pygments definitions
    
\makeatletter
\def\PY@reset{\let\PY@it=\relax \let\PY@bf=\relax%
    \let\PY@ul=\relax \let\PY@tc=\relax%
    \let\PY@bc=\relax \let\PY@ff=\relax}
\def\PY@tok#1{\csname PY@tok@#1\endcsname}
\def\PY@toks#1+{\ifx\relax#1\empty\else%
    \PY@tok{#1}\expandafter\PY@toks\fi}
\def\PY@do#1{\PY@bc{\PY@tc{\PY@ul{%
    \PY@it{\PY@bf{\PY@ff{#1}}}}}}}
\def\PY#1#2{\PY@reset\PY@toks#1+\relax+\PY@do{#2}}

\expandafter\def\csname PY@tok@gd\endcsname{\def\PY@tc##1{\textcolor[rgb]{0.63,0.00,0.00}{##1}}}
\expandafter\def\csname PY@tok@gu\endcsname{\let\PY@bf=\textbf\def\PY@tc##1{\textcolor[rgb]{0.50,0.00,0.50}{##1}}}
\expandafter\def\csname PY@tok@gt\endcsname{\def\PY@tc##1{\textcolor[rgb]{0.00,0.27,0.87}{##1}}}
\expandafter\def\csname PY@tok@gs\endcsname{\let\PY@bf=\textbf}
\expandafter\def\csname PY@tok@gr\endcsname{\def\PY@tc##1{\textcolor[rgb]{1.00,0.00,0.00}{##1}}}
\expandafter\def\csname PY@tok@cm\endcsname{\let\PY@it=\textit\def\PY@tc##1{\textcolor[rgb]{0.25,0.50,0.50}{##1}}}
\expandafter\def\csname PY@tok@vg\endcsname{\def\PY@tc##1{\textcolor[rgb]{0.10,0.09,0.49}{##1}}}
\expandafter\def\csname PY@tok@m\endcsname{\def\PY@tc##1{\textcolor[rgb]{0.40,0.40,0.40}{##1}}}
\expandafter\def\csname PY@tok@mh\endcsname{\def\PY@tc##1{\textcolor[rgb]{0.40,0.40,0.40}{##1}}}
\expandafter\def\csname PY@tok@go\endcsname{\def\PY@tc##1{\textcolor[rgb]{0.53,0.53,0.53}{##1}}}
\expandafter\def\csname PY@tok@ge\endcsname{\let\PY@it=\textit}
\expandafter\def\csname PY@tok@vc\endcsname{\def\PY@tc##1{\textcolor[rgb]{0.10,0.09,0.49}{##1}}}
\expandafter\def\csname PY@tok@il\endcsname{\def\PY@tc##1{\textcolor[rgb]{0.40,0.40,0.40}{##1}}}
\expandafter\def\csname PY@tok@cs\endcsname{\let\PY@it=\textit\def\PY@tc##1{\textcolor[rgb]{0.25,0.50,0.50}{##1}}}
\expandafter\def\csname PY@tok@cp\endcsname{\def\PY@tc##1{\textcolor[rgb]{0.74,0.48,0.00}{##1}}}
\expandafter\def\csname PY@tok@gi\endcsname{\def\PY@tc##1{\textcolor[rgb]{0.00,0.63,0.00}{##1}}}
\expandafter\def\csname PY@tok@gh\endcsname{\let\PY@bf=\textbf\def\PY@tc##1{\textcolor[rgb]{0.00,0.00,0.50}{##1}}}
\expandafter\def\csname PY@tok@ni\endcsname{\let\PY@bf=\textbf\def\PY@tc##1{\textcolor[rgb]{0.60,0.60,0.60}{##1}}}
\expandafter\def\csname PY@tok@nl\endcsname{\def\PY@tc##1{\textcolor[rgb]{0.63,0.63,0.00}{##1}}}
\expandafter\def\csname PY@tok@nn\endcsname{\let\PY@bf=\textbf\def\PY@tc##1{\textcolor[rgb]{0.00,0.00,1.00}{##1}}}
\expandafter\def\csname PY@tok@no\endcsname{\def\PY@tc##1{\textcolor[rgb]{0.53,0.00,0.00}{##1}}}
\expandafter\def\csname PY@tok@na\endcsname{\def\PY@tc##1{\textcolor[rgb]{0.49,0.56,0.16}{##1}}}
\expandafter\def\csname PY@tok@nb\endcsname{\def\PY@tc##1{\textcolor[rgb]{0.00,0.50,0.00}{##1}}}
\expandafter\def\csname PY@tok@nc\endcsname{\let\PY@bf=\textbf\def\PY@tc##1{\textcolor[rgb]{0.00,0.00,1.00}{##1}}}
\expandafter\def\csname PY@tok@nd\endcsname{\def\PY@tc##1{\textcolor[rgb]{0.67,0.13,1.00}{##1}}}
\expandafter\def\csname PY@tok@ne\endcsname{\let\PY@bf=\textbf\def\PY@tc##1{\textcolor[rgb]{0.82,0.25,0.23}{##1}}}
\expandafter\def\csname PY@tok@nf\endcsname{\def\PY@tc##1{\textcolor[rgb]{0.00,0.00,1.00}{##1}}}
\expandafter\def\csname PY@tok@si\endcsname{\let\PY@bf=\textbf\def\PY@tc##1{\textcolor[rgb]{0.73,0.40,0.53}{##1}}}
\expandafter\def\csname PY@tok@s2\endcsname{\def\PY@tc##1{\textcolor[rgb]{0.73,0.13,0.13}{##1}}}
\expandafter\def\csname PY@tok@vi\endcsname{\def\PY@tc##1{\textcolor[rgb]{0.10,0.09,0.49}{##1}}}
\expandafter\def\csname PY@tok@nt\endcsname{\let\PY@bf=\textbf\def\PY@tc##1{\textcolor[rgb]{0.00,0.50,0.00}{##1}}}
\expandafter\def\csname PY@tok@nv\endcsname{\def\PY@tc##1{\textcolor[rgb]{0.10,0.09,0.49}{##1}}}
\expandafter\def\csname PY@tok@s1\endcsname{\def\PY@tc##1{\textcolor[rgb]{0.73,0.13,0.13}{##1}}}
\expandafter\def\csname PY@tok@sh\endcsname{\def\PY@tc##1{\textcolor[rgb]{0.73,0.13,0.13}{##1}}}
\expandafter\def\csname PY@tok@sc\endcsname{\def\PY@tc##1{\textcolor[rgb]{0.73,0.13,0.13}{##1}}}
\expandafter\def\csname PY@tok@sx\endcsname{\def\PY@tc##1{\textcolor[rgb]{0.00,0.50,0.00}{##1}}}
\expandafter\def\csname PY@tok@bp\endcsname{\def\PY@tc##1{\textcolor[rgb]{0.00,0.50,0.00}{##1}}}
\expandafter\def\csname PY@tok@c1\endcsname{\let\PY@it=\textit\def\PY@tc##1{\textcolor[rgb]{0.25,0.50,0.50}{##1}}}
\expandafter\def\csname PY@tok@kc\endcsname{\let\PY@bf=\textbf\def\PY@tc##1{\textcolor[rgb]{0.00,0.50,0.00}{##1}}}
\expandafter\def\csname PY@tok@c\endcsname{\let\PY@it=\textit\def\PY@tc##1{\textcolor[rgb]{0.25,0.50,0.50}{##1}}}
\expandafter\def\csname PY@tok@mf\endcsname{\def\PY@tc##1{\textcolor[rgb]{0.40,0.40,0.40}{##1}}}
\expandafter\def\csname PY@tok@err\endcsname{\def\PY@bc##1{\setlength{\fboxsep}{0pt}\fcolorbox[rgb]{1.00,0.00,0.00}{1,1,1}{\strut ##1}}}
\expandafter\def\csname PY@tok@kd\endcsname{\let\PY@bf=\textbf\def\PY@tc##1{\textcolor[rgb]{0.00,0.50,0.00}{##1}}}
\expandafter\def\csname PY@tok@ss\endcsname{\def\PY@tc##1{\textcolor[rgb]{0.10,0.09,0.49}{##1}}}
\expandafter\def\csname PY@tok@sr\endcsname{\def\PY@tc##1{\textcolor[rgb]{0.73,0.40,0.53}{##1}}}
\expandafter\def\csname PY@tok@mo\endcsname{\def\PY@tc##1{\textcolor[rgb]{0.40,0.40,0.40}{##1}}}
\expandafter\def\csname PY@tok@kn\endcsname{\let\PY@bf=\textbf\def\PY@tc##1{\textcolor[rgb]{0.00,0.50,0.00}{##1}}}
\expandafter\def\csname PY@tok@mi\endcsname{\def\PY@tc##1{\textcolor[rgb]{0.40,0.40,0.40}{##1}}}
\expandafter\def\csname PY@tok@gp\endcsname{\let\PY@bf=\textbf\def\PY@tc##1{\textcolor[rgb]{0.00,0.00,0.50}{##1}}}
\expandafter\def\csname PY@tok@o\endcsname{\def\PY@tc##1{\textcolor[rgb]{0.40,0.40,0.40}{##1}}}
\expandafter\def\csname PY@tok@kr\endcsname{\let\PY@bf=\textbf\def\PY@tc##1{\textcolor[rgb]{0.00,0.50,0.00}{##1}}}
\expandafter\def\csname PY@tok@s\endcsname{\def\PY@tc##1{\textcolor[rgb]{0.73,0.13,0.13}{##1}}}
\expandafter\def\csname PY@tok@kp\endcsname{\def\PY@tc##1{\textcolor[rgb]{0.00,0.50,0.00}{##1}}}
\expandafter\def\csname PY@tok@w\endcsname{\def\PY@tc##1{\textcolor[rgb]{0.73,0.73,0.73}{##1}}}
\expandafter\def\csname PY@tok@kt\endcsname{\def\PY@tc##1{\textcolor[rgb]{0.69,0.00,0.25}{##1}}}
\expandafter\def\csname PY@tok@ow\endcsname{\let\PY@bf=\textbf\def\PY@tc##1{\textcolor[rgb]{0.67,0.13,1.00}{##1}}}
\expandafter\def\csname PY@tok@sb\endcsname{\def\PY@tc##1{\textcolor[rgb]{0.73,0.13,0.13}{##1}}}
\expandafter\def\csname PY@tok@k\endcsname{\let\PY@bf=\textbf\def\PY@tc##1{\textcolor[rgb]{0.00,0.50,0.00}{##1}}}
\expandafter\def\csname PY@tok@se\endcsname{\let\PY@bf=\textbf\def\PY@tc##1{\textcolor[rgb]{0.73,0.40,0.13}{##1}}}
\expandafter\def\csname PY@tok@sd\endcsname{\let\PY@it=\textit\def\PY@tc##1{\textcolor[rgb]{0.73,0.13,0.13}{##1}}}

\def\PYZbs{\char`\\}
\def\PYZus{\char`\_}
\def\PYZob{\char`\{}
\def\PYZcb{\char`\}}
\def\PYZca{\char`\^}
\def\PYZam{\char`\&}
\def\PYZlt{\char`\<}
\def\PYZgt{\char`\>}
\def\PYZsh{\char`\#}
\def\PYZpc{\char`\%}
\def\PYZdl{\char`\$}
\def\PYZhy{\char`\-}
\def\PYZsq{\char`\'}
\def\PYZdq{\char`\"}
\def\PYZti{\char`\~}
% for compatibility with earlier versions
\def\PYZat{@}
\def\PYZlb{[}
\def\PYZrb{]}
\makeatother


    % Exact colors from NB
    \definecolor{incolor}{rgb}{0.0, 0.0, 0.5}
    \definecolor{outcolor}{rgb}{0.545, 0.0, 0.0}



    
    % Prevent overflowing lines due to hard-to-break entities
    \sloppy 
    % Setup hyperref package
    \hypersetup{
      breaklinks=true,  % so long urls are correctly broken across lines
      colorlinks=true,
      urlcolor=blue,
      linkcolor=darkorange,
      citecolor=darkgreen,
      }
    % Slightly bigger margins than the latex defaults
    
    \geometry{verbose,tmargin=1in,bmargin=1in,lmargin=1in,rmargin=1in}
    
    

    \begin{document}
    
    
    \maketitle
    
    

    

    \subparagraph{TP 3 -- Informatique CM3 -- Octobre 2014}



    \section{Python @ Polytech'Lille}


    Le texte de cette sessions de travaux pratiques est également disponible
ici
http://nbviewer.ipython.org/github/ecalzavarini/python-at-polytech-lille/blob/master/Python-TP3-2014.ipynb

    Nous vous rappelons que les modalités pour accomplir ce TP :

Vous aurez à écrire plusieurs scripts (script1.py , script2.py ,
\ldots{}). Les scripts doivent être accompagnés par un document
descriptif unique (README.txt). Dans ce fichier, vous devrez décrire le
mode de fonctionnement des scripts et, si besoin, mettre vos
commentaires. Merci d'y écrire aussi vos noms et prenoms complets. Tous
les fichiers doivent etre mis dans un dossier appelé TP1-nom1-nom2 et
ensuite être compressés dans un fichier archive TP1-nom1-nom2.tgz .
Enfin vous allez envoyer ce fichier par email à l'enseignant: soit
Enrico (enrico.calzavarini@polytech-lille.fr) ou Stefano
(stefano.berti@polytech-lille.fr).

Vous avez une semaine de temps pour compléter le TP, c'est-à-dire que la
date limite pour envoyer vos travaux c'est le jour avant le TP suivant.


    \subparagraph{Tracer le graphique d'un champ vectoriel en deux dimensions}


    \begin{Verbatim}[commandchars=\\\{\}]
{\color{incolor}In [{\color{incolor}1}]:} 
        
        \PY{k+kn}{import} \PY{n+nn}{matplotlib.pylab} \PY{k+kn}{as} \PY{n+nn}{plt}
        \PY{k+kn}{import} \PY{n+nn}{numpy} \PY{k+kn}{as} \PY{n+nn}{np}
        \PY{k+kn}{from} \PY{n+nn}{math} \PY{k+kn}{import} \PY{o}{*}
        
        \PY{n}{x} \PY{o}{=} \PY{n}{np}\PY{o}{.}\PY{n}{arange}\PY{p}{(}\PY{l+m+mi}{0}\PY{p}{,}\PY{l+m+mi}{2}\PY{o}{*}\PY{n}{pi}\PY{p}{,}\PY{l+m+mf}{0.2}\PY{p}{)}
        \PY{n}{y} \PY{o}{=} \PY{n}{np}\PY{o}{.}\PY{n}{arange}\PY{p}{(}\PY{l+m+mi}{0}\PY{p}{,}\PY{l+m+mi}{2}\PY{o}{*}\PY{n}{pi}\PY{p}{,}\PY{l+m+mf}{0.2}\PY{p}{)}
        
        \PY{c}{\PYZsh{} np.meshgrid, produit une matrices de coordonnées à partir de deux vecteurs. }
        \PY{n}{X}\PY{p}{,}\PY{n}{Y} \PY{o}{=} \PY{n}{np}\PY{o}{.}\PY{n}{meshgrid}\PY{p}{(}\PY{n}{x}\PY{p}{,}\PY{n}{y}\PY{p}{)} 
        
        \PY{c}{\PYZsh{}définition du champ vectoriel}
        \PY{n}{U} \PY{o}{=} \PY{n}{np}\PY{o}{.}\PY{n}{sin}\PY{p}{(}\PY{n}{Y}\PY{p}{)}\PY{o}{*}\PY{n}{np}\PY{o}{.}\PY{n}{cos}\PY{p}{(}\PY{n}{Y}\PY{p}{)}
        \PY{n}{V} \PY{o}{=} \PY{o}{\PYZhy{}}\PY{n}{np}\PY{o}{.}\PY{n}{cos}\PY{p}{(}\PY{n}{X}\PY{p}{)}\PY{o}{*}\PY{n}{np}\PY{o}{.}\PY{n}{sin}\PY{p}{(}\PY{n}{X}\PY{p}{)}
        
        \PY{n}{C} \PY{o}{=}  \PY{p}{(} \PY{n}{U}\PY{o}{*}\PY{o}{*}\PY{l+m+mf}{2.} \PY{o}{+} \PY{n}{V}\PY{o}{*}\PY{o}{*}\PY{l+m+mf}{2.} \PY{p}{)}\PY{o}{*}\PY{o}{*}\PY{o}{.}\PY{l+m+mi}{5} \PY{c}{\PYZsh{} calcul du module du vecteur}
        
        \PY{n}{plt}\PY{o}{.}\PY{n}{figure}\PY{p}{(} \PY{n}{figsize}\PY{o}{=}\PY{p}{(}\PY{l+m+mi}{10}\PY{p}{,}\PY{l+m+mi}{10}\PY{p}{)} \PY{p}{)}  \PY{c}{\PYZsh{} taille de la figure : 10 x 10 pouces}
        
        \PY{n}{plt}\PY{o}{.}\PY{n}{quiver}\PY{p}{(} \PY{n}{X}\PY{p}{,} \PY{n}{Y}\PY{p}{,} \PY{n}{U}\PY{p}{,} \PY{n}{V} \PY{p}{,} \PY{n}{C} \PY{p}{,} \PY{n}{scale} \PY{o}{=} \PY{l+m+mf}{20.}\PY{p}{)} \PY{c}{\PYZsh{}trace le graphyque du champ vectoriel}
        
        \PY{n}{plt}\PY{o}{.}\PY{n}{axis}\PY{p}{(}\PY{p}{[}\PY{l+m+mi}{0}\PY{p}{,} \PY{l+m+mi}{2}\PY{o}{*}\PY{n}{pi}\PY{p}{,} \PY{l+m+mi}{0}\PY{p}{,} \PY{l+m+mi}{2}\PY{o}{*}\PY{n}{pi}\PY{p}{]}\PY{p}{)}
        \PY{n}{plt}\PY{o}{.}\PY{n}{axes}\PY{p}{(}\PY{p}{)}\PY{o}{.}\PY{n}{set\PYZus{}aspect}\PY{p}{(}\PY{l+m+mi}{1}\PY{p}{)}
        
        \PY{n}{plt}\PY{o}{.}\PY{n}{show}\PY{p}{(}\PY{p}{)}
\end{Verbatim}

    \begin{center}
    \adjustimage{max size={0.8\linewidth}{0.8\paperheight}}{Python-TP3-2014_files/Python-TP3-2014_5_0.png}
    \end{center}
    { \hspace*{\fill} \\}
    
    À noter que dans ``plt.quiver( X, Y, U, V , C , scale = 20.)'' le
paramètre ``scale'' regle la taille d'affichage des flèches selon le
souhaite de l'utilisateur. Une valeure de ``scale'' plus petit produit
des flèches plus larges.


    \subsubsection{Script 1 : Comprendre la déformation de type cisaillement}


    Écrire un script qui effectue les calculs et actions suivantes:

\begin{enumerate}
\def\labelenumi{\alph{enumi})}
\itemsep1pt\parskip0pt\parsep0pt
\item
  trace le graphique d'un champ de déformation de type cisaillement,
  c'est-à-dire
\end{enumerate}

\[ {\bf u} = (u_x, u_y) = \left( \begin{array}{cc}
\partial_x u_x  & \partial_y u_x \\
\partial_x u_y  & \partial_y u_y \end{array} \right) 
\left( \begin{array}{c}
x   \\
y  \end{array} \right)=  
\left( \begin{array}{c}
\gamma y   \\
0  \end{array} \right)\]

ou \(\gamma = \partial_y u_x\) est le taux de déformation, ici supposée
constant.

\begin{enumerate}
\def\labelenumi{\alph{enumi})}
\setcounter{enumi}{1}
\item
  calcule la matrice gradient de déformation:
  \[ \nabla u =  \left( \begin{array}{cc}
  \partial_x u_x & \partial_y u_x  \\
  \partial_x u_y & \partial_y u_y \end{array} \right) \]
\item
  décompose cette matrice dans sa partie symétrique (\(S\)) et dans sa
  partie antisymétrique (\(A\)) :
\end{enumerate}

\[ S = \frac{1}{2}\left(\nabla u  + (\nabla u )^T \right) \]

\[ A = \frac{1}{2}\left(\nabla u  - (\nabla u )^T \right) \]

\begin{enumerate}
\def\labelenumi{\alph{enumi})}
\setcounter{enumi}{3}
\item
  calcule le champ de déformation correspondant à \(S\) et \(A\),
  appellés respectivement \(u_S\) et \(u_A\)
\item
  trace les graphiques de deux nouveaux champs de déformation, \(u_S\)
  et \(u_A\).
\item
  enfin, trace le graphique de la somme de \(u_S\) et \(u_A\).
\end{enumerate}

Qu'apprenons-nous des ces quatre graphiques ?

Dans les cours de mécanique des milieux continus, on entend souvent dire
qu'une déformation de type cisaillement est équivalente à la somme d'une
rotation (champ \(A\)) et d'un étirement, elongation et compression,
(champ \(S\)). Avec ce script, vous montrerez d'une façon graphique que
cette affirmation est bien vérifiée.


    \subparagraph{Tranpsosition d'une matrice avec les arrays de \(numpy\)}


    \begin{Verbatim}[commandchars=\\\{\}]
{\color{incolor}In [{\color{incolor}2}]:} \PY{k+kn}{import} \PY{n+nn}{numpy} \PY{k+kn}{as} \PY{n+nn}{np}
        
        \PY{n}{a} \PY{o}{=} \PY{n}{np}\PY{o}{.}\PY{n}{array}\PY{p}{(}\PY{p}{[}\PY{p}{[}\PY{l+m+mi}{1}\PY{p}{,}\PY{l+m+mi}{2}\PY{p}{]}\PY{p}{,}\PY{p}{[}\PY{l+m+mi}{3}\PY{p}{,}\PY{l+m+mi}{4}\PY{p}{]}\PY{p}{]}\PY{p}{)} \PY{c}{\PYZsh{} a: matrice 2 x 2}
        \PY{k}{print}\PY{p}{(}\PY{l+s}{\PYZdq{}}\PY{l+s}{a = }\PY{l+s}{\PYZdq{}}\PY{p}{)}
        \PY{k}{print}\PY{p}{(}\PY{n}{a}\PY{p}{)}
        
        \PY{n}{at}\PY{o}{=}\PY{n}{np}\PY{o}{.}\PY{n}{transpose}\PY{p}{(}\PY{n}{a}\PY{p}{)} \PY{c}{\PYZsh{} transposition de la matrice a}
        \PY{k}{print}\PY{p}{(}\PY{l+s}{\PYZdq{}}\PY{l+s}{at = }\PY{l+s}{\PYZdq{}}\PY{p}{)}
        \PY{k}{print}\PY{p}{(}\PY{n}{at}\PY{p}{)}
\end{Verbatim}

    \begin{Verbatim}[commandchars=\\\{\}]
a = 
[[1 2]
 [3 4]]
at = 
[[1 3]
 [2 4]]
    \end{Verbatim}


    \subsubsection{Script 2 : Transport du vecteur vitesse}


    On veut écrire un script qui met en oeuvre une relation notable de la
cinématique des solides rigides: la formule du transport du vecteur
vitesse entre deux points \(A\) et \(B\) appartenants à un même solide
\(S_1\) (en mouvement par rapport à un autre solide \(S_0\)):

    \[\vec{\bf V}(B \in S_1/S_0) = \vec{\bf V}(A \in S_1/S_0) + \vec{\bf \Omega}_{S_1/S_0} \wedge \vec{\bf AB}\]

    Plus précisément, le script devra demander à l'utilisateur les
coordonnées des points \(A\) et \$ B\$, la vitesse en \(A\) ainsi que le
vecteur rotation \(\vec{\bf \Omega}_{S_1/S_0}\) du solide \(S_1\) par
rapport au solide \(S_0\). Il calculera le vectur vitesse en \(B\) et sa
norme \(|\vec{\bf V}(B \in S_1/S_0)|\).

    En utilisant des operations simples sur les elements des arrays
(vecteurs ou matrices), ainsi que des boucles \(for\), écrivez par
vous-mêmes les deux fonctions qui effectuent le produit scalaire et le
produit vectoriel de deux vecteurs, même si, comme vous pouvez
l'imaginer, \(numpy\) possède déjà des fonctions pour cela faire, qui
sont très pratiques à utiliser:

    \begin{Verbatim}[commandchars=\\\{\}]
{\color{incolor}In [{\color{incolor}3}]:} \PY{k+kn}{import} \PY{n+nn}{numpy} \PY{k+kn}{as} \PY{n+nn}{np}
        
        \PY{n}{a} \PY{o}{=} \PY{n}{np}\PY{o}{.}\PY{n}{array}\PY{p}{(}\PY{p}{[}\PY{l+m+mi}{1}\PY{p}{,}\PY{l+m+mi}{2}\PY{p}{,}\PY{l+m+mi}{3}\PY{p}{]}\PY{p}{)}
        \PY{n}{b} \PY{o}{=} \PY{n}{np}\PY{o}{.}\PY{n}{array}\PY{p}{(}\PY{p}{[}\PY{l+m+mi}{0}\PY{p}{,}\PY{l+m+mi}{1}\PY{p}{,}\PY{l+m+mi}{5}\PY{p}{]}\PY{p}{)}
        
        \PY{n}{np}\PY{o}{.}\PY{n}{dot}\PY{p}{(}\PY{n}{a}\PY{p}{,}\PY{n}{b}\PY{p}{)} \PY{c}{\PYZsh{} produit scalaire}
\end{Verbatim}

            \begin{Verbatim}[commandchars=\\\{\}]
{\color{outcolor}Out[{\color{outcolor}3}]:} 17
\end{Verbatim}
        
    \begin{Verbatim}[commandchars=\\\{\}]
{\color{incolor}In [{\color{incolor}4}]:} \PY{n}{np}\PY{o}{.}\PY{n}{cross}\PY{p}{(}\PY{n}{a}\PY{p}{,}\PY{n}{b}\PY{p}{)} \PY{c}{\PYZsh{} produit vectoriel}
\end{Verbatim}

            \begin{Verbatim}[commandchars=\\\{\}]
{\color{outcolor}Out[{\color{outcolor}4}]:} array([ 7, -5,  1])
\end{Verbatim}
        
    Vérifiez que votre script est correct en utilisant les fonctions
ci-dessus.

    Vue la nécessité de rentrer certaines données (les coordonnées des
points et les composantes de vecteurs) plusieurs fois, il pourrait être
pratique de définir aussi une fonction pour la lecture de ce type de
données.


    \subsubsection{Script 3 : Propagation d'une perturbation}


    L'équation de convection linéaire est une modèle mathématique simple
décrivant la propagation d'un champ scalaire (tel qu'une perturbation,
une vague, une vibration, ou bien un champ de concentration d'une
substance chimique) dans un milieu à dissipation négligeable. Elle est
utilisée dans une large gamme de domaines qui vont de la météorologie à
l'océanographie, à la physique et l'ingénierie. Cette équation s'écrit:

\[ \frac{\partial u}{\partial t} + c \frac{ \partial u}{\partial x} = 0\]

Si la condition initiale (entendue comme la forme de la perturbation
initiale) est \(u(x, t=0) = u_0 (x)\) , on peut montrer que la solution
exacte de l'équation est \(u (x, t) = u_0 (x-ct)\), c'est-à-dire que la
perturbation se propage sans changement de forme avec une vitesse \(c\).

Pour la résoudre numériquement, nous pouvons discrétiser cette équation
dans l'espace et dans le temps en utilisant le schéma de différences
finies en avant pour la dérivée temporelle et le schéma de différences
finies en arrière pour le dérivée par rapport à l'espace. Pensez à la
discrétisation des coordonnées spatiales \(x\) en points désignés par un
indice \(i\) qui varie de \(0\) à \(N\), et la disctérisation temporelle
par intervalles de temps discrets de taille \(\Delta t\).

A partir de la définition de dérivée (en supprimant tout simplement la
limite), nous savons que:

\[\frac{\partial u}{\partial x} \simeq \frac{u(x+\Delta x) - u(x)}{\Delta x} \]

Notre équation discrète, alors, est:

\[\frac{u_i^{n+1} - u_i^{n}}{\Delta t} + c \frac{u_i^{n+1} - u_{i-1}^{n}}{\Delta x} = 0\]

Où \(n\) et \(n + 1\) sont deux étapes consécutives dans le temps,
tandis que \(i-1\) et \(i\) sont deux points voisins de la coordonnée
discrétisée \(x\). Si on donne les conditions initiales, la seule
inconnue dans cette discrétisation est \(u_i^{n+1}\). Nous pouvons
résoudre pour notre inconnue pour obtenir une équation qui nous permet
d'avancer dans le temps, comme suit:

\[u_i^{n+1} = u_i^{n} - c \frac{\Delta t}{\Delta x}(u_i^{n} - u_i^{n-1})\]


    \paragraph{Écrire un script qui effectue les calculs et les actions suivantes :}


    \begin{enumerate}
\def\labelenumi{\arabic{enumi})}
\item
  Il définit une grille de \(N=81\) points régulièrement espacés dans un
  domaine spatial qui fait \(2\) metres de longueur, c'est-à-dire
  \(x \in [0,2\ m]\). Il définit ainsi la variable \(\Delta x\) comme la
  distance entre deux points adjacents de la grille.
\item
  Il met en place les conditions initiales suivantes. La condition
  initiale \(u_0\) a la valeur \(u = 2\) (en unités arbitraires, dans la
  suite \(u.a.\)) dans l'intervalle \(0.5 \ m ≤ x ≤ 1 \ m\) et
  \(u = 1 \ (u.a.)\) partout ailleurs dans \([0,2\ m]\) (c'est-à-dire
  une fonction chapeau).
\item
  Il trace un graphique de la condition initiale.
\item
  Il définit une fonction pour calculer \(u\) au pas de temps futur
  (\(n+1\)) à partir de la valeur de \(u\) au pas de temps présent
  (\(n\)). La valeur affectée à la variable pas de temps sera $\Delta t
  = 0.25~s $, et on posera la valueur $c = 1 ~m /s $ pour la vitesse
  de la perturbation dans le milieu.
\item
  Il trace sur le même graphique que celui utilisé pour la condition
  initiale (mais avec une autre couleur) la fonction \(u\) au temps
  \(t=6.25\ s\).
\end{enumerate}


    \subparagraph{Quelques operations utiles sur les arrays de \(numpy\)}


    Le fonctionnement de la fonction \(ones()\) est analogue à celui de la
fonction \(zeros()\) qu'on a déjà rencontré, voir l'exemple suivant:

    \begin{Verbatim}[commandchars=\\\{\}]
{\color{incolor}In [{\color{incolor}5}]:} \PY{k+kn}{import} \PY{n+nn}{numpy} \PY{k+kn}{as} \PY{n+nn}{np}
        
        \PY{n}{np}\PY{o}{.}\PY{n}{ones}\PY{p}{(}\PY{l+m+mi}{10}\PY{p}{)}      \PY{c}{\PYZsh{} fonction ones() pour un array de 10 éléments}
\end{Verbatim}

            \begin{Verbatim}[commandchars=\\\{\}]
{\color{outcolor}Out[{\color{outcolor}5}]:} array([ 1.,  1.,  1.,  1.,  1.,  1.,  1.,  1.,  1.,  1.])
\end{Verbatim}
        
    Il est possible de faire des cycles implicites sur les arrays de
\(numpy\), comme dans l'exemple suivant. Cette opération permet de
manipuler toute une partie d'un array en même temps, sans devoir
recourir à une boucle explicite (de type \(for\) ou \(while\)).

    \begin{Verbatim}[commandchars=\\\{\}]
{\color{incolor}In [{\color{incolor}6}]:} \PY{k+kn}{import} \PY{n+nn}{numpy} \PY{k+kn}{as} \PY{n+nn}{np}
        
        \PY{n}{a} \PY{o}{=} \PY{n}{np}\PY{o}{.}\PY{n}{array}\PY{p}{(}\PY{p}{[}\PY{l+m+mi}{1}\PY{p}{,}\PY{l+m+mi}{10}\PY{p}{]}\PY{p}{)}
        \PY{n}{a} \PY{o}{=} \PY{p}{[}\PY{l+m+mi}{1}\PY{p}{,}\PY{l+m+mi}{2}\PY{p}{,}\PY{l+m+mi}{3}\PY{p}{,}\PY{l+m+mi}{4}\PY{p}{,}\PY{l+m+mi}{5}\PY{p}{,}\PY{l+m+mi}{6}\PY{p}{,}\PY{l+m+mi}{7}\PY{p}{,}\PY{l+m+mi}{8}\PY{p}{,}\PY{l+m+mi}{9}\PY{p}{,}\PY{l+m+mi}{10}\PY{p}{]}
        \PY{n}{b} \PY{o}{=} \PY{n}{a}\PY{p}{[}\PY{l+m+mi}{2}\PY{p}{:}\PY{l+m+mi}{7}\PY{p}{]}
        \PY{k}{print}\PY{p}{(}\PY{l+s}{\PYZdq{}}\PY{l+s}{a = }\PY{l+s}{\PYZdq{}}\PY{p}{)}
        \PY{k}{print}\PY{p}{(}\PY{n}{a}\PY{p}{)}
        \PY{k}{print}\PY{p}{(}\PY{l+s}{\PYZdq{}}\PY{l+s}{b = }\PY{l+s}{\PYZdq{}}\PY{p}{)}
        \PY{k}{print}\PY{p}{(}\PY{n}{b}\PY{p}{)}
\end{Verbatim}

    \begin{Verbatim}[commandchars=\\\{\}]
a = 
[1, 2, 3, 4, 5, 6, 7, 8, 9, 10]
b = 
[3, 4, 5, 6, 7]
    \end{Verbatim}

    Pour copier les éléments d'un arrays dans un autre array, il est
possible de travailler comme dans les exemples suivants.

    \begin{Verbatim}[commandchars=\\\{\}]
{\color{incolor}In [{\color{incolor}7}]:} \PY{k+kn}{import} \PY{n+nn}{numpy} \PY{k+kn}{as} \PY{n+nn}{np}
        
        \PY{n}{a}\PY{o}{=}\PY{n}{np}\PY{o}{.}\PY{n}{array}\PY{p}{(}\PY{p}{[}\PY{l+m+mi}{1}\PY{p}{,} \PY{l+m+mi}{2}\PY{p}{,} \PY{l+m+mi}{3}\PY{p}{,} \PY{l+m+mi}{4}\PY{p}{,} \PY{l+m+mi}{5}\PY{p}{]}\PY{p}{)}
        \PY{k}{print}\PY{p}{(}\PY{l+s}{\PYZdq{}}\PY{l+s}{a = }\PY{l+s}{\PYZdq{}}\PY{p}{)}
        \PY{k}{print}\PY{p}{(}\PY{n}{a}\PY{p}{)}
        
        \PY{n}{b}\PY{o}{=}\PY{n}{a}\PY{o}{.}\PY{n}{copy}\PY{p}{(}\PY{p}{)}      \PY{c}{\PYZsh{} première méthode}
        \PY{k}{print}\PY{p}{(}\PY{l+s}{\PYZdq{}}\PY{l+s}{b = }\PY{l+s}{\PYZdq{}}\PY{p}{)}
        \PY{k}{print}\PY{p}{(}\PY{n}{b}\PY{p}{)}
        
        \PY{n}{c}\PY{o}{=}\PY{n}{np}\PY{o}{.}\PY{n}{copy}\PY{p}{(}\PY{n}{a}\PY{p}{)}   \PY{c}{\PYZsh{} seconde méthode}
        \PY{k}{print}\PY{p}{(}\PY{l+s}{\PYZdq{}}\PY{l+s}{c = }\PY{l+s}{\PYZdq{}}\PY{p}{)}
        \PY{k}{print}\PY{p}{(}\PY{n}{c}\PY{p}{)}
\end{Verbatim}

    \begin{Verbatim}[commandchars=\\\{\}]
a = 
[1 2 3 4 5]
b = 
[1 2 3 4 5]
c = 
[1 2 3 4 5]
    \end{Verbatim}


    \subsubsection{Script 4 (Bonus): Moments d'inertie d'une plaque d'aluminium}


    La matrice d'inertie d'un solide \(S\), au point \(O\) dans la base de
coordonnées \((\vec{x},\vec{y},\vec{z})\), s'écrit :

    \[[I_0(S)]  = \left( \begin{array}{ccc}  
I_{Ox} & -I_{Oxy} & -I_{Ozx} \\
  -I_{Oxy}     &  I_{Oy} & -I_{Oyz} \\
 -I_{Ozx}      & -I_{Oyz} & I_{Oz}  \\
\end{array} \right)_{(\vec{x},\vec{y},\vec{z})} \]

    avec \(I_{Ox}\) , \(I_{Oy}\) et \(I_{Oz}\), moments d'inertie par
rapport aux axes du repère \((\vec{x},\vec{y},\vec{z})\):

    \[I_{Ox} = \int_{P\in S} (y^2+z^2)\ dm\]

\[I_{Oy} = \int_{P\in S} (z^2+x^2)\ dm\]

\[I_{Oz} = \int_{P\in S} (x^2+y^2)\ dm\]

    et \(I_{Oxy}\) , \(I_{Oyz}\) et \(I_{Oyz}\) les produits d'inertie par
rapport aux axes du même repère :

    \[I_{Oyz} = \int_{P\in S} yz \ dm\]

\[I_{Ozx} = \int_{P\in S} zx \ dm\]

\[I_{Oxy} = \int_{P\in S} xy \ dm\]

    Le symbole \(dm\) dans les intégrales ci-dessus représente un élément
infinentesimal de masse du solide et le domaine de l'intégration est
pris sur tous les points \(P\) du solide \(S\).

    Dans ce script, nous souhaitons calculer la matrice d'inertie
\(I_{O}(S)\) pour une plaque rectangulaire homogène en aluminium.

    
    \begin{center}
    \adjustimage{max size={1.0\linewidth}{1.0\paperheight}}{Python-TP3-2014_files/Python-TP3-2014_40_0.jpeg}
    \end{center}
    { \hspace*{\fill} \\}
    

    La plaque, voir aussi la figure ci-dessus, mesure \(5\ m\) de longueur
(direction y), \(2\ m\) de largeur (direction x) et juste un \(1 \ cm\)
d'épaisseur (direction z). La densité massique de l'aluminium est
\(\rho_{Al} = 2.7 \ g / cm^{3}\). A partir de ces données on peut
calculer la densité massique surfacique de la plaque ainsi que sa masse
totale.

Le script doit être composé par des fonctions, qui effectuent le calcul
des intégrales de façon numérique (c'est-à-dire en effectuant des
sommations). La valeur de la précision (\(\Delta\), mesurée en mètres),
indiquant le niveau de discrétisation spatiale dans les intégrales,
devra être introduite par l'utilisateur au début du script.

Suggestions : 1) étant donné le faible épaisseur de la plaque on peut
négliger la coordonnée \(z\) dans le calcul des integrales. 2) Pour une
convergence rapide des calculs, choisir \(\Delta > 1\ mm\).


    \subparagraph{Multiplication élément par élément des tableaux numpy}


    Il y a deux manières de multiplier élément par élément les arraya
\(numpy\): soit avec le simple operateur \(*\) soit avec la fonction
\(np.multiply\). Cette deuxième possibilté offre plus de flexibilité.
Regardez les exemples ci-dessous:

    \begin{Verbatim}[commandchars=\\\{\}]
{\color{incolor}In [{\color{incolor}9}]:} \PY{n}{x}\PY{o}{=}\PY{n}{np}\PY{o}{.}\PY{n}{linspace}\PY{p}{(}\PY{l+m+mi}{0}\PY{p}{,}\PY{l+m+mi}{1}\PY{p}{,}\PY{l+m+mi}{11}\PY{p}{)}
        \PY{k}{print}\PY{p}{(}\PY{n}{x}\PY{p}{)}
        
        \PY{n}{y} \PY{o}{=} \PY{n}{x}\PY{o}{*}\PY{n}{x}   \PY{c}{\PYZsh{} en utilisant le symbole de multiplication}
        \PY{k}{print}\PY{p}{(}\PY{n}{y}\PY{p}{)}
        
        \PY{n}{z} \PY{o}{=} \PY{n}{np}\PY{o}{.}\PY{n}{multiply}\PY{p}{(}\PY{n}{x}\PY{p}{,}\PY{n}{x}\PY{p}{)}\PY{p}{[}\PY{p}{:}\PY{p}{]}   \PY{c}{\PYZsh{} en utilisant la fonction de multiplication }
        \PY{k}{print}\PY{p}{(}\PY{n}{z}\PY{p}{)}
        
        \PY{n}{w} \PY{o}{=} \PY{n}{np}\PY{o}{.}\PY{n}{multiply}\PY{p}{(}\PY{n}{x}\PY{p}{,}\PY{n}{x}\PY{p}{)}\PY{p}{[}\PY{l+m+mi}{3}\PY{p}{:}\PY{l+m+mi}{6}\PY{p}{]}  \PY{c}{\PYZsh{} choisir uniquement les éléments entre 3 et 5 }
        \PY{k}{print}\PY{p}{(}\PY{n}{w}\PY{p}{)}
        
        \PY{n}{h} \PY{o}{=} \PY{n}{np}\PY{o}{.}\PY{n}{multiply}\PY{p}{(}\PY{n}{x}\PY{p}{,}\PY{n}{x}\PY{p}{)}\PY{p}{[}\PY{l+m+mi}{1}\PY{p}{:}\PY{l+m+mi}{10}\PY{p}{:}\PY{l+m+mi}{2}\PY{p}{]} \PY{c}{\PYZsh{} choisir uniquement les éléments paires du tableau  }
        \PY{k}{print}\PY{p}{(}\PY{n}{h}\PY{p}{)}
\end{Verbatim}

    \begin{Verbatim}[commandchars=\\\{\}]
[ 0.   0.1  0.2  0.3  0.4  0.5  0.6  0.7  0.8  0.9  1. ]
[ 0.    0.01  0.04  0.09  0.16  0.25  0.36  0.49  0.64  0.81  1.  ]
[ 0.    0.01  0.04  0.09  0.16  0.25  0.36  0.49  0.64  0.81  1.  ]
[ 0.09  0.16  0.25]
[ 0.01  0.09  0.25  0.49  0.81]
    \end{Verbatim}
        

    % Add a bibliography block to the postdoc
    
    
    
    \end{document}
