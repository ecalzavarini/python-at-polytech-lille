
% Default to the notebook output style

    


% Inherit from the specified cell style.




    
\documentclass{article}

    
    
    \usepackage{graphicx} % Used to insert images
    \usepackage{adjustbox} % Used to constrain images to a maximum size 
    \usepackage{color} % Allow colors to be defined
    \usepackage{enumerate} % Needed for markdown enumerations to work
    \usepackage{geometry} % Used to adjust the document margins
    \usepackage{amsmath} % Equations
    \usepackage{amssymb} % Equations
    \usepackage[mathletters]{ucs} % Extended unicode (utf-8) support
    \usepackage[utf8x]{inputenc} % Allow utf-8 characters in the tex document
    \usepackage{fancyvrb} % verbatim replacement that allows latex
    \usepackage{grffile} % extends the file name processing of package graphics 
                         % to support a larger range 
    % The hyperref package gives us a pdf with properly built
    % internal navigation ('pdf bookmarks' for the table of contents,
    % internal cross-reference links, web links for URLs, etc.)
    \usepackage{hyperref}
    \usepackage{longtable} % longtable support required by pandoc >1.10
    \usepackage{booktabs}  % table support for pandoc > 1.12.2
    

    
    
    \definecolor{orange}{cmyk}{0,0.4,0.8,0.2}
    \definecolor{darkorange}{rgb}{.71,0.21,0.01}
    \definecolor{darkgreen}{rgb}{.12,.54,.11}
    \definecolor{myteal}{rgb}{.26, .44, .56}
    \definecolor{gray}{gray}{0.45}
    \definecolor{lightgray}{gray}{.95}
    \definecolor{mediumgray}{gray}{.8}
    \definecolor{inputbackground}{rgb}{.95, .95, .85}
    \definecolor{outputbackground}{rgb}{.95, .95, .95}
    \definecolor{traceback}{rgb}{1, .95, .95}
    % ansi colors
    \definecolor{red}{rgb}{.6,0,0}
    \definecolor{green}{rgb}{0,.65,0}
    \definecolor{brown}{rgb}{0.6,0.6,0}
    \definecolor{blue}{rgb}{0,.145,.698}
    \definecolor{purple}{rgb}{.698,.145,.698}
    \definecolor{cyan}{rgb}{0,.698,.698}
    \definecolor{lightgray}{gray}{0.5}
    
    % bright ansi colors
    \definecolor{darkgray}{gray}{0.25}
    \definecolor{lightred}{rgb}{1.0,0.39,0.28}
    \definecolor{lightgreen}{rgb}{0.48,0.99,0.0}
    \definecolor{lightblue}{rgb}{0.53,0.81,0.92}
    \definecolor{lightpurple}{rgb}{0.87,0.63,0.87}
    \definecolor{lightcyan}{rgb}{0.5,1.0,0.83}
    
    % commands and environments needed by pandoc snippets
    % extracted from the output of `pandoc -s`
    \DefineVerbatimEnvironment{Highlighting}{Verbatim}{commandchars=\\\{\}}
    % Add ',fontsize=\small' for more characters per line
    \newenvironment{Shaded}{}{}
    \newcommand{\KeywordTok}[1]{\textcolor[rgb]{0.00,0.44,0.13}{\textbf{{#1}}}}
    \newcommand{\DataTypeTok}[1]{\textcolor[rgb]{0.56,0.13,0.00}{{#1}}}
    \newcommand{\DecValTok}[1]{\textcolor[rgb]{0.25,0.63,0.44}{{#1}}}
    \newcommand{\BaseNTok}[1]{\textcolor[rgb]{0.25,0.63,0.44}{{#1}}}
    \newcommand{\FloatTok}[1]{\textcolor[rgb]{0.25,0.63,0.44}{{#1}}}
    \newcommand{\CharTok}[1]{\textcolor[rgb]{0.25,0.44,0.63}{{#1}}}
    \newcommand{\StringTok}[1]{\textcolor[rgb]{0.25,0.44,0.63}{{#1}}}
    \newcommand{\CommentTok}[1]{\textcolor[rgb]{0.38,0.63,0.69}{\textit{{#1}}}}
    \newcommand{\OtherTok}[1]{\textcolor[rgb]{0.00,0.44,0.13}{{#1}}}
    \newcommand{\AlertTok}[1]{\textcolor[rgb]{1.00,0.00,0.00}{\textbf{{#1}}}}
    \newcommand{\FunctionTok}[1]{\textcolor[rgb]{0.02,0.16,0.49}{{#1}}}
    \newcommand{\RegionMarkerTok}[1]{{#1}}
    \newcommand{\ErrorTok}[1]{\textcolor[rgb]{1.00,0.00,0.00}{\textbf{{#1}}}}
    \newcommand{\NormalTok}[1]{{#1}}
    
    % Define a nice break command that doesn't care if a line doesn't already
    % exist.
    \def\br{\hspace*{\fill} \\* }
    % Math Jax compatability definitions
    \def\gt{>}
    \def\lt{<}
    % Document parameters
    \title{Python-TP2-2014}
    
    
    

    % Pygments definitions
    
\makeatletter
\def\PY@reset{\let\PY@it=\relax \let\PY@bf=\relax%
    \let\PY@ul=\relax \let\PY@tc=\relax%
    \let\PY@bc=\relax \let\PY@ff=\relax}
\def\PY@tok#1{\csname PY@tok@#1\endcsname}
\def\PY@toks#1+{\ifx\relax#1\empty\else%
    \PY@tok{#1}\expandafter\PY@toks\fi}
\def\PY@do#1{\PY@bc{\PY@tc{\PY@ul{%
    \PY@it{\PY@bf{\PY@ff{#1}}}}}}}
\def\PY#1#2{\PY@reset\PY@toks#1+\relax+\PY@do{#2}}

\expandafter\def\csname PY@tok@gd\endcsname{\def\PY@tc##1{\textcolor[rgb]{0.63,0.00,0.00}{##1}}}
\expandafter\def\csname PY@tok@gu\endcsname{\let\PY@bf=\textbf\def\PY@tc##1{\textcolor[rgb]{0.50,0.00,0.50}{##1}}}
\expandafter\def\csname PY@tok@gt\endcsname{\def\PY@tc##1{\textcolor[rgb]{0.00,0.27,0.87}{##1}}}
\expandafter\def\csname PY@tok@gs\endcsname{\let\PY@bf=\textbf}
\expandafter\def\csname PY@tok@gr\endcsname{\def\PY@tc##1{\textcolor[rgb]{1.00,0.00,0.00}{##1}}}
\expandafter\def\csname PY@tok@cm\endcsname{\let\PY@it=\textit\def\PY@tc##1{\textcolor[rgb]{0.25,0.50,0.50}{##1}}}
\expandafter\def\csname PY@tok@vg\endcsname{\def\PY@tc##1{\textcolor[rgb]{0.10,0.09,0.49}{##1}}}
\expandafter\def\csname PY@tok@m\endcsname{\def\PY@tc##1{\textcolor[rgb]{0.40,0.40,0.40}{##1}}}
\expandafter\def\csname PY@tok@mh\endcsname{\def\PY@tc##1{\textcolor[rgb]{0.40,0.40,0.40}{##1}}}
\expandafter\def\csname PY@tok@go\endcsname{\def\PY@tc##1{\textcolor[rgb]{0.53,0.53,0.53}{##1}}}
\expandafter\def\csname PY@tok@ge\endcsname{\let\PY@it=\textit}
\expandafter\def\csname PY@tok@vc\endcsname{\def\PY@tc##1{\textcolor[rgb]{0.10,0.09,0.49}{##1}}}
\expandafter\def\csname PY@tok@il\endcsname{\def\PY@tc##1{\textcolor[rgb]{0.40,0.40,0.40}{##1}}}
\expandafter\def\csname PY@tok@cs\endcsname{\let\PY@it=\textit\def\PY@tc##1{\textcolor[rgb]{0.25,0.50,0.50}{##1}}}
\expandafter\def\csname PY@tok@cp\endcsname{\def\PY@tc##1{\textcolor[rgb]{0.74,0.48,0.00}{##1}}}
\expandafter\def\csname PY@tok@gi\endcsname{\def\PY@tc##1{\textcolor[rgb]{0.00,0.63,0.00}{##1}}}
\expandafter\def\csname PY@tok@gh\endcsname{\let\PY@bf=\textbf\def\PY@tc##1{\textcolor[rgb]{0.00,0.00,0.50}{##1}}}
\expandafter\def\csname PY@tok@ni\endcsname{\let\PY@bf=\textbf\def\PY@tc##1{\textcolor[rgb]{0.60,0.60,0.60}{##1}}}
\expandafter\def\csname PY@tok@nl\endcsname{\def\PY@tc##1{\textcolor[rgb]{0.63,0.63,0.00}{##1}}}
\expandafter\def\csname PY@tok@nn\endcsname{\let\PY@bf=\textbf\def\PY@tc##1{\textcolor[rgb]{0.00,0.00,1.00}{##1}}}
\expandafter\def\csname PY@tok@no\endcsname{\def\PY@tc##1{\textcolor[rgb]{0.53,0.00,0.00}{##1}}}
\expandafter\def\csname PY@tok@na\endcsname{\def\PY@tc##1{\textcolor[rgb]{0.49,0.56,0.16}{##1}}}
\expandafter\def\csname PY@tok@nb\endcsname{\def\PY@tc##1{\textcolor[rgb]{0.00,0.50,0.00}{##1}}}
\expandafter\def\csname PY@tok@nc\endcsname{\let\PY@bf=\textbf\def\PY@tc##1{\textcolor[rgb]{0.00,0.00,1.00}{##1}}}
\expandafter\def\csname PY@tok@nd\endcsname{\def\PY@tc##1{\textcolor[rgb]{0.67,0.13,1.00}{##1}}}
\expandafter\def\csname PY@tok@ne\endcsname{\let\PY@bf=\textbf\def\PY@tc##1{\textcolor[rgb]{0.82,0.25,0.23}{##1}}}
\expandafter\def\csname PY@tok@nf\endcsname{\def\PY@tc##1{\textcolor[rgb]{0.00,0.00,1.00}{##1}}}
\expandafter\def\csname PY@tok@si\endcsname{\let\PY@bf=\textbf\def\PY@tc##1{\textcolor[rgb]{0.73,0.40,0.53}{##1}}}
\expandafter\def\csname PY@tok@s2\endcsname{\def\PY@tc##1{\textcolor[rgb]{0.73,0.13,0.13}{##1}}}
\expandafter\def\csname PY@tok@vi\endcsname{\def\PY@tc##1{\textcolor[rgb]{0.10,0.09,0.49}{##1}}}
\expandafter\def\csname PY@tok@nt\endcsname{\let\PY@bf=\textbf\def\PY@tc##1{\textcolor[rgb]{0.00,0.50,0.00}{##1}}}
\expandafter\def\csname PY@tok@nv\endcsname{\def\PY@tc##1{\textcolor[rgb]{0.10,0.09,0.49}{##1}}}
\expandafter\def\csname PY@tok@s1\endcsname{\def\PY@tc##1{\textcolor[rgb]{0.73,0.13,0.13}{##1}}}
\expandafter\def\csname PY@tok@sh\endcsname{\def\PY@tc##1{\textcolor[rgb]{0.73,0.13,0.13}{##1}}}
\expandafter\def\csname PY@tok@sc\endcsname{\def\PY@tc##1{\textcolor[rgb]{0.73,0.13,0.13}{##1}}}
\expandafter\def\csname PY@tok@sx\endcsname{\def\PY@tc##1{\textcolor[rgb]{0.00,0.50,0.00}{##1}}}
\expandafter\def\csname PY@tok@bp\endcsname{\def\PY@tc##1{\textcolor[rgb]{0.00,0.50,0.00}{##1}}}
\expandafter\def\csname PY@tok@c1\endcsname{\let\PY@it=\textit\def\PY@tc##1{\textcolor[rgb]{0.25,0.50,0.50}{##1}}}
\expandafter\def\csname PY@tok@kc\endcsname{\let\PY@bf=\textbf\def\PY@tc##1{\textcolor[rgb]{0.00,0.50,0.00}{##1}}}
\expandafter\def\csname PY@tok@c\endcsname{\let\PY@it=\textit\def\PY@tc##1{\textcolor[rgb]{0.25,0.50,0.50}{##1}}}
\expandafter\def\csname PY@tok@mf\endcsname{\def\PY@tc##1{\textcolor[rgb]{0.40,0.40,0.40}{##1}}}
\expandafter\def\csname PY@tok@err\endcsname{\def\PY@bc##1{\setlength{\fboxsep}{0pt}\fcolorbox[rgb]{1.00,0.00,0.00}{1,1,1}{\strut ##1}}}
\expandafter\def\csname PY@tok@kd\endcsname{\let\PY@bf=\textbf\def\PY@tc##1{\textcolor[rgb]{0.00,0.50,0.00}{##1}}}
\expandafter\def\csname PY@tok@ss\endcsname{\def\PY@tc##1{\textcolor[rgb]{0.10,0.09,0.49}{##1}}}
\expandafter\def\csname PY@tok@sr\endcsname{\def\PY@tc##1{\textcolor[rgb]{0.73,0.40,0.53}{##1}}}
\expandafter\def\csname PY@tok@mo\endcsname{\def\PY@tc##1{\textcolor[rgb]{0.40,0.40,0.40}{##1}}}
\expandafter\def\csname PY@tok@kn\endcsname{\let\PY@bf=\textbf\def\PY@tc##1{\textcolor[rgb]{0.00,0.50,0.00}{##1}}}
\expandafter\def\csname PY@tok@mi\endcsname{\def\PY@tc##1{\textcolor[rgb]{0.40,0.40,0.40}{##1}}}
\expandafter\def\csname PY@tok@gp\endcsname{\let\PY@bf=\textbf\def\PY@tc##1{\textcolor[rgb]{0.00,0.00,0.50}{##1}}}
\expandafter\def\csname PY@tok@o\endcsname{\def\PY@tc##1{\textcolor[rgb]{0.40,0.40,0.40}{##1}}}
\expandafter\def\csname PY@tok@kr\endcsname{\let\PY@bf=\textbf\def\PY@tc##1{\textcolor[rgb]{0.00,0.50,0.00}{##1}}}
\expandafter\def\csname PY@tok@s\endcsname{\def\PY@tc##1{\textcolor[rgb]{0.73,0.13,0.13}{##1}}}
\expandafter\def\csname PY@tok@kp\endcsname{\def\PY@tc##1{\textcolor[rgb]{0.00,0.50,0.00}{##1}}}
\expandafter\def\csname PY@tok@w\endcsname{\def\PY@tc##1{\textcolor[rgb]{0.73,0.73,0.73}{##1}}}
\expandafter\def\csname PY@tok@kt\endcsname{\def\PY@tc##1{\textcolor[rgb]{0.69,0.00,0.25}{##1}}}
\expandafter\def\csname PY@tok@ow\endcsname{\let\PY@bf=\textbf\def\PY@tc##1{\textcolor[rgb]{0.67,0.13,1.00}{##1}}}
\expandafter\def\csname PY@tok@sb\endcsname{\def\PY@tc##1{\textcolor[rgb]{0.73,0.13,0.13}{##1}}}
\expandafter\def\csname PY@tok@k\endcsname{\let\PY@bf=\textbf\def\PY@tc##1{\textcolor[rgb]{0.00,0.50,0.00}{##1}}}
\expandafter\def\csname PY@tok@se\endcsname{\let\PY@bf=\textbf\def\PY@tc##1{\textcolor[rgb]{0.73,0.40,0.13}{##1}}}
\expandafter\def\csname PY@tok@sd\endcsname{\let\PY@it=\textit\def\PY@tc##1{\textcolor[rgb]{0.73,0.13,0.13}{##1}}}

\def\PYZbs{\char`\\}
\def\PYZus{\char`\_}
\def\PYZob{\char`\{}
\def\PYZcb{\char`\}}
\def\PYZca{\char`\^}
\def\PYZam{\char`\&}
\def\PYZlt{\char`\<}
\def\PYZgt{\char`\>}
\def\PYZsh{\char`\#}
\def\PYZpc{\char`\%}
\def\PYZdl{\char`\$}
\def\PYZhy{\char`\-}
\def\PYZsq{\char`\'}
\def\PYZdq{\char`\"}
\def\PYZti{\char`\~}
% for compatibility with earlier versions
\def\PYZat{@}
\def\PYZlb{[}
\def\PYZrb{]}
\makeatother


    % Exact colors from NB
    \definecolor{incolor}{rgb}{0.0, 0.0, 0.5}
    \definecolor{outcolor}{rgb}{0.545, 0.0, 0.0}



    
    % Prevent overflowing lines due to hard-to-break entities
    \sloppy 
    % Setup hyperref package
    \hypersetup{
      breaklinks=true,  % so long urls are correctly broken across lines
      colorlinks=true,
      urlcolor=blue,
      linkcolor=darkorange,
      citecolor=darkgreen,
      }
    % Slightly bigger margins than the latex defaults
    
    \geometry{verbose,tmargin=1in,bmargin=1in,lmargin=1in,rmargin=1in}
    
    

    \begin{document}
    
    
    \maketitle
    
    

    

    \subparagraph{TP 2 -- Informatique CM3 -- Septembre 2014}



    \section{Python @ Polytech'Lille}


    Le texte de cette sessions de travaux pratiques est également disponible
ici
http://nbviewer.ipython.org/github/ecalzavarini/python-at-polytech-lille/blob/master/Python-TP2-2014.ipynb


    \subparagraph{Gestion des accents en Python}


    Dans la première séance de TP, vous avez souvent rencontré un message
d'erreure issue de l'utilisation des caractères accentués dans les
scripts en Python. Voici une solution à ce problème. En première ligne
d'un script, il faut insérer la ligne :

    \begin{Verbatim}[commandchars=\\\{\}]
{\color{incolor}In [{\color{incolor}1}]:} \PY{c}{\PYZsh{} \PYZhy{}*\PYZhy{} coding: utf\PYZhy{}8 \PYZhy{}*\PYZhy{}}
\end{Verbatim}

    ou bien

    \begin{Verbatim}[commandchars=\\\{\}]
{\color{incolor}In [{\color{incolor}2}]:} \PY{c}{\PYZsh{} \PYZhy{}*\PYZhy{} coding: latin\PYZhy{}1 \PYZhy{}*\PYZhy{}}
\end{Verbatim}

    Il s'agit d'un pseudo-commentaire indiquant à Python le système de
codage utilisé, ici l'utf-8 (ou le latin-1) qui comprend les caractères
spéciaux comme les caractères accentués, les apostrophes, les cédilles ,
etc .


    \subparagraph{Utiliser les fonctions en Python}


    Vous savez déjà ce qui est une fonction mathématique d'une variable
réelle Considérons par exemple la fonction f(x) suivante définie:

f : x ---\textgreater{} 2 x + 1

pour la définir en Python :

    \begin{Verbatim}[commandchars=\\\{\}]
{\color{incolor}In [{\color{incolor}3}]:} \PY{c}{\PYZsh{}definition d\PYZsq{}une fonction}
        \PY{k}{def} \PY{n+nf}{f}\PY{p}{(}\PY{n}{x}\PY{p}{)}\PY{p}{:}
            \PY{k}{return} \PY{l+m+mi}{2} \PY{o}{*} \PY{n}{x} \PY{o}{+} \PY{l+m+mi}{1}
        
        \PY{c}{\PYZsh{}utilisation de la fonction}
        
        \PY{k}{print}\PY{p}{(}\PY{n}{f}\PY{p}{(}\PY{l+m+mi}{4}\PY{p}{)}\PY{p}{)}
\end{Verbatim}

    \begin{Verbatim}[commandchars=\\\{\}]
9
    \end{Verbatim}

    Avez-vous remarqué qu'à la deuxième ligne, on n'a pas commencé à écrire
au début de la ligne? De la même façon que pour les structures de
contrôle `if' ou `for', nous devons appliquer la règle d'indentation.
Cette indentation est indispensable pour que l'interpréteur Python
comprenne la fin d'un bloc de definition d'une fonction.

    Le principe de définition de fonctions est intéressant pour deux raisons
:

\begin{enumerate}
\def\labelenumi{\arabic{enumi})}
\item
  cela nous permet de ne pas répéter un calcul long à taper,
\item
  Python possède un type spécial dédié au fonctions, que l'on peut donc
  manipuler, mettre dans des listes pour les étudier les unes à la suite
  des autres.
\end{enumerate}

Par exemple :

    \begin{Verbatim}[commandchars=\\\{\}]
{\color{incolor}In [{\color{incolor}4}]:} \PY{k}{print}\PY{p}{(} \PY{n}{f}\PY{p}{(}\PY{l+m+mi}{1}\PY{p}{)} \PY{p}{,} \PY{n}{f}\PY{p}{(}\PY{l+m+mi}{2}\PY{p}{)} \PY{p}{,} \PY{n}{f}\PY{p}{(}\PY{l+m+mi}{3}\PY{p}{)} \PY{p}{,} \PY{n}{f}\PY{p}{(}\PY{l+m+mi}{4}\PY{p}{)} \PY{p}{,} \PY{n}{f}\PY{p}{(}\PY{l+m+mi}{5}\PY{p}{)} \PY{p}{,} \PY{n}{f}\PY{p}{(}\PY{l+m+mi}{6}\PY{p}{)}\PY{p}{)}
\end{Verbatim}

    \begin{Verbatim}[commandchars=\\\{\}]
(3, 5, 7, 9, 11, 13)
    \end{Verbatim}

    \begin{Verbatim}[commandchars=\\\{\}]
{\color{incolor}In [{\color{incolor}5}]:} \PY{k}{print}\PY{p}{(} \PY{n+nb}{type}\PY{p}{(}\PY{n}{f}\PY{p}{)}\PY{p}{)}
\end{Verbatim}

    \begin{Verbatim}[commandchars=\\\{\}]
<type 'function'>
    \end{Verbatim}

    \begin{Verbatim}[commandchars=\\\{\}]
{\color{incolor}In [{\color{incolor}6}]:} \PY{c}{\PYZsh{} definition d\PYZsq{}une seconde fonction}
        \PY{k}{def} \PY{n+nf}{hi}\PY{p}{(}\PY{n}{name}\PY{p}{)}\PY{p}{:}
            \PY{k}{print}\PY{p}{(}\PY{l+s}{\PYZdq{}}\PY{l+s}{hello }\PY{l+s}{\PYZdq{}} \PY{o}{+} \PY{n}{name} \PY{o}{+} \PY{l+s}{\PYZdq{}}\PY{l+s}{ from Python!!!}\PY{l+s}{\PYZdq{}}\PY{p}{)}
        
        \PY{c}{\PYZsh{}exemple d\PYZsq{}utilisation    }
        \PY{n}{hi}\PY{p}{(}\PY{l+s}{\PYZdq{}}\PY{l+s}{Mark}\PY{l+s}{\PYZdq{}}\PY{p}{)}    
\end{Verbatim}

    \begin{Verbatim}[commandchars=\\\{\}]
hello Mark from Python!!!
    \end{Verbatim}

    \begin{Verbatim}[commandchars=\\\{\}]
{\color{incolor}In [{\color{incolor}7}]:} \PY{c}{\PYZsh{}definition  d\PYZsq{}une trosieme fonction}
        \PY{k+kn}{from} \PY{n+nn}{random} \PY{k+kn}{import} \PY{n}{choice}
        \PY{k}{def} \PY{n+nf}{lettre}\PY{p}{(}\PY{p}{)}\PY{p}{:}
            \PY{k}{return} \PY{n}{choice}\PY{p}{(}\PY{l+s}{\PYZsq{}}\PY{l+s}{abcdefghijklmnopqrstuvwxyz}\PY{l+s}{\PYZsq{}}\PY{p}{)}
        
        \PY{c}{\PYZsh{}exemple d\PYZsq{}utilisation }
        \PY{n}{lettre}\PY{p}{(}\PY{p}{)}
\end{Verbatim}

            \begin{Verbatim}[commandchars=\\\{\}]
{\color{outcolor}Out[{\color{outcolor}7}]:} 'i'
\end{Verbatim}
        
    \begin{Verbatim}[commandchars=\\\{\}]
{\color{incolor}In [{\color{incolor}8}]:} \PY{c}{\PYZsh{}definition d\PYZsq{}une liste de fonctions}
        \PY{n}{mes\PYZus{}fonctions} \PY{o}{=} \PY{p}{[}\PY{n}{f}\PY{p}{,}\PY{n}{hi}\PY{p}{,}\PY{n}{lettre}\PY{p}{]}
        
        \PY{c}{\PYZsh{}utilisation}
        \PY{n}{mes\PYZus{}fonctions}\PY{p}{[}\PY{l+m+mi}{1}\PY{p}{]}\PY{p}{(}\PY{l+s}{\PYZdq{}}\PY{l+s}{Dominique}\PY{l+s}{\PYZdq{}}\PY{p}{)}
        
        \PY{n}{mes\PYZus{}fonctions}\PY{p}{[}\PY{l+m+mi}{2}\PY{p}{]}\PY{p}{(}\PY{p}{)}
\end{Verbatim}

    \begin{Verbatim}[commandchars=\\\{\}]
hello Dominique from Python!!!
    \end{Verbatim}

            \begin{Verbatim}[commandchars=\\\{\}]
{\color{outcolor}Out[{\color{outcolor}8}]:} 'n'
\end{Verbatim}
        

    \subsubsection{Script 1 : calcul des forces sur un ballon de football}


    \begin{Verbatim}[commandchars=\\\{\}]
{\color{incolor}In [{\color{incolor}9}]:} \PY{k+kn}{from} \PY{n+nn}{IPython.display} \PY{k+kn}{import} \PY{n}{Image}
        \PY{n}{Image}\PY{p}{(}\PY{n}{filename}\PY{o}{=}\PY{l+s}{\PYZsq{}}\PY{l+s}{kick.jpg}\PY{l+s}{\PYZsq{}}\PY{p}{)}
\end{Verbatim}
\texttt{\color{outcolor}Out[{\color{outcolor}9}]:}
    
    \begin{center}
    \adjustimage{max size={0.9\linewidth}{0.9\paperheight}}{Python-TP2-2014_files/Python-TP2-2014_20_0.jpeg}
    \end{center}
    { \hspace*{\fill} \\}
    

    Les forces sur un ballon de football en vol suite à un coup de pied d'un
joueur sont deux: la force de la pesanteur (le poids \(F_P\)) et la
force de traînée exercée par le frottement de l'air sur le ballon
(\(F_T\)). Leurs expressions sont les suivantes :

    \[F_P = M\ g\]

\[F_T = 0.5 \ C_D \ \rho \ u^2\ S \]

    Ou \(M\) est la masse du ballon, \(g\) l'accélération de gravité,
\(C_D\) le coefficient de traînée, \(\rho\) la densité massique de
l'air, \(u\) la vitesse du ballon et enfin \(S\) la section du ballon
\(S=\pi\ R^2\) (avec R le rayon).

    Nous demandons d'écrire un script qui tout d'abord demande à
l'utilisateur d'entrer les valeurs du rayon du ballon (en mètres), de la
masse du ballon (en Kg) et de la vitesse du ballon ainsi que l'unité de
mesure pour cette dernière (soit ``m/s'' ou ``Km/h''). Ensuite le script
devra calculer les forces \(F_P\) et \(F_T\), afficher les deux valeurs
ainsi que leur rapport \(F_T/F_P\).

    On demande de faire tout cela à l'aide de quatre fonctions :

\begin{enumerate}
\def\labelenumi{\arabic{enumi})}
\item
  une fonction qui convertit la vitesse de ``km/s'' en ``m/s'' si besoin
\item
  une fonction qui calcule la surface \(S\) du ballon à partir du rayon
  \(R\)
\item
  une fonction qui calcule \(F_T\)
\item
  une fonction qui calcule \(F_P\)
\end{enumerate}

    Les données du problème sont l'accélération de gravité
\(g=9.81 m /s^2\), \(\rho = 1.2 m\ kg/m^{-3}\), C D = 0.2.

Tourner le script plusieurs fois et dire ce qui change dans le rapport
F\_T/F\_P pour des valuers de vitesse croissante et à masse et rayon
fixes.


    \paragraph{Script 2 : etude de la fonction de transfert du système ressort -
amortisseur d'un automobile}


    L'objet de ce script est d'illustrer à l'aide de Python les rôles
respectifs joués par le ressort et l'amortisseur d'un système
automobile. Pour ce faire, nous étudions un ressort couplé à un
amortisseur en parallèle.

Comme dans la figure ci-dessous:

    \begin{Verbatim}[commandchars=\\\{\}]
{\color{incolor}In [{\color{incolor}10}]:} \PY{k+kn}{from} \PY{n+nn}{IPython.display} \PY{k+kn}{import} \PY{n}{Image}
         \PY{n}{Image}\PY{p}{(}\PY{n}{filename}\PY{o}{=}\PY{l+s}{\PYZsq{}}\PY{l+s}{masse\PYZhy{}ressort\PYZhy{}amortisseur.jpg}\PY{l+s}{\PYZsq{}}\PY{p}{)}
\end{Verbatim}
\texttt{\color{outcolor}Out[{\color{outcolor}10}]:}
    
    \begin{center}
    \adjustimage{max size={0.9\linewidth}{0.9\paperheight}}{Python-TP2-2014_files/Python-TP2-2014_29_0.jpeg}
    \end{center}
    { \hspace*{\fill} \\}
    

    L'équation vérifiée par le système est la suivante :

\[ \ddot{x} + \frac{\omega_0}{Q} \dot{x} + \omega_0^2\ x = \frac{F}{m} \cos( \omega t) \]

    Ici \(m\) est la masse de la roue, \(F\) et \(\omega\) sont
respectivement l'intensité de la force appliquée et sa pulsation,
\(\omega_0\) la fréquence caractéristique du ressort et enfin le
coefficient \(Q\) (dit coefficient de qualité).

Un système très amorti a un Q faible. À l'inverse, un Q élevé correspond
à un système peu amorti. Pour fixer les idées, le Q d'une voiture avec
des amortisseurs en bon état est légèrement supérieur à 1.

    De la solution de l'equation du système ressort-amortisseur on trouve la
fonction de transfert qui décrit la réponse du systeme en fonction de la
pulsation \(\omega\).

En particulier la fonction de transfert s'écrit :

    \[ T(\omega) = \frac{F}{m \ \omega_0^2} \frac{1}{\sqrt{ \left( 1 - \omega^2/\omega_0^2 \right)^2 +  Q^{-2} \ \omega^2/\omega_0^2 }} \]

    ou en utilisant la pulsation adimensionnée u = ω/ω0 :

    \[ T(u) = \frac{F}{m \ \omega_0^2} \frac{1}{\sqrt{ \left( 1 - u^2 \right)^2 +  u^{2}/ \      Q^2}} \]

    Nous demandons d'écrire un script qui trace un graphique du module de la
fonction de transfert en fonction de la pulsation adimensionnée u pour
toutes les valeurs de Q dans l'intervalle {[}0 , 10{]} avec un
incrementation de \(\delta Q = 2.0\).

    Prendre $ F / (m \omega_0^{2} ) = 2 $

    Ça pourrait être utile :


    \subsubsection{La boucle while}


    Le but de la boucle ``while'' est de répéter certaines instructions tant
qu'une condition est respectée. On n'est pas donc obligé de savoir au
départ le nombre de répétitions à faire.

    \begin{Verbatim}[commandchars=\\\{\}]
{\color{incolor}In [{\color{incolor}11}]:} \PY{n}{nb\PYZus{}repetitions} \PY{o}{=} \PY{l+m+mi}{3}
         \PY{n}{i} \PY{o}{=} \PY{l+m+mi}{1}
         
         \PY{k}{while} \PY{n}{i} \PY{o}{\PYZlt{}}\PY{o}{=} \PY{n}{nb\PYZus{}repetitions} \PY{p}{:}
             \PY{k}{print} \PY{l+s}{\PYZdq{}}\PY{l+s}{Et }\PY{l+s}{\PYZdq{}}\PY{o}{+}\PY{n+nb}{str}\PY{p}{(}\PY{n}{i}\PY{p}{)}\PY{o}{+}\PY{l+s}{\PYZdq{}}\PY{l+s}{!}\PY{l+s}{\PYZdq{}}
             \PY{n}{i} \PY{o}{=} \PY{n}{i}\PY{o}{+}\PY{l+m+mi}{1}
         \PY{k}{print} \PY{l+s}{\PYZdq{}}\PY{l+s}{Zéro!}\PY{l+s}{\PYZdq{}}
\end{Verbatim}

    \begin{Verbatim}[commandchars=\\\{\}]
Et 1!
Et 2!
Et 3!
Zéro!
    \end{Verbatim}

    L'instruction ``break'' sert, non pas à interrompre le programme, mais à
sortir de la boucle.

    \begin{Verbatim}[commandchars=\\\{\}]
{\color{incolor}In [{\color{incolor}12}]:} \PY{n}{nb\PYZus{}repetitions} \PY{o}{=} \PY{l+m+mi}{3}
         \PY{n}{i} \PY{o}{=} \PY{l+m+mi}{1}
         
         \PY{k}{while} \PY{n}{i} \PY{o}{\PYZlt{}}\PY{o}{=} \PY{n}{nb\PYZus{}repetitions} \PY{p}{:}
             \PY{k}{print} \PY{l+s}{\PYZdq{}}\PY{l+s}{Et }\PY{l+s}{\PYZdq{}}\PY{o}{+}\PY{n+nb}{str}\PY{p}{(}\PY{n}{i}\PY{p}{)}\PY{o}{+}\PY{l+s}{\PYZdq{}}\PY{l+s}{!}\PY{l+s}{\PYZdq{}}
             \PY{n}{i} \PY{o}{=} \PY{n}{i}\PY{o}{+}\PY{l+m+mi}{1}
             \PY{k}{if} \PY{n}{i}\PY{o}{==}\PY{l+m+mi}{3}\PY{p}{:} 
                 \PY{k}{break}
         \PY{k}{print} \PY{l+s}{\PYZdq{}}\PY{l+s}{Zéro!}\PY{l+s}{\PYZdq{}}
\end{Verbatim}

    \begin{Verbatim}[commandchars=\\\{\}]
Et 1!
Et 2!
Zéro!
    \end{Verbatim}


    \subsubsection{Script 3}


    Écrire un script qui calcule la valeur critique de Q, c'est-à-dire la
valuer intermédiaire entre le régime des vibrations sur amorties et cel
de résonance.

    Pour faire cela, on doit calculer la valeur maximale de T pour des
valeurs décroissantes de Q, et d'arrêter lorsque T\_\{max\} est égal à F
/ (m ω20).

    Utiliser encore : F / (m ω20)=2.


    \paragraph{Calcul de la valeur maximale avec numpy}


    Pour ce script, il peut être utile d'utiliser la fonction numpy pour
calculer le maximum d'une liste :

    \begin{Verbatim}[commandchars=\\\{\}]
{\color{incolor}In [{\color{incolor}13}]:}  \PY{k+kn}{import} \PY{n+nn}{numpy} \PY{k+kn}{as} \PY{n+nn}{np}
          
          \PY{n}{val}\PY{o}{=}\PY{n}{np}\PY{o}{.}\PY{n}{linspace}\PY{p}{(}\PY{l+m+mi}{0}\PY{p}{,}\PY{l+m+mi}{2}\PY{p}{,}\PY{l+m+mi}{100}\PY{p}{)}
          
          \PY{n}{val\PYZus{}max} \PY{o}{=} \PY{n}{np}\PY{o}{.}\PY{n}{amax}\PY{p}{(}\PY{n}{val}\PY{p}{)}
             
          \PY{k}{print}\PY{p}{(}\PY{n}{val\PYZus{}max}\PY{p}{)}
\end{Verbatim}

    \begin{Verbatim}[commandchars=\\\{\}]
2.0
    \end{Verbatim}

    \begin{Verbatim}[commandchars=\\\{\}]
{\color{incolor}In [{\color{incolor}14}]:} \PY{k+kn}{from} \PY{n+nn}{IPython.core.display} \PY{k+kn}{import} \PY{n}{HTML}
         \PY{k}{def} \PY{n+nf}{css\PYZus{}styling}\PY{p}{(}\PY{p}{)}\PY{p}{:}
             \PY{n}{styles} \PY{o}{=} \PY{n+nb}{open}\PY{p}{(}\PY{l+s}{\PYZsq{}}\PY{l+s}{custom.css}\PY{l+s}{\PYZsq{}}\PY{p}{,} \PY{l+s}{\PYZsq{}}\PY{l+s}{r}\PY{l+s}{\PYZsq{}}\PY{p}{)}\PY{o}{.}\PY{n}{read}\PY{p}{(}\PY{p}{)}
             \PY{k}{return} \PY{n}{HTML}\PY{p}{(}\PY{n}{styles}\PY{p}{)}
         \PY{n}{css\PYZus{}styling}\PY{p}{(}\PY{p}{)}
\end{Verbatim}

            \begin{Verbatim}[commandchars=\\\{\}]
{\color{outcolor}Out[{\color{outcolor}14}]:} <IPython.core.display.HTML at 0x10c0844d0>
\end{Verbatim}
        

    % Add a bibliography block to the postdoc
    
    
    
    \end{document}
